\chapter{Introduction}
\ac{ICD} and \ac{ICD}-like processes are electronic decay processes,
which include neighbouring atoms or molecules \cite{Cederbaum97}.
They can occur after inner-valence ionization and
are most generally described by:

\begin{equation*}
 AB \quad \xrightarrow{h\nu}\quad AB^+ + e^-_{ph} \quad
    \rightarrow \quad A^+ + B^+ + e^-_{ph} + e^-_{sec}
\end{equation*}
Here, a system $AB$ is ionized in the inner-valence and the corresponding
photo-electron is emitted. The ionized system can then electronically
rearrange and as a consequence the system is split into two positively
charged components and a second electron, also called ICD electron, which is
emitted.

This kind of processes occurs in a multitude of system like noble gas clusters
and hydrogen-bonded systems. Furthermore, it has been found to explain
the electron detachment essential in repair of DNA lesions in the
photolyases enzymes \cite{Harbach13}. Furthermore, it is discussed as
source for low kinetic energy electrons in the body after an initiation
by a radioactive decay in the medical treatment of cancer. These low kinetic
energy electron cause
double strand breaks of DNA more efficiently than electrons of higher energies.
None of the competing processes has until now been proven to create these slow
electrons and to explain the locally observed damage \cite{}.

In order to be observable, two criteria, the energy and the coupling criterion,
have to be fulfilled. The energy criterion requires the energy conservation
and therefore, that the energy of the doubly ionized final state is lower than
the energy of the singly ionized initial state. If this is not the case, the
channel at hand is closed and the corresponding fragments of the
channel are not observed after the decay.
The coupling criterion requires the process to be efficient enough to compete
with other decay processes like the Auger decay (an other autoionization
process) and the radiative decay.
Hence, the study of the \ac{ICD}-like processes consists of two parts:
\begin{itemize}
 \item determination of the kinetic energy of the secondary electron
       and hence, which channels are open
 \item calculation of the decay width $\Gamma=\frac{\hbar}{\tau}$, which
       is proportional to the decay rate $\frac{1}{\tau}$ and
       anti-proportional to the lifetime $\tau$
\end{itemize}

Throughout the last decade, such processes have been studied intensively
in experiment 
and in theory using non-relativistic quantum chemistry
(see ref. \cite{Hergenhahn11} and references herein).

However, if heavy elements are involved in these decay processes, relativistic
effects might play an important role. Therefore, the aim of this
thesis is to investigate, how spin-orbit coupling and scalar-relativistic
effects influence both the energies of the involved initial and final states
as well as the decay widths of the processes.
It is going to be shown that the spin-orbit coupling leads to a larger number
of decay channels compared to the non-relativistic description. These channels
might open at different geometries of the system under investigation and
can therefore lead to channel openings at geometries, at which the channel
in the corresponding non-relativistic description would be closed. Additionally,
scalar-relativistic effects shift the energies of the initial and final states.
These shifts are most pronounced in the core region and hence might only play
a minor role for decay processes in the valence. Furthermore, the
quantity of interest is the energy difference between the initial and final
states. If the
energy shift occurs in the same direction, the effect on the measured kinetic
energy of the secondary electron is rather small.

For the calculation of the decay widths, it has to be noted that the type of
wavefunctions is inherently different than those of non-relativistic ones. They
are characterized by the total angular momentum $J$ and its projection $M_J$ rather
than the angular momentum quantum numbers $L,M_L$ and spin quantum numbers
$S,M_S$. Their spatial difference to the non-realtivistic wavefunction allows
for decays, which are non-relativistically forbidden by symmetry. E.g., the
transition between a d$_{3/2}$ and an s$_{1/2}$ state is allowed in the
relativistic description, while it breaks the Laporte rule in the non-relativistic
framework.

Non-relativistically, the decay width has been studied using asymptotic
approximations \cite{Gokhberg10_1} and various quantum chemical methods.
The asymptotic expressions for the decay width of the \ac{ICD} and a competing
\ac{ICD}-like process, the ETMD3 (Electron Transfer Mediated Decay with three
involved units), are derived in this thesis. The basic influence of
spin-orbit coupling on the decay widths of the \ac{ICD} and ETMD3 is studied
for various geometries and initial and final state symmetries.
The quantum chemical methods, which have been used for the non-relativistic 
description of the decay widths are the Wigner-Weisskopf theory \cite{Santra02},
CAP-CI
(Complex Absorbing Potential based on a Configuration Interaction wavefunction)
\cite{SakuraiModern94,Santra01_3} and FanoADC-Stieltjes \cite{Averbukh05}.
While the CAP-CI method is the most precise of the three method of the above,
it at the same
is not size-consistent and requires a huge basis set. Therefore, it is not
suited for the investigation of larger systems. In contrast to this, the
Wigner-Weisskopf theory is based on the lowest non-vanishing order of perturbation
theory and therefore computationally cheap. However, the price for this low
computational cost is the poor accuracy of the results. A compromise between
accuray and computational cost is the FanoADC-Stieltjes approach, which includes
higher perturbational orders and is size-consistent.
Therefore, the FanoADC-Stieltjes method was implemented in the relativistic
quantum chemical programme package \verb|DIRAC| \cite{DIRAC13} and the first results
obtained with this method are to be found in this thesis.

For the experimental validation of the \ac{ICD}-like processes, most often
noble gas clusters of 100--2000 atoms are studied. In order to compare the
theoretically obtained results with the experimental measurements, it has
to be noted that the cluster environment also affects the secondary electron
spectrum. By stabilization of ionized atoms both in the initial and final
states, the kinetic energy spectrum of the secondary electrons is shifted to
larger energies. Hence, additional channel openings might be observed.
Furthermore, the larger number of decay partners increases the decay rate
and for statistic reasons the decay rate for a specific initially ionized
atom in a heteronuclear cluster strongly depends on its position in the cluster.

In order to model the secondary electron spectra of clusters, a method
was developed based on the decomposition of cluster structures into manifolds
of decaying pairs and triples. For these pairs and triples, the kinetic energies
of the secondary electrons and the corresponding decay widths are evaluated based
on either the asymptotic expressions for the decay widths or quantum chemical
calculations for dimers and trimers. For the automatic evaluation of
a large number of cluster structures the programme
\ac{HARDRoC} \cite{HARDRoC} was developed.

The thesis is outlined as follows:

First, basic theory involved in this thesis is discussed. Afterwards, the
\emph{ab initio} methods developed and used are introduced and
finally, the obtained results are presented.
An additional description of the programmes developed in this thesis are to
be found in the appendix.

In the theory part, the large variety of autoionization processes,
especially the \ac{ICD}-like processes, is presented. Then, an introduction
into relativistic quantum chemistry and resonances, which are needed for
the calculation of the decay widths, is given. From the latter, asymptotic
approximations of the decay width for the \ac{ICD} and ETMD3 are derived
in the relativistic framework.
Finally, noble gas clusters and their experimental creation and measurement
are presented.

In the method part, the description of continuum properties with $\mathcal{L}^2$
functions, the \ac{ADC} and especially the FanoADC-Stieltjes approach are
presented.

In the results part, the systems are studied with increasing number of constituents.
First, Auger processes of noble gas atoms are studied for testing purposes
of the relativistic FanoADC-Stieltjes implementation and in order to investigate
basic relativistic effects. Then, channel openings and decay widths for
different geometries of pairs and triples are studied including relativistic
effects. Afterwards, the relative asymptotic decay width behaviour
of different decay channels, which are split due to spin-orbit coupling are
investigated. After this, the decay widths of the ArXe dimer is investigated
using the relativistic FanoADC-Stieltjes approach. Finally, secondary electron
spectra of heteronuclear ArXe and NeAr clusters are studied based on the
results of the preceding chapters.
In the ArXe clusters, relativistic effects, basic effects of the cluster environment
and different cluster structures are investigated.
For the NeAr clusters, a new procedure to determine structural information of mixed
noble gas clusters with competing \ac{ICD} processes is presented.
