\chapter{Introduction}
Electronic vacancies in the sub-outer-valence of an atom or molecule
created by radiation or radioactive decay are excited system.
These can decay via photon emission, coupling to vibrational degrees
of freedom or via electronic decay processes like the Auger
decay \cite{Meitner22,Auger23} or the
manifold of Interatomic / Intermolecular Coulombic Decay (ICD) processes.
The latter was theoretically predicted in 1997 by Cederbaum and involves
neighbouring atoms and molecules in the decay.
It can most generally described by:

\begin{equation*}
 AB \quad \xrightarrow{h\nu}\quad AB^+ + e^-_{ph} \quad
    \xrightarrow{ICD} \quad A^+ + B^+ + e^-_{ph} + e^-_{sec}
\end{equation*}
Here, a system $AB$ is ionized in the inner-valence of atom or molecule $A$
and the corresponding
photo-electron $e^-_{ph}$ is emitted. The ionized system $AB^+$
can then electronically
rearrange and as a consequence the system is split into two positively
charged components $A^+$, $B^+$ and a secondary electron $e^-_{sec}$,
also called ICD electron, which is emitted.

This kind of processes occurs in a multitude of system like noble gas clusters
and hydrogen-bonded systems. Furthermore, it has been found to explain
the electron detachment essential in repair of DNA lesions in the
photolyases enzymes \cite{Harbach13}. It is also discussed as
source for low kinetic energy electrons in the body after an initiation
by a radioactive decay in the medical treatment of cancer
\cite{Kim11, Hergenhahn12, Boudaiffa00, Pan03, Martin04}. These low kinetic
energy electron cause
double strand breaks of DNA more efficiently than electrons of higher energies.
None of the competing processes has until now been proven to create these slow
electrons and to explain the locally observed damage.

This process can occur if,
two criteria, the energy and the coupling criterion,
are fulfilled. To fulfill the energy criterion it is required that
that the energy of the doubly ionized final state excluding the secondary
electron is lower than
the energy of the singly ionized initial state. If this is not the case, the
channel is closed and the corresponding fragments of the
channel are not observed after the decay.
To fulfill the coupling criterion it is required that the process
is efficient enough to compete
with other decay processes like the Auger decay
and the radiative decay.
Hence, a typical study of \ac{ICD}-like processes consists of two parts:
\begin{itemize}
 \item determination of the kinetic energy of the secondary electron
       and as a consequence, which channels are open
 \item calculation of the decay width $\Gamma=\frac{\hbar}{\tau}$, which
       is proportional to the decay rate $\frac{1}{\tau}$ and
       anti-proportional to the lifetime $\tau$
\end{itemize}

Throughout the last two decades, such processes have been studied intensively
in experiment 
and in theory using non-relativistic quantum chemistry
(see the review in Ref. \cite{Hergenhahn11} and references herein).
However, if heavy elements are involved in these decay processes, relativistic
effects are expected tare expected to play an important role.
Therefore, the aim of this
thesis is to investigate, how spin-orbit coupling and scalar-relativistic
effects influence both, the energies of the involved initial and final states
as well as the decay widths of the processes.
It is will be shown that the spin-orbit coupling leads to a larger number
of decay channels compared to the non-relativistic description. These channels
open at different geometries of the system under investigation and
can therefore lead to channel openings at geometries, at which the channel
in the corresponding non-relativistic description would be closed.

Scalar-relativistic effects shift the energies of the initial and final states.
These shifts are most pronounced in the core region and hence might only play
a minor role for decay processes in the valence. Furthermore, the
quantity of interest is the energy difference between the initial and final
states. If the
energy shift occurs in the same direction, the effect on the measured kinetic
energy of the secondary electron is rather small.

For the calculation of initial and final state energies as well as decay widths
using relativistic methods, it has to be noted that the type of
wavefunctions is inherently different than from the non-relativistic case. They
are characterized by the total angular momentum $J$ and its projection $M_J$ rather
than the angular momentum quantum numbers $L,M_L$ and spin quantum numbers
$S,M_S$. The different spatial structure compared to the non-realtivistic
wavefunction allows
for decays, which are non-relativistically forbidden by symmetry. E.g., the
transition between a d$_{3/2}$ and an s$_{1/2}$ state is allowed in the
relativistic description, while it breaks the Laporte rule in the non-relativistic
framework.

The difficulty in the description of the decay width is caused by
the necessity to include both the bound states as well as the freely
moving secondary electron, which have to fulfill different boundary conditions.
Non-relativistically, it has been studied using asymptotic
approximations \cite{Gokhberg10_1} and various quantum chemical methods based
on $\mathcal{L}^2$ functions.
The asymptotic expressions for the decay width of the \ac{ICD} and a competing
\ac{ICD}-like process, the ETMD3 (Electron Transfer Mediated Decay) with three
involved units, are derived in this thesis. The basic influence of
spin-orbit coupling on the decay widths of the \ac{ICD} and ETMD3 is studied
for various geometries and initial and final state configurations.
The quantum chemical methods, which have been used for the non-relativistic 
description of the decay widths are the Wigner-Weisskopf theory \cite{Santra02},
CAP-CI
(Complex Absorbing Potential based on a Configuration Interaction wavefunction)
\cite{SakuraiModern94,Santra01_3} and the FanoADC-Stieltjes method
\cite{Averbukh05}.
While the CAP-CI method is the most precise of the three methods above,
it is at the same
not size-consistent and requires a huge basis set. Therefore, it is not
suited for the investigation of larger systems. In contrast to this, the
Wigner-Weisskopf theory is based on the lowest non-vanishing order of perturbation
theory and therefore computationally affordable even for large systems.
However, the price for the lower
computational costs are less accurate results. A compromise between
accuray and computational cost is the FanoADC-Stieltjes approach, which
is based on the Algebraic Diagrammatic Construction (ADC) and therefore
likewise includes
higher perturbational orders \emph{and} is size-consistent.
For these reasons,
the FanoADC-Stieltjes method was implemented in the relativistic
quantum chemical programme package Dirac \cite{DIRAC13}. First results
obtained with this method are found in this thesis.

For the experimental validation of the \ac{ICD}-like processes, most often
noble gas clusters of 100--2000 atoms are studied, because noble gases can
easily be introduced into the apparatus and cleaning after the experiment is
not necessary. In order to compare the
theoretically obtained results with the experimental measurements of the
kinetic energy of the secondary electron, it has
to be noted that the cluster environment also affects the secondary electron
spectrum.
The ionized constituents of both the initial and the final states are stabilized
by the cluster environment. However, the closer the vacancy is to the nucleus,
the weaker is the interaction with the environment.
Therefore, the stabilization of the ionized atoms in the final state is more
pronounced than in for the initially ionized atom and as a consequence, the
kinetic energy spectrum of the secondary electron is shifted towards higher
energies.
Hence, additional channel openings can be observed.
Furthermore, the larger number of decay partners increases the decay rate
and for statistic reasons the decay rate for a specific initially ionized
atom in a heteronuclear cluster strongly depends on its position in the cluster.

In order to model the secondary electron spectra of clusters, a method
was developed based on the decomposition of cluster structures into manifolds
of decaying pairs and triples. For these pairs and triples, the kinetic energies
of the secondary electrons and the corresponding decay widths are evaluated based
on either the asymptotic expressions for the decay widths or quantum chemical
calculations for dimers and trimers. For the automatic evaluation of
a large number of cluster structures the programme
\ac{HARDRoC} \cite{HARDRoC} was developed.

The thesis consists of three main parts:
First, basic theory involved in this thesis is discussed in part
\ref{part:theory}. In part \ref{part:methodology}, the
\emph{ab initio} methods developed and used are introduced and
finally in part \ref{part:results}, the obtained results are presented.
An additional description of the programmes developed in this thesis are to
be found in the Appendix \ref{app:programmes}.

In the theory part, the large variety of autoionization processes,
especially the \ac{ICD}-like processes, is presented in chapter
\ref{chapter:autoionization}. Then, an introduction
into relativistic quantum chemistry in chapter \ref{chapter:relativistic}
and resonances in chapter \ref{chapter:resonances}, which are needed for
the calculation of the decay widths, is given. From the latter, asymptotic
approximations of the decay width for the \ac{ICD} and ETMD3 are derived
in the relativistic framework in chapter \ref{chapter:asymptotic}.
Finally, noble gas clusters and their experimental creation and measurement
are presented in chapter \ref{chapter:clusters}.

In the method part, the description of continuum properties with $\mathcal{L}^2$
functions, the \ac{ADC} and especially the FanoADC-Stieltjes approach are
presented.

In the results part, the systems are studied with increasing number of constituents.
First, in chapter \ref{chapter:res_auger}, Auger processes of noble gas
atoms are studied as the smallest systems undergoing electronic decay
processes for testing purposes
of the relativistic FanoADC-Stieltjes implementation. Hereby, basic influences
of the relativistic effects on the autoionization processes are already
investigated.
Then, channel openings and decay widths for
different geometries of pairs and triples are studied including relativistic
effects in chapter \ref{chapter:geom}. Afterwards, the relative asymptotic
decay width behaviour
of different decay channels, which are split due to spin-orbit coupling are
investigated in chapter \ref{chapter:res_symmetry}.
Based on the relativistic FanoADC-Stieltjes approach validated for the
Auger process, the decay widths of the ArXe dimer is
investigated using the relativistic FanoADC-Stieltjes approach in chapter
\ref{chapter:fano_arxe}. Finally, in chapter \ref{chapter_clusters},
secondary electron
spectra of heteronuclear ArXe and NeAr clusters are studied based on the
results of the preceding chapters.
In the ArXe clusters, relativistic effects, basic effects of the cluster environment
and different cluster structures are investigated.
For the NeAr clusters, a new procedure to determine structural information of mixed
noble gas clusters with competing \ac{ICD} processes is presented.

All this shows that a relativistic treatment for heavy processes involving
heavy atoms and consideration of neighbour arrangements for the explanation
of noble gas clusters are inherently important.
