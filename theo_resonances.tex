\section{Resonances}

Classically, a resonance is the maximum coupling of two or more
oscillating systems, in which energy is either transferred from 
one system to another or converted into a different kind of energy.
In quantum mechanics analogously the maximum coupling of an initial 
state to either 
another state or to
a continuum of states is referred to as a resonance. Hereby this
interaction can be mediated via an external field.
In consequence of the coupling the system evolves over time from
the initial state into at least one other state. Depending on the investigated
process, this change is either reversible like in Rabi oscillations \cite{Rabi}
or irreversible like in our case of autoionization processes.

The time evolution of a resonance process can be characterized by
the averaged lifetime $\tau$ of the initial state or its counterpart
in the energy domain,
the averaged decay width $\Gamma = \frac{\hbar}{\tau}$.

Historically the underlying theory was developed by numerous scienctists
like Dirac \cite{Dirac27_1,Dirac27_2}, Wentzel \cite{Wentzel27}, 
Weisskopf and Wigner \cite{Weisskopf30},
Fermi \cite{Fermi32}, Breit and Wigner \cite{Breit36},
Kapur and Peierls \cite{Kapur38}, Siegert \cite{Siegert39} and many others.

In the following we are going to discuss the necessary basics of scattering
theory as summarized by Gell-Mann \cite{Gell-Mann53}.


%Its time evolution and optionally the corresponding decay width has historically
%been investigated using either perturbation \cite{} or scattering
%theory \cite{} until it became part of Feshbach's unified theory of
%nuclear reactions \cite{Feshbach58,Feshbach62}. This completely general theory
%is based on projection operators partitioning the Hilbert space into subspaces of
%initial and final states.
%In contrast to earlier approaches, it holds for all coupling schemes as well as
%all quantum numbers. They will be taken care of in the definition of the
%projection operators.
%Shortly after,
%Fano amplified the latter ansatz to describe excitation spectra, which
%are inverse processes to Feshbach's nuclear reactions.\cite{Fano61}


We start from the time dependent Schroedinger equation

\begin{equation}
  i \frac{\mathrm{d}}{\mathrm{d}t} \Psi(t) = (K + V) \Psi(t) ,
\end{equation}
where $K$ denotes the Hamilton operator of non-interacting colliding
particles, or in our case initial and final states. Its solution
$\Phi_i(t) = \phi_i e^{-iE_it}$ are stationary states of the system.

\begin{equation}
  i \frac{\mathrm{d}}{\mathrm{d}t} \Phi(t) = K \Phi(t)
\end{equation}

Out purpose it the description of the transition rates from an initial state
$\Phi_i(t)$ to a final state $\Phi_f(t)$ mediated by the interaction $V$ between
them. We are going to achieve this by taking the time derivative of the system's
probability $\omega_{fi}(t)$ to be in a certain final state at time t.

\begin{align}
  w_{fi}(t) &= \frac 1{N_i} |f_{fi}|^2 \\
  f_{fi}(t) &= \braket{\Phi_f(t)|\Psi_i(t)} \label{equation:scattering_overlap}\\
  N_i       &= \braket{\Psi_i(t)|\Psi_i(t)}
\end{align}

In eq. (\ref{equation:scattering_overlap}) not the stationary state $\Phi_i$
has been chosen to represent the initial state but $\Psi_i$. Since our knowledge
about the initial state wave function is limited to its behaviour without the
interaction $V$, being turned on at $t=0$, we have to describe the initial state
wave function at some time in the distant past $T<0$ and propagate it until $t=0$.
Therefore the question arises, which time $T$ should be selected for this purpose.
Since no time is better than any other and the result might depend on the decision,
one averages over propagations starting at different times $T$.

\begin{equation}
  \Psi_i(t) = \frac 1\tau \int\limits_{-\tau}^0 \mathrm{d}T \,
              e^{-iH(t-T)} \, \Phi_i(T)
\end{equation}
Here $\tau$ is allowed to approach $+\infty$ at the end of the calculation.

A more convenient way to include this ansatz in the further derivation
is its Fourier transformation
\begin{equation}
  \Psi_i(t) = \varepsilon \int\limits_{-\infty}^0 \mathrm{d}T \,
              e^{\varepsilon T} e^{-iH(t-T)} \Phi_i(T)
\end{equation}

with $\varepsilon = \tau^{-1}$. Evaluating the integral, this leads to:
\begin{align}
   \Psi_i(t) &= \varepsilon \, e^{-iHt} \int\limits_{-\infty}^0 \mathrm{d}T \,
                e^{\varepsilon T} e^{i(H-E_i)T} \phi_i\\
             &= e^{-iHt} \frac{\varepsilon}{\varepsilon+i(H-E_j)} \phi_i .
\end{align}

Using the Schroedinger equation
\begin{equation}
  (H-E_i) \phi_i = V \phi_i
\end{equation}

of the whole system, one easily arrives at an expression for the initial state
at time $t=0$.
\begin{align}
  \Psi_i(0) &= \frac{\varepsilon + i(H-E_i) - i(H-E_i)}{\varepsilon + i(H-E-i)} \phi_i\\
            &= \phi_i + \frac{1}{E_i-H+i\varepsilon} V \phi_i \label{equation:in_state_0}\\
            &\approx \phi_i + \frac{1}{E_i-K+i\varepsilon} V \Psi_i(0) \label{equation:in_state_0_approx}
\end{align}

The latter equation holds for a small perturbation $V$ as can be seen by comparing
the power expansions of equations (\ref{equation:in_state_0}) and
(\ref{equation:in_state_0_approx}).\\
Since the norm $N_{fi}$ is time-independent, we now have
everything we need for the determination of the decay rate.
Inserting equation (\ref{equation:in_state_0_approx}) into equation
(\ref{equation:scattering_overlap}) we evaluate the overlap between the initial
and the final state.
\begin{equation}
  f_{fi}(0) = \delta_{fi} + \frac{1}{E_i-E_f+i\varepsilon} R_{fi}(\varepsilon)
\end{equation}

Here, 
\begin{equation}
  R_{fi}(\varepsilon) = \braket{\phi_f|V|\Psi_i(0)}
\end{equation}
denotes the coupling of the initial with the final state via the interaction
operator $V$.

The time dependence of coupling between the two time-independent states is now
introduced to give:
\begin{equation}
  f_{fi}(t) = \braket{\phi_f| e^{i(E_f-H)t} |\Psi_i(0)} .
\end{equation}

Its absolute square is proportional to the probability of the system to be
in the final state $\phi_f$ and its time derivative at $t=0$ yields
\begin{equation}
  \left . \frac{\mathrm{d}}{\mathrm{d}t} |f_{fi}|^2 \right |_{t=0}
  = 2\delta_{fi} \operatorname{Im}R_{ii}(\varepsilon) 
    + \frac{2\varepsilon}{(E_i-E_f)^2+\varepsilon^2} |R_{fi}(\varepsilon)|^2 .
\end{equation}

It consists of two parts, the first is propotional to the probability to stay
in the initial state and the second one describes the transition into the
final state. The latter has the typical Lorentzian shape with the full-width-half-maximum
(FWHM) $\varepsilon$ which from now on will be called the decay width $\Gamma$.
The larger the width is, the faster is the transition into the final state.

Feshbach's derivation starts from the Schroedinger equation of the total
system under investigation

\begin{equation}
  H \Psi = E \Psi \label{schroedinger}
\end{equation}

and projection operators $P$ and $Q$. $P$ projects the final state out
of the total wavefunction $\Psi$ and is defined with respect to eigenstates
of the system in the asymptotic time limit, which means long after the process
itself finished. $Q$ is analogously defined with respect to the rest of the
system as $Q = 1 - P$. Therefore after insertion to eq. \ref{schroedinger}

\begin{equation}
  H (P+Q) \Psi = E \Psi
\end{equation}

\begin{align}
  (E - H_{PP}) P \Psi & = H_{PQ} Q \Psi \label{se_PP}\\
  (E - H_{QQ}) Q \Psi & = H_{QP} P \Psi \label{se_QQ}
\end{align}

can easily be derived with

\begin{align*}
  H_{PP} & \equiv PHP & \quad\quad H_{PQ} & \equiv PHQ\\
  H_{QP} & \equiv QHP & \quad\quad H_{QQ} & \equiv QHQ .
\end{align*}

From equation \ref{se_QQ} a straigth-forward solution for the system excluding
the selected final states can be found.

\begin{equation}
  Q \Psi = \frac{1}{E^{(+)}-H_{QQ}} H_{QP} P \Psi \label{feshbach_qpsi}
\end{equation}

The latter expression holds in case of selectively chosen open channels, which
do not represent all open channels. In case of all open channels being selected,
$E^{(+)}$ is to be substituted by $E$.

After insertion of eq. \ref{feshbach_qpsi} into eq. \ref{se_PP} one arrives at

\begin{equation}
  \mathscr{H} \,P \Psi = E \,P \Psi \label{se_ppsi}
\end{equation}

with $\mathscr{H}$ being the effective Hamiltonian of the final states.
\begin{equation}
  \mathscr{H} = H_{PP} + H_{PQ} \frac{1}{E^{(+)}-H_{QQ}} H_{QP}
\end{equation}

In order to solve these expressions we define $\{\Phi_n\}$ to be the solutions
of the Hamiltonian excluding the final state solutions or the closed channels
solutions in case of all open channels being defined as final states.
They are assumed to be bound and to fulfill the
Schroedinger equation

\begin{equation}
  (\varepsilon_n - H_{QQ}) \Phi_n = 0 .
\end{equation}

This approach is not exact, because the states being bound implicate
their lifetimes to be infinite, which they are intrinsically
to the problem not supposed to be. However, for states having a long lifetime,
this approximation is reasonable. -> How long is long?

Together with a set of continuum wavefunctions $\{\Phi(\alpha,E)\}$, they are
defined to fulfill the following orthogonality relations

\begin{align}
  \braket{\Phi_n|\Phi_n} = 1 \quad  & \quad \braket{\Phi_n|\Phi(\alpha,E)} = 0\\
  \braket{\Phi(\alpha,E)|\Phi(\alpha',E')} & = \delta(\alpha-\alpha') \delta(E-E')
\end{align}

and to form an orthonormal basis. These continuums wavefunctions are characterized
by their energy $E$ and their quantum numbers, which are at this stage embraced
to the variable $\alpha$.

\begin{equation}
  1 = \sum\limits_n \ket{\Phi_n}\bra{\Phi_n} + \int \mathrm{d}\alpha \int \mathrm{d}E
      \ket{\Phi(\alpha,E)}\bra{\Phi(\alpha,E)} \label{feshbach_1}
\end{equation}

Expanding eq. \ref{se_ppsi} into this complete set yields an effective Hamiltonian
of

\begin{equation}
  \mathscr{H} = H_{PP}\, + \,
  \sum\limits_n H_{PQ} \,\frac{\ket{\Phi_n}\bra{\Phi_n}}{E^{(+)}-\varepsilon_n}\, H_{QP} \,+\,
  \int \mathrm{d}\alpha \int\mathrm{d}E \,H_{PQ} \,
  \frac{\ket{\Phi(\alpha,E)}\bra{\Phi(\alpha,E)}}{E^{(+)}-\varepsilon} \, H_{QP}
\end{equation}

which is useful to split into two parts. One describing the Hamiltonian of the
initial state being in resonance with the continuum $\Phi_s$ with its energy
$\varepsilon_s$ varying strongly close to the resonance energy and the rest
$\mathscr{H}'$

\begin{equation}
  \mathscr{H} = \mathscr{H}' + H_{PQ} \,\frac{\ket{\Phi_s}\bra{\Phi_s}}{E^{(+)}-\varepsilon_s}\, H_{QP}
\end{equation}

with
\begin{equation}
  \mathscr{H}' = H_{PP}\, + \,
  \sum\limits_{n\ne s} H_{PQ} \,\frac{\ket{\Phi_n}\bra{\Phi_n}}{E^{(+)}-\varepsilon_n}
  \, H_{QP} \,+\,
  \int \mathrm{d}\alpha \int\mathrm{d}E \,H_{PQ} \,
  \frac{\ket{\Phi(\alpha,E)}\bra{\Phi(\alpha,E)}}{E^{(+)}-\varepsilon} \, H_{QP} .
\end{equation}

This reformulation leads to the following version of eq. \ref{se_ppsi}
\begin{equation}
  (E - \mathscr{H}')\, P \Psi =
   H_{PQ} \,\frac{\ket{\Phi_s}\bra{\Phi_s}}{E^{(+)}-\varepsilon_s}\, H_{QP} P \Psi .
\end{equation}

The eigenfunctions of $\mathscr{H}'$ have to fulfill outgoing boundary conditions,
which is labelled by the superscript $(+)$

\begin{equation}
  (E-\mathscr{H}') = \psi_f^{(+)} = 0 \label{sol_outg}
\end{equation}

and will be used for an ansatz of the open channel wavefunction, where x has to
be determined
\begin{equation}
  P \Psi = \psi_f^{(+)} + x
\end{equation}

which formally gives the solution
\begin{equation}\label{sol_ppsi}
  P \Psi = \psi_f^{(+)} + \frac{1}{E^{(+)} - \mathscr{H}'}
           \frac{H_{PQ}\ket{\Phi_s}
           \braket{\Phi_s|H_{QP}|P\Psi}}{E^{(+)} - \varepsilon_s} .
\end{equation}

Hier scattering theory und Uebergangswahrscheinlichkeitsamplitude.

This expression needs to be analyzed a bit further and therefore eq. \ref{solppsi}
is multiplied from the left with $\bra{\Phi_s|H_{QP}}$.
\begin{equation}
  \braket{\Phi_s|H_{QP}|P\Psi} = \braket{\Phi_s|H_{QP}|\Psi_f^{(+)}} +
  \frac{1}{E^{(+)} - \mathscr{H}'}
  \frac{\braket{\Phi_s|H_{QP}H_{PQ}|\Phi_s} \braket{\Phi_s|H_{QP}|P\Psi}}
       {E^{(+)} - \varepsilon_s}  \label{s_ppsi}
\end {equation}

Defining the quantity

\begin{equation}
  W_{QQ} = H_{QP}\frac{1}{E^{(+)} - \mathscr{H}'}H_{PQ}
\end{equation}

and solving eq. \ref{s_ppsi} for

\begin{equation}
  \braket{\Phi_s|H_{QP}|P\Psi} = \frac{\braket{\Phi_s|H_{QP}|\Psi_f^{(+)}}(E-\varepsilon_s)}
{E - \varepsilon_s - \braket{\Phi_s|W_{QQ}|\Phi_s}}
\end{equation}

yields an easier expression for Txy.
\begin{equation}
  (E - \mathscr{H}') \psi_i^{(-)} = 0
\end{equation}

\begin{equation}
  \mathscr{T}_{if} = \mathscr{T}_{if}^{(P)} + 
                     \frac{\braket{\psi_i^{(-)}|H_{PQ}|\Phi_s}
                           \braket{\Phi_s|H_{QP}|\Psi_f^{(+)}}}
                          {E-\varepsilon_s - \Delta_s + i \frac{\Gamma}{2}}
\end{equation}

The complex matrix element $\braket{\Phi_s|W_{QQ}|\Phi_s}$ wears the influence
of the coupling on the rest of the system. Its real part can be identified as
an energy shift $\Delta_s(E)$ to the "unperturbed" resonance energy whereas its
complex part is connected to the decay width $\Gamma_s(E)$.

Introducing a delta function (see Appendix \ref{delta_function}) the splitting into the real and complex part can be accomplished.

\begin{align}
  \braket{\Phi_s|W_{QQ}|\Phi_s} & = \Delta_s(E) - i \frac{\Gamma_s(E)}{2}\\
                                & = \braket{\Phi_s|H_{QP}
                                    \frac{1}{E^{(+)}  - \mathscr{H}'}H_{PQ}|\Phi_s}\\
                                & = \braket{\Phi_s|H_{QP}
                                    \frac{\mathcal{P}}{E^{(+)} - \mathscr{H}'}|\Phi_s}
                                    - i\pi \braket{\Phi_s|H_{QP}\delta(E-\mathscr{H}')H_{PQ}
                                    |\Phi_s}
\end{align}



From the latter the decay width can easily be concluded. Inserting a complete set
of eigenfunctions as defined in eq. \ref{sol_outg} the decay width can be described
as a sum over the different open channel solutions for a given resonant state.
\begin{align}
  \Gamma_s & = 2 \pi \braket{\Phi_s|H_{QP}\delta(E-\mathscr{H}')H_{PQ}|\Phi_s}\\
           & = 2 \pi \sum\limits_r \left | \braket{\Phi_s|H_{QP}|\psi_r^{(+)}} \right|^2\\
           & = \sum\limits_r \Gamma_{sr}(E)
\end{align}
