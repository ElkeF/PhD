\chapter{Relativistic Quantum Chemistry}
\label{chapter:relativistic}
\section{Special Relativity}

The theory of relativity describes observations of physical properties
depending on the movement
of the inertial system with respect to the observer.
First proposed
by Albert Einstein in 1905 \cite{Einstein05}, it postulates the equality of
laws of physics in uniformly moving inertial systems.
One of the motivations for its development was the possibility to unify
classical mechanics with electrodynamics.

The special relativity is based on two postulates:
\begin{enumerate}
\item Principle of relativity: All physical laws are the same in all uniformly
      moving relative to each other.
\item The speed of light in vacuum is constant and independent of the location
      and movement of the light source.
\end{enumerate}

Their consistent application led to a new understanding of space and time
and awareness that energy and mass are equivalent \cite{nolting4}.


\section{Lorentz-Transformation}
In non-relativisitc physics the transformation of coordinates between two
inertial systems moving relative to each other can be calculated from the
Galilei transformation. However, this allows for relative velocities larger
than the speed of light, which contradicts the second postulate of relativity.
Therefore, for a relativistic description it has to be replaced by the Lorentz
tranformation between the coordinates of the two interial
systems $\Sigma$ and $\Sigma'$.
As an example consider an object in the inertial system $\Sigma'$ moving
in $x$ direction with respect to the inertial system of an observer $\Sigma$.
Without loss of generality $\Sigma$ can be supposed to be resting.
Let the two 
inertial systems to coincide at time $t=0$. Under these conditions, the spacial
coordinates transform as equation (\ref{lorentz1}) and equation (\ref{lorentz2}).
Without a relative movement in a certain direction, the corresponding coordinates
$y=y'$ and $z=z'$ do not change.

\begin{eqnarray}\begin{array}{rcl}\label{lorentz1}
 x'  &=& \gamma\left(x-\beta ct\right)\\
 ct' &=& \gamma\left(ct-\beta x\right)
\end{array}\end{eqnarray}

\begin{eqnarray}\begin{array}{rcl}\label{lorentz2}
 x   &=& \gamma\left(x'+\beta ct'\right)\\
 ct  &=& \gamma\left(ct'+\beta x\right)
\end{array}\end{eqnarray}

This way two new variables are introduced: the ratio between the relative speed
of the inertial systems and the speed of light $\beta = \frac vc$ and the
Lorentz factor $\gamma = \frac 1{\sqrt{1-\beta^2}}$.

For small relative speeds, or alternatively $c\rightarrow\infty$, the Lorentz
transformation equals the Galilei transformation.

The Lorentz transformation motivates an equal treatment of time and spacial
coordinates. Therefore the contravariant 4-position vector $x^\mu$ is introduced

\begin{equation}
 x^\mu = \begin{pmatrix}
          ct \\ x \\ y \\ z
         \end{pmatrix}
\end{equation}

Since all fundamental laws of physics need to hold in all inertial systems,
they need to be invariant under Lorentz transformation. Hence, they need to be
reformulated with respect to the new set of coordinates according to
the principles of relativity \cite{einstein88}.


\section{Space-Time}
From the latter section it is obvious that distances and times are not
invariant to a Lorentz transformation. Therefore they can not be treated
separately. But, motivated by the evolution of a spherical wave, one corresponding
invariant property can be found, which is the square of the space-time $s$

\begin{equation}\label{linvarianz}
 s^2 = x^\mu x_\mu =  \left(ct\right)^2 - x^2 = \left(ct'\right)^2 - x'^2 
\end{equation}

where $x_\mu$ is the covariant form of the 4-vector $x^\mu$ mediated by the
metric  $g= \mathrm{diag}(1,-1,-1,-1)$ and Einstein's summation convention is
applied.
The space-time vectors $x$ together with the metric $g$ span the so-called
Minkowski space.



\section{Relativistic Mechanics}
Starting from the four dimensional space time the relativistic velocity vector
$u^\mu = \frac{\mathrm{d}}{\mathrm{d}\tau} x^\mu$
and the corresponding momentum $p=mu$ can be derived.
Furthermore it can be shown that the mass of an object also depends on the
velocity of the moving particle with respect to the observer,

\begin{equation}\label{relmasse}
m = \gamma m_0
\end{equation}
where, $m_0$ is the
mass of the particle at rest.
Considering the acceleration of the particle leads to the famous relation between
energy and mass $E=mc^2$, which is the sum over the rest mass energy and the
kinetic energy of the particle.

Utilizing the relativistic 4-momentum the following relation for the energy
can be obtained as

\begin{equation}\label{relE}
E = \sqrt{p^2c^2 - m_0^2c^4}
\end{equation}

Because of the Lorentz invariance of $p^2 -\frac{E^2}{c^2} = -m_0^2c^2$
the definition of the momentum vector can be defined as

\begin{equation}\label{viererp}
{p} = \begin{pmatrix}\frac{iE}{c}\\p_x\\p_y\\p_z
\end{pmatrix}
\end{equation}

In the following, only rest masses are considered and hence the rest masses
will be denoted as $m$.


\section{Relativistic Electrodynamics}
The Maxwell equations describing the interactions of magnetic and electric fields
and charged particles are Lorentz covariant. Therefore, their shape
is conserved under Lorentz transformation and the modifications to the
non-relativistic formulation are limited to the introduction of a four-component
vector potential

\begin{equation}
A = \begin{pmatrix}\frac{\phi}{c}\\\mathbf{A}\end{pmatrix},
\end{equation}
where $\mathbf{A}$ is a three dimensional vector potential and $\phi$ is a scalar
potential.



\section{Relativistic Quantum Mechanics}
Motivated by the knowledge of relativistic phenomena in classical mechanics
the quantum mechanics were reformulated accordingly. This lead to the
Dirac equation, which is the equation of motion of a particle with a spin
$\frac12$ including relativistic effects. We will first discuss basic
properties of the relativistic effects on a freely moving particle and afterwards
use these basics for the description of the hydrogen atom and later for the more
complex many particle systems.

\subsection{Klein-Gordon Equation for a Freely Moving Particle}
By utilizing the correspondence principle and substituting the classical variables
by their quantum analoga one obatins for the square of the relativistic energy



\begin{equation}\label{kgg}
E^2\Psi = \left(i\hbar \frac\partial{\partial t}\right)^2\Psi = \left(m^2c^4-\hat{\mathbf{p}}^2c^2\right) \Psi
\end{equation}
This expression is known as the Klein-Gordon equation and describes the movement of
a spinless particle \cite{kellogg97}.
Even though equation (\ref{kgg}) is Lorentz covariant it needs some further
investigation. First, its solutions enable negative energies, which at that
time had no proper interpretation
and second, the corresponding time-dependent probability
density $\Psi^*(\vec{r},t)\Psi(\vec{r},t)$ is not positive definite, which
contradicts the probabilistic picture of quantum mechanics.

Despite these weaknesses the Klein-Gordon equation was the basis for the
derivation of the Dirac equation.


\subsection{Dirac Equation}
The derivation of the Dirac equation was motivated by the description of the
moving electron and to include all intrinsic properties such as spin.
It was later found to hold for fermions with a spin of $\frac12$ in general.
Special relativity requires a treatment of time and spacial coordinates on an
equal footing. Since the non-relativistic Schrödinger equation contains the first
derivative with respect to time and the second derivative with respect to the spacial
coordinates, the Schrödinger is obviously not Lorentz invariant and hence,
does not include relativistic effects.


\subsubsection{Dirac Equation for a Freely Moving Particle}\label{freep}
1928 Dirac established his famous equation which incorporates first order
derivatives of both time and spacial coordinates \cite{Dirac28_1}

\begin{equation}\label{diracg}
i\hbar \frac\partial{\partial t} \Psi = c (\alpha\cdot\hat{\mathbf{p}}+\beta mc) \Psi
\end{equation}

Here, $\hat{\mathbf{p}}$ denotes the three dimensional momentum operator and
the variables $\alpha$ and $\beta$ have to determined appropriately considering
the relations based on the comparison of the Dirac equation to the Klein-Gordon
equation
\begin{align}
\left[\alpha_i,\alpha_j\right]_+ = \alpha_i\alpha_j + \alpha_j\alpha_i
                                &= 2\delta_{ij}\\
  \alpha_i\beta + \beta\alpha_i &= 0 \\
  \beta^2                       &= 1
\end{align}

Since numbers do not commute, the variables $\alpha_i$ are obviously matrices.
They have eigenvalues of either $+1$ or $-1$, their traces have to be 0
and their dimension has to be even. These requirements are fulfilled by the
three Pauli matrices. However, in order to describe four parameters, a set of
at least four matrices is required. Therefore a dimension of at least four is
mandatory. In case of $4\times 4$ matrices more than one set fulfilles the
requirements. The one conventionally used is constructed from the Pauli matrices
and reads

\begin{equation}\begin{array}{lll}
\boldsymbol{\alpha}_x= \left(\begin{array}{cccc}
0 & 0 & 0 & 1\\
0 & 0 & 1 & 0\\
0 & 1 & 0 & 0\\
1 & 0 & 0 & 0\end{array}\right)&
\boldsymbol{\alpha}_y = \left(\begin{array}{cccc}
0 & 0 & 0 & -i\\
0 & 0 & i & 0\\
0 & -i& 0 & 0\\
i & 0 & 0 & 0\end{array}\right)&
\boldsymbol{\alpha}_z = \left(\begin{array}{cccc}
0 & 0 & 1 & 0\\
0 & 0 & 0 &-1\\
1 & 0 & 0 & 0\\
0 & -1& 0 & 0
\end{array}\right)
\end{array}\end{equation}

\begin{equation}
\boldsymbol{\beta} = \left(\begin{array}{cccc}
1 & 0 & 0 & 0\\
0 & 1 & 0 & 0\\
0 & 0 &-1 & 0\\
0 & 0 & 0 &-1
\end{array}\right)
\end{equation}


Due to the matrix dimension of the operators, the wave function $\Psi$ has to have
the same dimension and hence is represented by a four dimensional vector

\begin{equation}
\Psi = \begin{pmatrix}
\Psi_1\\\Psi_2\\\Psi_3\\\Psi_4
\end{pmatrix}
\end{equation}

A convenient way of writing the Dirac equation (\ref{diracg})
using $\alpha$ and $\beta$
is the two-component form

\begin{equation}\label{dirac2}
\left(\begin{array}{cc}
-E +mc^2 & c(\boldsymbol{\sigma}\cdot\hat{\mathbf{p}})\\
c(\boldsymbol{\sigma}\cdot\hat{\mathbf{p}}) & -E-mc^2
\end{array}\right)
\begin{pmatrix}\Psi^L\\\Psi^S\end{pmatrix}
=0
\end{equation}
Here, $E$ is the energy of the freely moving particle
and $\boldsymbol{\sigma}$ is the row vector
containing the Pauli matrices. In this formulation the wavefunction is partitioned
into the so-called large and small components depicted by the superscripts $L$ and
$S$ of the wavefunctions $\Psi$, respectively.

The latter eigenvalue equation (\ref{dirac2}) can be written as two coupled
differential equations

\begin{equation}\begin{array}{rcl}
(-E+mc^2)\Psi^L + c(\boldsymbol{\sigma}\cdot\hat{\mathbf{p}}) \Psi^S &=& 0\\
c(\boldsymbol{\sigma}\cdot\hat{\mathbf{p}}) \Psi^L +(-E-mc^2)\Psi^S &=& 0
\end{array}\end{equation}

From equations (\ref{dirac2}) the non-relativistic limit can be calculated 
as well as
a relation between the large and the small component of the
wavefunction can be obtained.

\begin{equation}\label{zusklgr}
\Psi^S = \frac{c(\mathbf{\sigma}\cdot\hat{\mathbf{p}})}{E+mc^2}\,\Psi^L
\end{equation}

Inserting equation (\ref{zusklgr}) into the first equation in (\ref{dirac2})
yields
\begin{equation}
(-E+mc^2)\Psi^L + \frac{c^2(\mathbf{\sigma}\cdot\hat{\mathbf{p}})^2}{E+mc^2}\,\Psi^L = 0
\end{equation}

In general, the energies can be shifted to any meaningful value, which then
serves as a new reference energy. For a comparison with the non-relativistic result,
it is convenient to shift the energies such
that the kinetic energy of the particle coincides with the positive energy as
in the non-relativistic description. Hence, the energies are shifted by $mc^2$.
The non-relativistic limit is then achieved by taking the limit $c\to \infty$

\begin{equation}
E'\Psi^L = \frac{c^2(\mathbf{\sigma}\cdot\hat{\mathbf{p}})^2}{E'+2mc^2}\,\Psi^L\label{hamneu} \text{ and}
\end{equation}
   
\begin{equation}
\lim_{c\rightarrow\infty} E'\Psi^L = \frac{(\mathbf{\sigma}\cdot\hat{\mathbf{p}})^2}{2m} \qquad\text{with}\quad (\mathbf{\sigma}\cdot\hat{\mathbf{p}})^2=\hat{\mathbf{p}}^2\label{nonrelgr}
\end{equation}

Obviously, equation (\ref{nonrelgr}) is equivalent to the non-relativistic
eigenvalue problem with the large component of the wavefunction as eigenfunction.
The two components can be identified with the two possible spin states
$\alpha = \begin{pmatrix} 1\\0 \end{pmatrix}$ and
$\beta= \begin{pmatrix} 0\\1 \end{pmatrix}$.

The solution of the Dirac equation leads to both positive and negative energy
eigenvalues $E_+$ and $E_-$, which are the same as in the Klein-Gordon equation

\begin{equation}
E_\pm = \pm \sqrt{m^2c^4+p^2c^2}
\end{equation}

\begin{figure}
 \centering
 \begin{tikzpicture}[
          scale=1.0,>=stealth
        ]
%     \tiny
 % \draw[very thin,color=gray] (-0.1,-4.1) grid (10.9,4.9);
 \draw [->, thick] (0,-3.5) -- (0,3) node [left]{$E$};
 \draw [thick] (-0.1,-2) -- (9,-2);
 \node at (-0.8,-2) {$-mc^2$};
 \draw [thick] (-0.1,2) -- (9,2);
 \node at (-0.8,2) {$mc^2$};
 \draw [thick] (-0.1,0) -- (0.1,0);
 \node at (-0.5,0) {0};

 \shade [ball color=diplom1] (1.5,2.5) circle (0.15);
 \node [align=center] at (1.5,-3) {freely moving\\electron};

 \draw [thick] (3.5,1.7) -- (5.5,1.7);
 \shade [ball color=diplom1] (4.5,1.7) circle (0.15);
 \node [align=center] at (4.5,-3) {bound\\electron};

 \shade [ball color=diplom1] (7.5,2.5) circle (0.15);
 \draw [thick,diplom1] (7.5,-2.5) circle (0.15);
 \draw [thick,->] (7.5,-2.35) -- (7.5,2.35);
 \node [align=center] at (7.5,-3) {pair creation};

\end{tikzpicture}

 \caption{Illustration of electronic properties of the Dirac equation solutions.}
 \label{figure:dirac_solutions}
\end{figure}

This means, that two continua of eigenvalues exist with an energy gap of $2mc^2$
as shown in Figure \ref{figure:dirac_solutions}.
Particles at rest always have the energy $mc^2$ and hence positive energies.
Such particles should normally deexcite to a state of lower energy and emit
the excess energy as a photon. In order to take care of this unphysical behaviour
Dirac postulated that all states of negative energies are occupied and hence
inaccessible. An excitation of an electron from the negative energy region into
the positive energy region would then correspond to a pair creation process yielding
an electron and a hole. This hole needs to have the same mass and the opposite charge
of the electron and was called positron. It was experimentally found and hence the
theory was verified in 1933
\cite{Anderson33}. However, it has to be noted that even though the Dirac equation
allows for the description of such a pair creation process and motivated further
theoretical investigation, the more rigorous
description for all kinds of interactions between particles and electromagnetic
fields is described by \ac{QED}. The latter will also be important for the
description of interaction between electrons in many particle systems.



\subsubsection{Solution of the Dirac Equation for the Hydrogen Atom}\label{hatom}
The hydrogen atom is the most simple system of quantum chemistry. Its solutions
are the foundation for the description of any more complex system.
In the hydrogen atom the electron is no longer moving freely but within the Coulomb
field of the proton. The proton is assumed to be fixed in space and hence the
potential in which the electron moves can be described by the scalar potential
$\phi=\frac{eZ}r$ only. The other components of the four dimensional potential $A$
are zero. Therefore the electronic part of the time-independent Dirac equation
for one electron in the field of a positive charge reads
\begin{equation}
\left(e\phi + c(\alpha\cdot\hat{\mathbf{p}})+\beta mc^2\right) \Psi = E \Psi.
\end{equation}

In non-relativistic quantum chemistry the solutions can be separated into a radial
$R(r)$ and an angular part $Y(\theta,\phi)$.

\begin{equation}
\Psi^{NR}(r,\theta,\phi) = \frac1r R_{nl}(r)Y^{ml}_l(\theta,\phi)
\end{equation}
Here the angular part can be described by the spherical harmonics
$Y_l^{m_l}(\theta,\phi)$.

Analogously the relativistic wave functions can be separated, which leads to

\begin{equation}\label{seprel}
\Psi(r,\theta,\phi,\tau) = \frac1r \left( \begin{array}{rl}
P_{n\kappa}(r)  & \xi_{\kappa,m}(\theta,\phi)\\
iQ_{n\kappa}(r) & \xi_{-\kappa,m}(\theta,\phi)
\end{array}\right)
\end{equation}

Here, $P_{n\kappa}(r)$ and $Q_{n\kappa}(r)$ are the radial solutions of the
large and small component of the 4-spinor, respectively. $\xi_{\kappa,m}(\theta,\phi)$
and $\xi_{-\kappa,m}(\theta,\phi)$ are two component spinors describing the
angular distribution. As in the non-relativistic approach, the radial solutions
depend on the principal quantum number $n$. The radial solutions of the large and
small components each have nodes at certain distances. However, these nodes
normally do not coincide and therefore the radial density of the total radial
solution does not show nodes as in the non-relativistic case.

Non-relativistically, the solutions
of the angular wave function depend on the angular and magnetic quantum numbers
$l$ and $m_l$. However, the Dirac equation additionally to relativistic effects
incorporates spin and hence also spin-orbit coupling. Therefore $l$ and $m_l$ are
not good quantum numbers, but rather the total angular momentum $j=l\pm s$.
For the characterization of the angular solutions it is convenient
to introduce another quantum number $\kappa=\mp(j+\frac12)$, which is connected
to the eigenvalue of the operator
$\mathbf{k}= \boldsymbol{\sigma}\cdot \mathbf{l} + \hbar \mathbf{1}_2$

\begin{equation}
 \mathbf{k} \, \xi_{\pm \kappa,m} = \pm \hbar (j+\frac12) \, \xi_{\pm \kappa,m}
 = \hbar \kappa_\pm \, \xi_{\pm \kappa,m}
\end{equation}


The resulting angular wavefunction of a particle with a spin of $\frac12$ reads

\begin{equation}\begin{array}{rcll}
\xi_{\kappa,m} &=& \frac1{\sqrt{2l+1}}\begin{pmatrix}\sqrt{l+m_l+1}Y_l^{m_l}\\\sqrt{l-m_l}Y_l^{m_l+1}\end{pmatrix} &\quad\text{for }\kappa<0\\\vphantom{\rule{0pt}{30pt}}
\xi_{\kappa,m} &=& \frac1{\sqrt{2l+1}}\begin{pmatrix}-\sqrt{l-m_l}Y_l^{m_l}\\\sqrt{l+m_l+1}Y_l^{m_l+1}\end{pmatrix} &\quad\text{for }\kappa>0\\
\end{array}\end{equation}

In Table \ref{qzspinor} the non-ambiguous assignment of quantum numbers and atomic
spinors is illustrated.

\begin{table}[h]
\centering
\begin{tabular}{lccccccc}
\toprule
        & $s_{\frac12}$ & $p_{\frac12}$ & $p_{\frac32}$ & $d_{\frac32}$ & $d_{\frac52}$ & $f_{\frac52}$ & $f_{\frac72}$\\\vphantom{\rule{0pt}{20pt}}
$l$     & 0            & 1           & 1            & 2            & 2            & 3            & 3\\\vphantom{\rule{0pt}{20pt}}
$j$     & $\frac12$    & $\frac12$   & $\frac32$    & $\frac32$    & $\frac52$    & $\frac52$    & $\frac72$\\\vphantom{\rule{0pt}{20pt}}
$\kappa$& -1           & 1           & -2           & 2            & -3           & 3            & -4\\
\bottomrule
\end{tabular}
\caption{Assignment of the quantum numbers to two-component spinors \cite{dyallfaegri}.}
\label{qzspinor}
\end{table}

The shape of the absolut square angular solutions differs from their
non-relativistic counterparts. It no longer depends on $l$ but on the total
angular momentum $j$ and its projection $m_j$ as illustrated in Figure
\ref{figure:ang_orb}. The $s_{\frac12}$ orbital has spherical symmetry as
in the non-relativistic case, but the $p_{\frac12}$ orbital shows spherical
symmetry as well. In case of the $p_{\frac32}$ orbital with $m_j=\pm \frac12$
the shape is similar to a non-relativistic $p_z$-orbital. However, it does not
have a node at the origin. The $p_{\frac32}$ orbital with $m_j=\pm \frac32$
has the same shape as the non-relativistic $p_{\pm 1}$ orbitals, from which
the $p_x$ and $p_y$ orbital can be constructed. In contrast to the
$m_j=\pm \frac12$ case, the $p_{\frac32,\frac32}$ orbital does have a node at
the origin.

\begin{figure}
\centering
  \begin{tikzpicture}[scale=0.8]
    \begin{axis}[%
        %colormap name = blues,
        %3d box=complete,
        axis equal,
        width=10cm,
        height=10cm,
        axis lines = center,
        xlabel = {$x$},
        ylabel = {$y$},
        zlabel = {$z$},
        ticks=none,
        enlargelimits=0.3,
        view/h=45,
        scale uniformly strategy=units only,
    ]
    \addplot3[%
        %opacity = 0.5,
        %surf,
        mesh,
        draw=diplom2,
        z buffer = sort,
        %samples = 21,
        variable = \u,
        variable y = \v,
        domain = 0:180,
        y domain = 0:360,
    ]
    ({cos(u)*sin(v)}, {sin(u)*sin(v)}, {cos(v)});
    %({( (cos(u))^2/2.0/pi + (sin(u))^2/8.0/pi) * cos(v)*sin(u)},
    % {( (cos(u))^2/2.0/pi + (sin(u))^2/8.0/pi) * sin(v)*sin(u)},
    % {( (cos(u))^2/2.0/pi + (sin(u))^2/8.0/pi) * cos(u)});
    %({ (sin(u))^2/8.0/pi * cos(v)*sin(u)},
    % { (sin(u))^2/8.0/pi * sin(v)*sin(u)},
    % { (sin(u))^2/8.0/pi * cos(u)});
    \end{axis}
\end{tikzpicture}
   \hfill
  \begin{tikzpicture}[scale=0.8]
    \begin{axis}[%
        %colormap name = blues,
        axis equal,
        width=10cm,
        height=10cm,
        axis lines = center,
        xlabel = {$x$},
        ylabel = {$y$},
        zlabel = {$z$},
        ticks=none,
        enlargelimits=0.3,
        view/h=45,
        scale uniformly strategy=units only,
    ]
    \addplot3[%
        %opacity = 0.5,
        %surf,
        mesh,
        draw=diplom2,
        z buffer = sort,
        %samples = 21,
        variable = \u,
        variable y = \v,
        domain = 0:180,
        y domain = 0:360,
    ]
    ({cos(u)*sin(v)}, {sin(u)*sin(v)}, {cos(v)});
    %({( (cos(u))^2/2.0/pi + (sin(u))^2/8.0/pi) * cos(v)*sin(u)},
    % {( (cos(u))^2/2.0/pi + (sin(u))^2/8.0/pi) * sin(v)*sin(u)},
    % {( (cos(u))^2/2.0/pi + (sin(u))^2/8.0/pi) * cos(u)});
    %({ (sin(u))^2/8.0/pi * cos(v)*sin(u)},
    % { (sin(u))^2/8.0/pi * sin(v)*sin(u)},
    % { (sin(u))^2/8.0/pi * cos(u)});
    \end{axis}
\end{tikzpicture}
   \hfill
  \begin{tikzpicture}[scale=0.8]
    \begin{axis}[%
        %colormap name = blues,
        axis equal,
        width=10cm,
        height=10cm,
        axis lines = center,
        xlabel = {$x$},
        ylabel = {$y$},
        zlabel = {$z$},
        ticks=none,
        enlargelimits=0.3,
        view/h=45,
        scale uniformly strategy=units only,
    ]
    \addplot3[%
        %opacity = 0.5,
        %surf,
        mesh,
        draw=diplom2,
        z buffer = sort,
        %samples = 21,
        variable = \u,
        variable y = \v,
        domain = 0:180,
        y domain = 0:360,
    ]
    %({cos(u)*sin(v)}, {sin(u)*sin(v)}, {cos(v)});
    ({( (cos(u))^2/2.0/pi + (sin(u))^2/8.0/pi) * cos(v)*sin(u)},
     {( (cos(u))^2/2.0/pi + (sin(u))^2/8.0/pi) * sin(v)*sin(u)},
     {( (cos(u))^2/2.0/pi + (sin(u))^2/8.0/pi) * cos(u)});
    %({ (sin(u))^2/8.0/pi * cos(v)*sin(u)},
    % { (sin(u))^2/8.0/pi * sin(v)*sin(u)},
    % { (sin(u))^2/8.0/pi * cos(u)});
    \end{axis}
\end{tikzpicture}
 \hfill
  \raisebox{1cm}{\begin{tikzpicture}[scale=0.8]
    \begin{axis}[%
        %colormap name = blues,
        axis equal,
        width=10cm,
        height=10cm,
        axis lines = center,
        xlabel = {$x$},
        ylabel = {$y$},
        zlabel = {$z$},
        ticks=none,
        enlargelimits=0.3,
        view/h=45,
        scale uniformly strategy=units only,
    ]
    \addplot3[%
        %opacity = 0.5,
        %surf,
        mesh,
        draw=diplom2,
        z buffer = sort,
        %samples = 21,
        variable = \u,
        variable y = \v,
        domain = 0:180,
        y domain = 0:360,
    ]
    %({cos(u)*sin(v)}, {sin(u)*sin(v)}, {cos(v)});
    %({( (cos(u))^2/2.0/pi + (sin(u))^2/8.0/pi) * cos(v)*sin(u)},
    % {( (cos(u))^2/2.0/pi + (sin(u))^2/8.0/pi) * sin(v)*sin(u)},
    % {( (cos(u))^2/2.0/pi + (sin(u))^2/8.0/pi) * cos(u)});
    ({ (sin(u))^2/8.0/pi * cos(v)*sin(u)},
     { (sin(u))^2/8.0/pi * sin(v)*sin(u)},
     { (sin(u))^2/8.0/pi * cos(u)});
    \end{axis}
\end{tikzpicture}
}
\caption{Density plots of the angular solutions of the Dirac equation
         of the hydrogen atom $\xi_{\kappa,m}$. From left to right:
         $s_{\frac12}$, $p_{\frac12}$, $p_{\frac32,\frac12}$,
         $p_{\frac32,\frac32}$.}
\label{figure:ang_orb}
\end{figure}


From the above considerations an expression for the energy of bound states can be
determined
\begin{equation}\label{HErel}
E = mc^2 \left[1+ \frac1{c^2}
    \left(\frac{\frac{Ze^2}{\hbar c}}{n-|\kappa|+
         \sqrt{\kappa^2-\frac{\frac{Ze^2}{\hbar c}}{c^2}}}\right)^2
    \right]^{\frac12}
\end{equation}
The smaller $|\kappa|$ and $n$ are, the more is the corresponding one particle
state energetically stabilized. This energetic stabilization is accompanied
by a contraction of the radial density. Hence, the $1s$ orbital exhibits the
largest energy shift and largest contraction. From equation
(\ref{HErel}) it can also be seen
that for a given azimuthal quantum number $l$ the splitting into states of quantum
numbers $j$ is achieved such that the state with the lower total angular momentum
has the lower energy.


\section{Many-Particle Systems in Relativistic Quantum Chemistry}
Until now only the interaction of one electron with a static potential has been
discussed. However, in chemistry most systems contain several electrons, whose
interactions also have to be taken into account.
In non-relativistic quantum chemistry it is assumed that the Born-Oppenheimer
approximation, which leads to a separation of electronic and nuclear solutions
is reasonable and the electronic solutions can be obtained with the
nuclear positions as fixed parameters.
Here, a Hamiltonian of an $N$ electron system is applied, which consists
of a one-particle and a two-particle part. The one-particle operator $\hat{h}(i)$
describes the kinetic energy of a single electron and its interaction with the
static Coulomb field of the nuclei and the two-particle part $\mathbf{\hat{g}}_{ij}$
describes the interactions of the electrons amongst each other, which in the
non-relativistic case is the Coulomb operator $\frac1{r_{ij}}$.

\begin{equation}
\mathcal{H}= \sum\limits_{i=1}^N \hat{h}(i) + \sum\limits_{i<j} {\hat{g}}_{ij}
\end{equation}

From the solutions of the hydrogen atom the behaviour of an electron in a constant
Coulomb field is known. Inserting it into the latter equation yields the
relativistic Hamiltonian

\begin{equation}
\mathcal{H}_{D}= \sum\limits_{i=1}^N (-e \phi(r_i)+
                 c(\hat{\boldsymbol{\alpha}}^{i}\cdot\hat{\mathbf{p}}^{i})
                 +\boldsymbol{\beta}mc^2)
                 + \sum\limits_{i<j} {\hat{g}}_{ij}
\end{equation}

The electron-electron interaction term ${\hat{g}}_{ij}$ can only be derived
from \ac{QED}, where the interaction of two moving particles is investigated.
The result is expanded in series of $\frac1c$ and truncated after the second
order term. This yields the Coulomb-Breit term $\hat{g}_{ij}^{CB}$, which
contains the unretarded Coulomb interaction $\hat{g}_{ij}^C=\frac1{r_{ij}}$
describing the interaction of particles and
the Breit term $\hat{g}_{ij}^B$ describing the current-current interaction.

\begin{equation}\begin{array}{rcl}
\hat{g}_{ij}^{CB} &=& \hat{g}_{ij}^C + \hat{g}_{ij}^B\\
                &=& \frac1{r_{ij}} - \frac1{2r_{ij}}\left(\boldsymbol{\alpha}_i
                    \cdot\boldsymbol{\alpha}_j+\frac{(\boldsymbol{\alpha}_i
                    \cdot r_{ij})(\boldsymbol{\alpha}_j\cdot r_{ij})}
                    {r^2_{ij}}\right)
\end{array}\end{equation}

The latter can be divided into the Gaunt term
$\hat{g}_{ij}^G = - \frac1{2r_{ij}}\boldsymbol{\alpha}_i\cdot\boldsymbol{\alpha}_j$
describing the interaction
with magnetic fields and a term purely containing retardation effects.
Since the Gaunt term provides the largest contribution in the Breit term and the
implementation of the full Breit term is difficult, very often only the Gaunt term
is considered in numerical calculations.
However, the Breit term is damped by a factor of $\frac 1{c^2}$ hidden in the
$\boldsymbol{\alpha}$ operators. Therefore, it is normally applied
for high precision calculations
only and the Dirac-Coulomb operator $\mathcal{H}_{DC}$ is applied instead.

\begin{equation}\label{DCHamil}
\mathcal{H}_{DC} = \sum\limits_{i=1}^N (-e \phi(r_i)\mathds{1}+c\hat{\boldsymbol{\alpha}}^{i}\cdot\hat{\mathbf{p}}^{i}+\boldsymbol{\beta}mc^2) + \sum\limits_{i>j}^N \frac1{r_{ij}}
\end{equation}


Since the Gaunt interaction has been shown to affect the ionization energies of
TlH by no more than \unit[0.06]{eV} \cite{Pernpointner05} and the errors of
decay width calculations
are by far larger than the effect of the
Gaunt interaction, the Dirac-Coulomb Hamiltonian is used throughout this thesis.
Even though this Hamiltonian is not Lorentz invariant the description is good
enough to cover all relevant effects.

\subsection{Dirac-Coulomb Hartree-Fock}
Starting from the Dirac-Coulomb operator of equation (\ref{DCHamil}) it is assumed
that the wavefunction can be written as a anti-symmetrized product of one particle
functions $\phi_i$ called a Slater determinant as in the
non-relativistic quantum chemistry.

\begin{eqnarray}
\Psi &=& \mathcal{A} \left(\prod\limits_{i=1}^N\phi_i(r_i)\right)\\\vphantom{\rule{0pt}{50pt}}
     &=& \frac1{\sqrt{n!}}\begin{vmatrix}
\phi_1(r_1) & \phi_2(r_1) & \cdots & \phi_N(r_1)\\
\phi_1(r_2) & \phi_2(r_2) & \cdots & \phi_N(r_2)\\
\vdots      & \vdots     & \ddots  & \vdots\\
\phi_1(r_N) & \phi_2(r_N) & \cdots & \phi_N(r_N)\\
\end{vmatrix}
\end{eqnarray}

From this ansatz follows the energy as a functional:

\begin{eqnarray}
  E &=& \braket{\Psi|\mathcal{H}_{DC}|\Psi}\\\vphantom{\rule{0pt}{30pt}}
  &=& \sum\limits_{i=1}^N\braket{\phi_i|\hat{h}_D(i)|\phi_i}  + \frac12 \sum\limits_{i,j}^N\left(\braket{\phi_i\phi_j|\frac1{r_{ij}}|\phi_i\phi_j} - \braket{\phi_i\phi_j|\frac1{r_{ij}}|\phi_j\phi_i}\right)
\end{eqnarray}

From the latter expression the Dirac-Hartree-Fock equation is derived, which
describes the motion of an electron in the averaged field of the other
electrons $v^{DHF}(i)$.

\begin{eqnarray}
\mathbf{f}_i\ket{\phi_i} &=& \left(\mathbf{h}_D(i) + v^{DHF}(i)\right)\ket{\phi_i}\\
                         &=& \left( \mathbf{h}_D(i) + \sum\limits_{l=1}^{N}(\mathbf{J}_l-\mathbf{K}_l)\right)\ket{\phi_i}\\
                         &=& \varepsilon_i\ket{\phi_i}
\end{eqnarray}

Here, the operators $\mathbf{J}_l$ and $\mathbf{K}_l$ both depend on the
one-particle functions and denote the Coulomb and the
exchange term, respectively.
Starting from an educated guess of the one-particle functions, the Fock operator
$\mathbf{F}$ is constructed and applied to yield another set of one-particle
functions. This iterative procedure is carried out until the system is
converged or \emph{self-consistent}, which is why the method is called a
\ac{SCF} method \cite{szabo}.

In non-relativistic quantum mechanics the variational principle ensures the
lowest energy solution to be closest to the true energy and hence the wavefunction
to be optimized. In relativistic quantum mechanics negative energy solutions
are possible but unphysical. The variational principle would therefore lead to
negative energy states. In order to prevent this so-called \emph{variational collapse},
the basis is subject to certain restrictions, which will be discussed later in
section \ref{basis}.
 
For the actual energy calculations the one-particle functions \{$\phi_i$\}
are constructed as a linear combination of atomic functions. The latter are
expanded in a finite set of basis functions \{$\varphi_\mu$\} for the large and small
component. The resulting molecular one-particle function then reads
\begin{equation}
\phi_i = \sum\limits_\mu^{N^L} c_{\mu\alpha}^{iL} \varphi_{\mu\alpha}^L +  \sum\limits_\mu^{N^L} c_{\mu\beta}^{iL} \varphi_{\mu\beta}^L + \sum\limits_\mu^{N^S} c_{\mu\alpha}^{iS} \varphi_{\mu\alpha}^S + \sum\limits_\mu^{N^S} c_{\mu\beta}^{iS} \varphi_{\mu\beta}^S.
\end{equation}

Here, $N^L$ and $N^S$ are the numbers of basis functions of the large and small
component, respectively. This means that the energy depends on the expansion
coefficients $c_{\mu s}^{iX}$, where $X=L,S$ denotes the compoent of the spinor and
$s$ denotes the spin state. The resulting relativistic Roothan-Hall equations
in matrix notation read as

\begin{equation}
\mathcal{FC} = \mathcal{SCE}
\end{equation}

The Fock matrix $\mathcal{F}$ and the overlap matrix $\mathcal{S}$
hereby consist of the matrix elements:

\begin{equation}\begin{array}{rcl}\label{roothanm}
F^{XY,sp}_{\mu\nu} &=& \braket{\varphi^X_{\mu s}|\mathbf{f}_i|\varphi_{\nu p}^Y}\\
S^{XY,sp}_{\mu\nu} &=& \braket{\varphi^X_{\mu s}|\varphi_{\nu p}^Y}
\end{array}\end{equation}




%\subsection{Zweikomponentige Methoden}
%Wie bereits erw?hnt wurde, handelt es sich bei der Diracgleichung um ein System gekoppelter Differentialgleichungen, die hier noch einmal aufgef?hrt werden sollen.
%
%\begin{equation}\label{koppdirac}
%\left(\begin{array}{ll}
%V & c(\mathbf{\sigma}\cdot\hat{\mathbf{p}})\\
%c(\mathbf{\sigma}\cdot\hat{\mathbf{p}}) & V -2mc^2\\\end{array}\right)
%\begin{pmatrix}\Psi^L\\\Psi^S\end{pmatrix}
%= \begin{pmatrix}\Psi^L\\\Psi^S\end{pmatrix} E
%\end{equation}
%
%Dabei ist die kleine Komponente des Spinors an die gro?en Komponente gekoppelt, wobei der verkn?pfende Operator $\hat{R}$ sei.
%\begin{equation}
%\Psi^S = \hat{R}\Psi^L
%\end{equation}
%
%Die Grundidee der zweikomponentigen Methoden ist, durch eine unit?re Transformation des Hamiltonoperators eine Entkopplung der L?sungen positiver und negativer Energien zu erlangen. Dadurch k?nnten nur die positiven Energien unter ausschlie?licher Verwendung der gro?en Komponente der Basis berechnet werden.
%
%\begin{equation}
%\hat{U}^\dagger\left(\begin{array}{cc}
%\hat{h}_{11} & \hat{h}_{12}\\
%\hat{h}_{21} & \hat{h}_{22}\\\end{array}\right) \hat{U}=
%\left(\begin{array}{cc}
%\hat{h}_+ & 0 \\
%0         & \hat{h}_-
%\end{array}\right)
%\end{equation}
%
%Dabei hat die Transformationsmatrix $\hat{U}$ folgende Struktur:
%\begin{equation}
%\hat{U} = \left(\begin{array}{cc}
%\frac1{\sqrt{1+\hat{R}^\dagger\hat{R}}}          & \frac1{\sqrt{1+\hat{R}^\dagger\hat{R}}} \hat{R}^\dagger\\
%-\frac1{\sqrt{1+\hat{R}\hat{R}^\dagger}} \hat{R} & \frac1{\sqrt{1+\hat{R}\hat{R}^\dagger}}
%\end{array}\right)
%\end{equation}
%
%Da die Nichtdiagonalelemente des transformierten Hamiltonoperators 0 sein m?ssen, ergibt sich die Bedingung
%\begin{equation} \label{nondigbed}
%2mc^2\hat{R} = c (\mathbf{\sigma}\cdot\hat{\mathbf{p}}) + \left[V,\hat{R}\right] - \hat{R}c(\mathbf{\sigma}\cdot\hat{\mathbf{p}})\hat{R}.
%\end{equation}
%Im nicht-relativistischen Grenzfall ist der Kopplungsoperator $\hat{R}=\frac{(\mathbf{\sigma}\cdot\hat{\mathbf{p}})}{2mc}$ bekannt. Bei dessen Wahl als Startpunkt f?r eine iterative L?sung der Gleichung \ref{nondigbed} wird ersichtlich, dass immer h?here Ableitungen von $\hat{R}$ auftreten. Der Operator muss also angen?hert werden, wobei die beiden etabliertesten Methoden die Foldy-Wouthuysen-Methode und die Douglas-Kroll-Methode sind, die unterschiedliche Herangehensweise w?hlen. W?hrend die Foldy-Wouthuysen-Methode einen st?rungstheoretischen Ansatz verwendet, verwendet die Douglas-Kroll-Methode einen variationellen.
%
%Ein neuerer Ansatz manipuliert nicht den Kopplungsoperator $\hat{R}$, sondern verwendet eine algebraische Herangehensweise. Unter einmaliger Verwendung des Kopplungsoperators im nicht-relativistischen Grenzfall wird zun?chst die Dirac-Gleichung \ref{koppdirac} gel?st, in der die komplette Transformation enthalten ist. Aus den L?sungen wird auf die komplette Kopplungsmatrixmatrix $\mathbf{R}$ zur?ckgeschlossen, mit deren Hilfe dann die entkoppelten Hamiltonoperatoren f?r die postivien und die negativen Energien $\hat{h}_+$ und $\hat{h}_-$ konstruiert werden k?nnen. Auf diese Weise werden Operatoren erhalten, die im Vergleich zu denen der st?rungstheoretischen Foldy-Wouthuysen-Methode f?r die gew?hlte Basis in unendlicher Ordnung genau sind, weshalb die Methode IOTC (\emph{infinite-order two-component})\cite{Ilias07} genannt wird.\\
%Soll das IOTC nicht nur f?r die Beschreibung von Einteilchen-, sondern auch von Vielteilchensystemen verwendet werden, wird die sogenannte AMFI-Korrektur (\emph{atomic mean-field approximation to the spin-orbit interaction}) hinzugef?gt, die die Spin-Bahn-Kopp\-lung beschreibt und damit skalarrelativistische Ergebnisse um diesen Anteil korrigieren kann. Da bei der Berechnung der Korrektur nur die Integrale ?ber die Orbitale, die am selben Atom lokalisiert sind, ber?cksichtigt werden, sind f?r die Betrachtung von Molek?len geringe Abweichungen zum vierkomponentig ermittelten Resultat zu erwarten.\cite{Hess96}
% 



\subsection{Construction of a Basis for the Relativistic Treatment}\label{basis}
As already discussed, the four components of a spinor are expanded into a finite
set of one-particle basis functions. Thereby the basis sets of the large and small
component are treated separately.

\begin{equation}\begin{array}{ccc}
\phi_i^L = \begin{pmatrix}\phi^L_{i\alpha}\\\phi^L_{i\beta}\\0\\0\end{pmatrix} &\hspace{3em} &\phi_i^S = \begin{pmatrix}0\\0\\\phi^S_{i\alpha}\\\phi^S_{i\beta}\end{pmatrix}
\end{array}\end{equation}

The two components are expressed as linear combinations of scalar atomic
basis functions $\varphi_\mu^X$, which are weighted by the expansion coefficients
$c_{\mu s}^{iX}$.

\begin{equation}\begin{array}{ccc}
\phi^L_{is} = \sum\limits_{\mu=1}^{N^L} c_{\mu s}^{iL} \varphi_\mu^L &\hspace{3em}& \phi^S_{is} = \sum\limits_{\mu=1}^{N^S} c_{\mu s}^{iL} \varphi_\mu^S
\end{array}\end{equation}

They are constructed starting from the solutions of the hydrogen atom centered
at the corresponding atoms, which have the form
\begin{equation}
s_\tau^X = N_\tau^X r^{n-1}\mathrm{e}^{\alpha_\tau^X r}Y_l^{m_l}(\theta,\varphi)
\end{equation}

It is not possible to solve the associated integrals of the \emph{slater-type
functions} analytically and therefore, they are expanded into a set of primitive
Gauss functions $g_\tau^X$ yielding contracted Gauss functions.
These contracted Gauss functions do not show the correct asymptotic behaviour for
short and large distances ($r\rightarrow 0$, $r\rightarrow\infty$), but allow to
solve the integrals analytically and thereby reducing the computational cost.
In spherical coordinates they are given by
\begin{equation}
g_\tau^X = N_\tau^Xr^l\mathrm{e}^{-\alpha_\tau^Xr^2}Y_l^{m_l}(\theta,\varphi)
\end{equation}
and in cartesian coordinates by
\begin{equation}
g_\tau^X = N_\tau^X x^{n_x^X}y^{n_y^X}z^{n_z^X} \mathrm{e}^{-\alpha_\tau^Xr^2}.
\end{equation}

Here, $N_\tau^X$ denotes the normalization factor and the sum of the spacial
coordinate exponents $n_i^X$ corresponds to the azimuthal quantum number $l$.
\begin{equation}
n_x^X+n_y^X+n_z^X=l
\end{equation}
The final contraction then gives the basis functions
\begin{equation}
\varphi^X_\mu = \sum\limits_\tau d_{\mu\tau}^Xg_\tau^X,
\end{equation}
where the factor $d_{\mu\tau}^X$ weights the different primitive Gauss functions.

In order to prevent the variational collapse, the basis sets have to fulfill
the \emph{kinetic balance} condition, which is based on the non-relativistic limit of
equation (\ref{zusklgr}). Since the energy $E$ of the electron is much smaller than
$2mc^2$, the following relation between the coefficients
of the large and small component functions has to be obeyed:

\begin{equation}\label{kinbal}
\phi_S = \frac1{2mc}(\boldsymbol{\sigma}\cdot\hat{p})\,\phi_L
\end{equation}


Because of the appearance of the operator $\boldsymbol{\sigma}\cdot\hat{p}$
a given basis function of the large component with a distinct $l$ value
leads to the generation of two functions of the small component
with $l-1$ and $l+1$
This means that the small component basis
contains almost twice as many basis functions as the basis of the large component.
Hence, the computational cost of a relativistic calculation is higher than its
non-relativistic analog.


\section{Correlation Methods}
In quantum chemical calculations the electron correlation is treated by taking into
account excitations of different classes from the ground state. Formally, this is
achieved by pairs of creation and annihilation operators acting on the ground
state. By allowing all possible combinations of such operators pair creation
processes would be introduced in the configurations. However, in quantum chemistry
energies high enough to overcome the energy gap of $2mc^2$ is hardly achieved.
Therefore, it is reasonable to exclude these configurations from the calculations.
This practice is called \emph{no-pair approximation}.





%\section{Relativistic Effects in Many Electron Systems}
\section{Relativistic Effects and Spin-Orbit Coupling in Many Electron Systems}
\subsection{Spin-Orbit Coupling}
Since spin is intrinsically included in the Dirac equation, so is the
coupling of the angular and the spin momentum. This leads to
only the total quantum number $J$ and the corresponding projection $M_J$
being good quantum numbers rather than $L, M_L$ and $S, M_S$. As a
consequence, the energy levels of a specific $L$ value are split and the
symmetry of some of the wavefunctions are changed (see Figure
\ref{figure:ang_orb}).
The spin-orbit coupling constant $a$ is proportional to $Z^4$.
This means that the most pronounced spin-orbit splittings are to be observed
in heavy atoms, where also scalar-relativistic effects can not be
neglected.

In order to describe relativistic effects without spin-orbit coupling
several spin-free formulations of the Dirac equation, where the results
depend on the definition \cite{Visscher99}.
Extensive discussions about these Hamiltonians and there applications
can be found in references \cite{ReiherWolf09,Saue11}.

Observed consequences of the spin-orbit coupling are the spectral splittings,
the so-called \emph{inert electron
pair}, which describes the observation that lower oxidation states
are favoured in elements of higher periods compared to their lower period
analogs. This \emph{inert electron pair} does not contribute in bondings
to other atoms and hence systems like e.g. TlH exhibit shorter bond lengths
then would be expected non-relativistically.


\subsection{Scalar-Relativistic Effects}
The orbital energies of the hydrogen atoms have been shown to decrease
due to the inclusion of relativistic effects. However, in many electron
atoms the interaction of both nucleus and several electron have to be taken
into account. Electrons close to the nucleus shield the nuclear charge such
that electrons further apart feel a reduced effective nuclear charge and
are therefore less attracted to the nucleus which leads to a decontracted
radial electron density.
Electrons close to the nucleus are electrons inhabiting $s$ and
$p_{1/2}$-orbitals which
have a non-zero electron density at the nucleus. Therefore, these orbitals
experience a contraction of the radial densities and the other orbitals
react and rearrange accordingly. As a rule of thumb, $s$- and $p$-orbitals
are energetically stabilized compared to the non-relativistic result
and the radial densities are contracted, while
the $d$- and $f$-orbitals are energetically destabilized
and the corresponding radial densities are decontracted.

Consequences are shifts of orbital energies compared to each other. An example
is the colour of gold, where the band energy of the 6$s$ valence band is
decreased, while the band energy of the 5$d$ orbitals is increased. Hence,
the energy difference is decreased and an excitation is observed in the
visible spectrum of light.
