\chapter{Relativistic Quantum Chemistry}
\section{Special Relativity}

The theory of relativity describes physical properties depending on the movement
of the inertial system.
The special relativity was first proposed
by ALbert Einstein in 1905 \cite{Einstein05} and postulates the equality of
the laws of physics in uniformly moving inertial systems.
It was motivated by the possibility to unitize the classic mechanics with
the electrodynamics.

The special relativity is thereby based on two postulates:
\begin{enumerate}
\item principle of relativity: All physical laws are the same in all uniformly
      moving relative to each other.
\item The speed of light is constant in vacuum and independent of the location
      and movement of the light source.
\end{enumerate}

Their consistent application lead to a new understanding of space and time
and awareness that energy and mass are equivalent.\cite{nolting4}


\section{Lorentz-Transformation}
The Lorentz transformation replaces the Galilei transformation of the
non-relativistic description for the transition between two inertial systems
$\Sigma$ and $\Sigma'$.
In contrast to the latter it takes into account that information can not travel
faster than the speed of light.
Without loss of generality $\Sigma$ can be supposed to be resting. Consider the
the second inertial system $\Sigma'$ to move in $x$ direction and the two
inertial systems to coincide at time $t=0$. Under these conditions, the spacial
coordinates transform as equation (\ref{lorentz1}) and equation (\ref{lorentz2}).
Without a relative movement in a certain direction, the corresponding coordinates
$y=y'$ and $z=z'$ do not change.

\begin{eqnarray}\begin{array}{rcl}\label{lorentz1}
 x'  &=& \gamma\left(x-\beta ct\right)\\
 ct' &=& \gamma\left(ct-\beta x\right)
\end{array}\end{eqnarray}

\begin{eqnarray}\begin{array}{rcl}\label{lorentz2}
 x   &=& \gamma\left(x'+\beta ct'\right)\\
 ct  &=& \gamma\left(ct'+\beta x\right)
\end{array}\end{eqnarray}

This way two new variables are introduced: the ratio between the relative speed
and of the inertial systems and the speed of light $\beta = \frac vc$ and the
Lorentz factor $\gamma = \frac 1{\sqrt{1-\beta^2}}$.

For small relative speeds, or alternatively $c\rightarrow\infty$, the Lorentz
transformation equals the Galilei transformation.

Since in all inertial systems the same laws of physics have to hold, these laws
have to be invariant with respect to the Lorentz transformation. Therefore

Since all fundamental laws of physics need to hold in all inertial systems,
they need to be invariant under Lorentz transformation. Hence, they need to be
reformulated with respect to the new set of coordinates according to
the principles of relativity. \cite{einstein88}


\section{Space-Time}
From the latter section it is obvious that distances and times are not
invariant to a Lorentz transformation. Therefore they can not be treated
separately. But, motivated by the evolution of a spherical wave, one corresponding
invariant property can be found, which is the square of the space-time $s$

\begin{equation}\label{linvarianz}
 s^2 = x^2 - \left(ct\right)^2 = x'^2 - \left(ct'\right)^2
\end{equation}

with

\begin{equation}\label{vierervec}
 \mathrm{s} = \begin{pmatrix}x\\y\\z\\ict\end{pmatrix}
\end{equation}



\subsection{Relativistic Mechanics}
Starting from the four dimensional space time the relativistic velocity vector $u$
and the corresponding canonical or linear? momentum $p=mu$ can be derived.
Furthermore it can be shown that the mass of an object also depends on the
velocity of the moving particle with respect to the observer. Here, $m_0$ is the
mass of the particle at rest.

\begin{equation}\label{relmasse}
m = \gamma m_0
\end{equation}

Considering the acceleration of the particle leads to the famous relation between
energy and mass $E=mc^2$, which is the sum over the rest mass energy and the
kinetic energy of the particle.

Utilizing the relativistic momentum the following relation for the energy can be
obtained

\begin{equation}\label{relE}
E = \sqrt{p^2c^2 - m_0^2c^4}
\end{equation}

Because of the Lorentz invariance of $p^2 -\frac{E^2}{c^2} = -m_0^2c^2$
the definition of the momentum vector can be defined as

\begin{equation}\label{viererp}
\mathbf{p} = \begin{pmatrix}\frac{iE}{c}\\p_x\\p_y\\p_z
\end{pmatrix}
\end{equation}

In the following, all masses $m$ are to be considered as the corresponding
rest masses $m_0$.

\subsection{Relativistic Electrodynamics}
The Maxwell equations describing the interactions of magnetic and electric fields
and charged particles are already Lorentz covariant. Therefore, their shape
is conserved under Lorentz transformation and the modifications to the
non-relativistic formulation are limited to the introduction of a four-component
vector potential

\begin{equation}
A = \begin{pmatrix}\frac{\phi}{c}\\\mathbf{A}\end{pmatrix},
\end{equation}
where $\mathbf{A}$ is a three dimensional vector potential and $\phi$ is a scalar
potential.



\section{Relativistic Quantum Mechanics}
Motivated by the knowledge of the relativistic phenomena in classical mechanics
the quantum mechanics where reformulated accordingly, which lead to the substitution
of the Schrödinger equation by the Dirac equation. We will first discuss basic
properties of the relativistic effects on a freely moving particle and afterwards
use these basics for the description of the hydrogen atom and later for the more
complex many particle systems.

\subsection{Klein-Gordon Equation for a Freely Moving Particle}
By utilizing the correspondence principle and substituting the classical variables
by their quantum analoga one obatins for the square of the relativistic energy



\begin{equation}\label{kgg}
E^2\Psi = \left(i\hbar \frac\partial{\partial t}\right)^2\Psi = \left(m^2c^4-\hat{\mathbf{p}}^2c^2\right) \Psi
\end{equation}
This expression is known as the Klein-Gordon equation and describes the movement of
a spinless particle.\cite{kellogg97}
Even though equation (\ref{kgg}) is Lorentz covariant it needs some further
investigation. First, its solutions enables negative energies, which so far have
no proper interpretation and second, the corresponding time-dependent probability
density $\Psi^*(\vec{r},t)\Psi(\vec{r},t)$ is not positive definite, which
contradicts the probabilistic picture of quantum mechanics.

Despite these weaknesses the Klein-Gordon equation was the basis for the
derivation of the Dirac equation.


\subsection{Dirac Equation}
The derivation of the Dirac equation was motivated by the description of the
moving electron. It was found later that it holds for fermions with a spin of
$\frac12$.
Special relativity requires a treatment of time and spacial coordinates on an
equal footing. Since the non-relativistic Schrödinger equation contains the first
derivative with respect to time and the second derivative with respect to the spacial
coordinates, the necessity of a reformulation is obvious.


\subsubsection{Dirac Equation for a Freely Moving Particle}\label{freep}
1928 Dirac established his famous equation which incorporates first order
derivatives of both time and spacial coordinates \cite{Dirac28_1}

\begin{equation}\label{diracg}
i\hbar \frac\partial{\partial t} \Psi = c (\alpha\cdot\hat{\mathbf{p}}+\beta mc) \Psi
\end{equation}

The variables $\alpha$ and $\beta$ have to determined appropriately considering
the relations based on the comparison of the Dirac equation to the Klein-Gordon
equation
\begin{align}
\left[\alpha_i,\alpha_j\right]_+ = \alpha_i\alpha_j + \alpha_j\alpha_i
                                &= 2\delta_{ij}\\
  \alpha_i\beta + \beta\alpha_i &= 0 \\
  \beta^2                       &= 1
\end{align}

Since numbers do not commute, the variables obviously are matrices.
They have eigenvalues of either $+1$ or $-1$, their traces have to be 0
and their dimension has to be even. These requirements are fulfilled by the
three Pauli matrices. However, in order to describe four parameters, a set of
at least four matrices is required. Therefore a dimension of at least four is
mandatory. In case of $4\times 4$ matrices more than one set fulfilles the
requirements. The one conventionally used is constructed from the Pauli matrices
and reads as

\begin{equation}\begin{array}{lll}
\alpha_x= \left(\begin{array}{cccc}
0 & 0 & 0 & 1\\
0 & 0 & 1 & 0\\
0 & 1 & 0 & 0\\
1 & 0 & 0 & 0\end{array}\right)&
\alpha_y = \left(\begin{array}{cccc}
0 & 0 & 0 & -i\\
0 & 0 & i & 0\\
0 & -i& 0 & 0\\
i & 0 & 0 & 0\end{array}\right)&
\alpha_z = \left(\begin{array}{cccc}
0 & 0 & 1 & 0\\
0 & 0 & 0 &-1\\
1 & 0 & 0 & 0\\
0 & -1& 0 & 0
\end{array}\right)
\end{array}\end{equation}

\begin{equation}
\beta = \left(\begin{array}{cccc}
1 & 0 & 0 & 0\\
0 & 1 & 0 & 0\\
0 & 0 &-1 & 0\\
0 & 0 & 0 &-1
\end{array}\right)
\end{equation}


Due to the matrix dimension of the operators, the wave function $\Psi$ has to have
the same dimension and hence is represented by a four dimensional vector

\begin{equation}
\Psi = \begin{pmatrix}
\Psi_1\\\Psi_2\\\Psi_3\\\Psi_4
\end{pmatrix}
\end{equation}

A very convenient way of writing the Dirac equation (\ref{diracg})
using $\alpha$ and $\beta$
is the two-component form

\begin{equation}\label{dirac2}
\left(\begin{array}{cc}
-E +mc^2 & c(\mathbf{\sigma}\cdot\hat{\mathbf{p}})\\
c(\mathbf{\sigma}\cdot\hat{\mathbf{p}}) & -E-mc^2
\end{array}\right)
\begin{pmatrix}\Psi^L\\\Psi^S\end{pmatrix}
=0
\end{equation}
Here, $E$ is the energy of the freely moving particle, $\hat{\mathbf{p}}$ denotes
the three dimensional momentum operator and $\mathbf{\sigma}$ is the row vector
containing the Pauli matrices. In this formulation the wavefunction is partitioned
into the so-called large and small components depicted by the superscripts $L$ and
$S$ of the wavefunctions $\Psi$, respectively.

The latter eigenvalue equation (\ref{dirac2}) can be written as two coupled
differential equations

\begin{equation}\begin{array}{rcl}
(-E+mc^2)\Psi^L + c(\mathbf{\sigma}\cdot\hat{\mathbf{p}}) \Psi^S &=& 0\\
c(\mathbf{\sigma}\cdot\hat{\mathbf{p}}) \Psi^L +(-E-mc^2)\Psi^S &=& 0
\end{array}\end{equation}

which yield a relation between the large and the small component of the
wavefunction, which is important for the description of relativistic
many particle systems.

\begin{equation}\label{zusklgr}
\Psi^S = \frac{c(\mathbf{\sigma}\cdot\hat{\mathbf{p}})}{E+mc^2}\,\Psi^L
\end{equation}

Inserting equation (\ref{zusklgr}) into the first equation in (\ref{dirac2})
yields
\begin{equation}
(-E+mc^2)\Psi^L + \frac{c^2(\mathbf{\sigma}\cdot\hat{\mathbf{p}})^2}{E+mc^2}\,\Psi^L = 0
\end{equation}

In general the energies can be shifted to any meaningful value, which then
serves as a knew reference energy. For a comparison with the non-relativistic result,
it is convenient to shift the energies such
that the kinetic energy of the particle coincides with the positive energy as
in the non-relativistic description. Hence, the energies are shifted by $mc^2$.
The non-relativistic limit in then achieved by letting $c$ approach infitnity.

\begin{equation}
E'\Psi^L = \frac{c^2(\mathbf{\sigma}\cdot\hat{\mathbf{p}})^2}{E'+2mc^2}\,\Psi^L\label{hamneu}
\end{equation}

\begin{equation}
\lim_{c\rightarrow\infty} E'\Psi^L = \frac{(\mathbf{\sigma}\cdot\hat{\mathbf{p}})^2}{2m} \qquad\text{with}\quad (\mathbf{\sigma}\cdot\hat{\mathbf{p}})^2=\hat{\mathbf{p}}^2\label{nonrelgr}
\end{equation}

Gleichung \ref{nonrelgr} ist offensichtlich das nicht-relativistische Eigenwertproblem eines freien Teilchens mit der gro?en Komponente als Eigenfunktion. Ihre beiden Komponenten k?nnen dabei mit den beiden m?glichen Spinzust?nden des Teilchens identifiziert werden.

Die L?sung der Dirac-Gleichung f?hrt sowohl zu positiven als auch zu negativen Energien $E_+$ und $E_-$
\begin{equation}
E_\pm = \pm \sqrt{m^2c^4+p^2c^2}.
\end{equation}
Es gibt also ein kontinuierliches Spektrum an Energiezust?nden mit $mc^2\le E_+<\infty$ und $-\infty<E_-\le -mc^2$, mit einer Energiel?cke von $2mc^2$ zwischen den Kontinua. Es gilt nun, diese Energiezust?nde zu interpretieren. Ruhende Elektronen ($p=0$) haben immer mindestens die Energie $mc^2$, besitzen also positive Energien. Andererseits m?ssten die Elektronen bei m?glichen unbesetzen negativen Zust?nden spontan unter Abgabe elektromagnetischer Strahlung in diese negativen Zust?nde wechseln. Dieses unphysikalische Verhalten wurde von Dirac durch das Postulat, dass alle Zust?nde negativer Energie besetzt seien, behoben. Betrachtet man eine Anregung aus dem negativen Kontinuum zu den positiven Energien f?hrt das zu einem zus?tzlichen Elektron und einem verbleibenden Loch im negativen Kontinuum. Dieses Loch l?sst sich wie ein Teilchen beschreiben, das aufgrund der Ladungsneutralit?t des Gesamtsystems eine positive Ladung tr?gt, aber sonst die gleichen intrinsischen Eigenschaften hat wie das Elektron. Diese Positronen, denen die negativen Energien zugeordnet werden k?nnen, konnten 1933 erstmals experimentell nachgewiesen werden\cite{Anderson33} und f?hrten zu einer Verifizierung der Dirac-Gleichung. Die Theorie, die unter anderem die Paarerzeugung richtig beschreibt, ist die Quantenelektrodynamik, die hier nicht n?her erl?utert werden soll, die jedoch auch bei der Beschreibung der Wechselwirkung zwischen mehreren Elektronen relevant sein wird.


\subsubsection{L?sung der Diracgleichung f?r das Wasserstoffatom}\label{hatom}
Im Wasserstoffatom bewegt sich das Elektron nicht mehr frei, sondern im Coulombfeld des Kerns. Dementsprechend muss das vierkomponentige Potential $A$ in der Dirac-Gleichung ber?cksichtigt werden. Da das durch den Kern verursachte Coulombfeld ein Zentralfeld ist, verschwindet das Vektorpotential ($\mathbf{A}=0$) und lediglich der skalare Anteil von $A$, der in diesem Fall die Form $\phi=\frac{eZ}r$ besitzt, ist relevant.
Die zeitunabh?ngige Diracgleichung f?r ein Elektron im Zentralfeld ergibt sich damit zu
\begin{equation}
\left(e\phi + c(\alpha\cdot\hat{\mathbf{p}})+\beta mc^2\right) \Psi = E \Psi.
\end{equation}

Im nicht-relativistischen Fall werden die L?sungen durch Separation der Wellenfunktion in einen Radialteil $R(r)$ und einen Winkelanteil $Y(\theta,\phi)$ erhalten.

\begin{equation}
\Psi^{NR}(r,\theta,\phi) = \frac1r R_{nl}(r)Y^{ml}_l(\theta,\phi)
\end{equation}
Die L?sungen des Winkelanteils $Y_l^{m_l}(\theta,\phi)$ sind dabei die Kugelfl?chenfunktionen.

Analog scheint es sinnvoll, in der relativistischen Betrachtung ebenfalls einen Separationsansatz vorzunehmen, sodass sich aufgrund der Struktur des relativistischen Problems Gleichung \ref{seprel} ergibt.

\begin{equation}\label{seprel}
\Psi(r,\theta,\phi,\tau) = \frac1r \left( \begin{array}{rl}
P_{n\kappa}(r)  & \xi_{\kappa,m}(\theta,\phi)\\
iQ_{n\kappa}(r) & \xi_{-\kappa,m}(\theta,\phi)
\end{array}\right)
\end{equation}

Hierbei sind  $P_{n\kappa}(r)$ und $Q_{n\kappa}(r)$ die Radialteile der gro?en bzw. der kleinen Komponente des Spinors und $\xi_{\kappa,m}(\theta,\phi)$ sowie $\xi_{-\kappa,m}(\theta,\phi)$ zweikomponentige Spinoren, die den Winkelanteil beschreiben. Wie im nicht-relativistischen Fall sind die Radialanteile abh?ngig von der Hauptquantenzahl $n$ und die Winkelanteile von der magnetischen Quantenzahl $m_l$. Zur eindeutigen L?sung der relativistischen Winkelanteile muss noch die relativistische Quantenzahl $\kappa=\mp(j+\frac12)$ eingef?hrt werden, da sonst aufgrund der zwei m?glichen Spinzust?nde $s=\pm \frac12$ keine bijektive Abbildung zwischen einer gegebenen Drehimpulsquantenzahl $j=l+s$ und einer relativistischen Winkelfunktion existiert.

Letzenendes ergeben sich die Winkelfunktionen f?r ein Teilchen mit Spin $\frac12$ zu

\begin{equation}\begin{array}{rcll}
\xi_{\kappa,m} &=& \frac1{\sqrt{2l+1}}\begin{pmatrix}\sqrt{l+m_l+1}Y_l^{m_l}\\\sqrt{l-m_l}Y_l^{m_l+1}\end{pmatrix} &\quad\text{f?r }\kappa<0\\\vphantom{\rule{0pt}{30pt}}
\xi_{\kappa,m} &=& \frac1{\sqrt{2l+1}}\begin{pmatrix}-\sqrt{l-m_l}Y_l^{m_l}\\\sqrt{l+m_l+1}Y_l^{m_l+1}\end{pmatrix} &\quad\text{f?r }\kappa>0\\
\end{array}\end{equation}

In der folgenden Tabelle \ref{qzspinor} ist die eindeutige Zuordnung der Quantenzahlen zu den atomaren Spinoren illustriert.

\begin{table}[h]
\centering
\begin{tabular}{lccccccc}
\toprule
        & $s_{\frac12}$ & $p_{\frac12}$ & $p_{\frac32}$ & $d_{\frac32}$ & $d_{\frac52}$ & $f_{\frac52}$ & $f_{\frac72}$\\\vphantom{\rule{0pt}{20pt}}
$l$     & 0            & 1           & 1            & 2            & 2            & 3            & 3\\\vphantom{\rule{0pt}{20pt}}
$j$     & $\frac12$    & $\frac12$   & $\frac32$    & $\frac32$    & $\frac52$    & $\frac52$    & $\frac72$\\\vphantom{\rule{0pt}{20pt}}
$\kappa$& -1           & 1           & -2           & 2            & -3           & 3            & -4\\
\bottomrule
\end{tabular}
\caption{Die Zuordnung der atomaren Bezeichnungen der Spinoren zu den Quantenzahlen.\cite{dyallfaegri}}
\label{qzspinor}
\end{table}

Mit den obigen Betrachtungen kann dann ein Ausdruck f?r die Energie gebundener Zust?nde bestimmt werden.
\begin{equation}\label{HErel}
E = mc^2 \left[1+\frac1{c^2}\left(\frac1{n-|\kappa|+\sqrt{\kappa^2-\frac1{c^2}}}\right)^2\right]^{\frac12}
\end{equation}
Die Energie eines Zustandes ist demnach von der Hauptquantenzahl $n$ und der relativistischen Quantenzahl $\kappa$ abh?ngig.
Aus der Betrachtung der $\kappa$-Abh?ngigkeit von Gleichung \ref{HErel} l?sst sich schlie?en, dass f?r eine gegebene Drehimpulsquantenzahl $l$ bei einer Aufspaltung der Energieniveaus der Zustand mit dem kleineren Gesamtdrehimpuls $j$ immer die niedrigere Energie besitzt.

\subsection{Behandlung von Vielteilchensystemen in der relativistischen Quantenchemie}
\subsubsection{Dirac-Hartree-Fock}
Bis hierher wurde lediglich die Wechselwirkung zwischen einem Elektron mit einer Ladung diskutiert. In den allermeisten Systemen liegen jedoch mehrere Elektronen vor, deren Wechselwirkung untereinander ebenfalls ber?cksichtigt werden muss. In der nicht-relativistischen Quantenmechanik wird die G?ltigkeit der Born-Oppenheimer-N?herung angenommen und f?r die L?sung des elektronischen Problems die Hartree-Fock-Methode herangezogen. Die Herleitung der Hartree-Fock-Gleichungen verwendet einen Hamilton-Operator eines $N$-Elektronen-Systems, der in einen Einteilchenteil, der die Elektronen im Coulombfeld des Kernes angibt, sowie einen Zweiteilchenteil $\mathbf{\hat{g}}_{ij}$, der die Wechselwirkung der Elektronen untereinander beschreibt und im nicht-relativistischen Fall durch $\frac1{r_{ij}}$ gegeben ist, unterteilt wird.

\begin{equation}
\mathcal{H}= \sum\limits_{i=1}^N \hat{h}(i) + \mathbf{\hat{g}}_{ij}
\end{equation}

Aus den Betrachtungen des Abschnittes \ref{hatom} ?ber das Wasserstoffatom ist der Operator des Einteilchenproblemes bekannt. Somit ergibt sich der relativistische Hamiltonoperator zu:

\begin{equation}
\mathcal{H}_{D}= \sum\limits_{i=1}^N (-e \phi(r_i)+c(\hat{\alpha}^{i}\cdot\hat{\mathbf{p}}^{i})+mc^2) + \mathbf{\hat{g}}_{ij}
\end{equation}

F?r die Ableitung des Wechselwirkungstermes ist eine Betrachtung in der Quantenelektrodynamik n?tig. Durch Entwicklung des Ergebnisses nach Ordnungen der Sommerfeldschen Feinstrukturkonstante $\alpha=\frac1c$ und Abbrechen nach dem dem Term zweiter Ordnung in $\alpha$ ergibt den sogenannten Coulomb-Breit-Term $\hat{g}_{ij}^{CB}$, der sich additiv aus der nicht-relativistischen Coulomb-Absto?ung $\hat{g}_{ij}^C=\frac1{r_{ij}}$ und dem Breit-Term $\hat{g}_{ij}^B$ zusammensetzt.

\begin{equation}\begin{array}{rcl}
\hat{g}_{ij}^{CB} &=& \hat{g}_{ij}^C + \hat{g}_{ij}^B\\
                &=& \frac1{r_{ij}} - \frac1{r_{ij}}\left(\alpha_i\cdot\alpha_j+\frac{(\alpha_i\cdot r_{ij})(\alpha_j\cdot r_{ij})}{r^2_{ij}}\right)
\end{array}\end{equation}

Die Implementierung dieses Ansatzes ist schwierig, weshalb in erster N?herung statt des Breit-Operators der Gaunt-Operator $\hat{g}_{ij}^G$ Verwendung findet, dessen Beitrag zum Breit-Operator der gr??te ist.

\begin{equation}
\hat{g}_{ij}^G = - \frac1{r_{ij}}\alpha_i\cdot\alpha_j
\end{equation}

Dieser Term beschreibt die magnetische Wechselwirkung zweier Elektronen. Ist diese f?r das betrachtete System oder den betrachtete Prozess nicht relevant, kann dieser Term ebenfalls vernachl?ssigt werden, wodurch sich der Dirac-Coulomb-Operator $\mathcal{H}_{DC}$ ergibt.
\begin{equation}\label{DCHamil}
\mathcal{H}_{DC} = \sum\limits_{i=1}^N (-e \phi(r_i)\mathds{1}+c\hat{\alpha}^{i}\cdot\hat{\mathbf{p}}^{i}+mc^2) + \sum\limits_{i>j}^N \frac1{r_{ij}}
\end{equation}



Da gezeigt wurde, dass die Ionisierungsenergien der ?u?eren Valenz sich beim Weglassen des Gaunt-Terms beim TlH nur sehr gering um etwa \unit[0.06]{eV} ?ndern\cite{Pernpointner05}, wird angenommen, dass dieses auch f?r andere Systeme gilt und in dieser Arbeit der Dirac-Coulomb-Operator verwendet. Obwohl dieser Operator aufgrund der Form des Zweiteilchenoperators nicht lorentzinvariant ist, liefert er eine gen?gend genaue Beschreibung der Systeme.

Vom Dirac-Coulomb-Operator (Gleichung \ref{DCHamil}) ausgehend wird analog zur nicht-relati\-vis\-tischen Quantenmechanik angenommen, dass sich die Wellenfunktion als Slaterdeterminante, also als antisymmetrisiertes Produkt von molekularen Einteilchenwellenfunktionen $\phi_i$ schreiben l?sst.

\begin{eqnarray}
\Psi &=& \mathcal{A} \left(\prod\limits_{i=1}^N\phi_i(r_i)\right)\\\vphantom{\rule{0pt}{50pt}}
     &=& \frac1{\sqrt{n!}}\begin{vmatrix}
\phi_1(r_1) & \phi_2(r_1) & \cdots & \phi_N(r_1)\\
\phi_1(r_2) & \phi_2(r_2) & \cdots & \phi_N(r_2)\\
\vdots      & \vdots     & \ddots  & \vdots\\
\phi_1(r_N) & \phi_2(r_N) & \cdots & \phi_N(r_N)\\
\end{vmatrix}
\end{eqnarray}

Mit diesem Ansatz ergibt sich der Erwartungswert der Energie als Funktional:

\begin{eqnarray}
  E &=& \braket{\Psi|\mathcal{H}_{DC}|\Psi}\\\vphantom{\rule{0pt}{30pt}}
  &=& \sum\limits_{i=1}^N\braket{\phi_i|\hat{h}_D(i)|\phi_i}  + \frac12 \sum\limits_{i,j}^N\left(\braket{\phi_i\phi_j|\frac1{r_{ij}}|\phi_i\phi_j} - \braket{\phi_i\phi_j|\frac1{r_{ij}}|\phi_j\phi_i}\right)
\end{eqnarray}

Nach einigen Umformungen werden die Dirac-Hartree-Fock-Gleichungen, die jedes Elektron in einem gemittelten Feld $v^{DHF}(i)$ beschreiben, das durch alle anderen Elektronen hervorgerufen wird, erhalten.

\begin{eqnarray}
\mathbf{f}_i\ket{\phi_i} &=& \left(\mathbf{h}_D(i) + v^{DHF}(i)\right)\ket{\phi_i}\\
                         &=& \left( \mathbf{h}_D(i) + \sum\limits_{l=1}^{N}(\mathbf{J}_l-\mathbf{K}_l)\right)\ket{\phi_i}\\
                         &=& \varepsilon_i\ket{\phi_i}
\end{eqnarray}

Die Operatoren $\mathbf{J}_l$ und $\mathbf{K}_l$ sind dabei der Coulomb- und der Austausch-Term, die jeweils von den Einteilchenwellenfunktionen abh?ngen. Durch wissenschaftliches Raten einer Anfangsbasis wird der erste Fockoperator $\mathbf{F}$ konstruiert und dieser angewendet, wobei ein neuer Satz an Wellenfunktionen erzeugt wird, aus denen ein neuer Fockoperator aufgebaut wird. Dieses iterative Verfahren wird bis zur Selbstkonsistenz angewandt, weshalb man auch von einer \emph{self-consistent-field} Methode spricht.\cite{szabo}

In der nicht-relativistischen Quantenmechanik liegt die so erhaltene Energie aufgrund des Variationsprinzips immer oberhalb der exakten Energie des Systems. Diese Eigenschaft ist bei der relativistischen Betrachtung nicht gegeben, da die niedrigste Energie, wie in Abschnitt \ref{freep} diskutiert, zu einem positronischen Zustand geh?rt. Um den sogenannten \emph{variational collapse} zu verhindern, bei dem eine positronische L?sung erhalten wird, muss die gew?hlte Basis gewissen Bedingungen unterliegen, auf die in Abschnitt \ref{basis} bei der Konstruktion von Basiss?tzen noch eingegangen wird.
 
Zur tats?chlichen Berechnung der Energien werden die Einteilchenwellenfunktion \{$\phi_i$\} des Gesamtsystems durch eine Linearkombination atomarer Wellenfunktionen erhalten. Diese wiederum werden in einen endlichen Satz von atomaren Basisfunktionen \{$\varphi_\mu$\} f?r die gro?e und die kleine Komponente entwickelt. F?r die molekulare Einteilchenwellenfunktion ergibt sich damit
\begin{equation}
\phi_i = \sum\limits_\mu^{N^L} c_{\mu\alpha}^{iL} \varphi_{\mu\alpha}^L +  \sum\limits_\mu^{N^L} c_{\mu\beta}^{iL} \varphi_{\mu\beta}^L + \sum\limits_\mu^{N^S} c_{\mu\alpha}^{iS} \varphi_{\mu\alpha}^S + \sum\limits_\mu^{N^S} c_{\mu\beta}^{iS} \varphi_{\mu\beta}^S.
\end{equation}

Hierbei sind $N^L$ sowie $N^S$ die Anzahl der Basisfunktionen der gro?en bzw. der kleinen Komponente, die nicht notwendigerweise gleich sein m?ssen. Die Energie ist also eine Funktion der Entwicklungskoeffizienten $c_{\mu s}^{iX}$, wobei $X=L,S$ die Komponente des Spinors beschreibt und $s$ den Spinzustand. Die resultierenden relativistischen Roothan-Gleichungen lassen sich in der Matrixschreibweise zu

\begin{equation}
\mathcal{FC} = \mathcal{SCE}
\end{equation}

zusammenfassen. Die Fockmatrix $\mathcal{F}$ und die ?berlappmatrix $\mathcal{S}$ bestehen dabei aus den in Gleichung \ref{roothanm} angegebenen Matrixelementen.

\begin{equation}\begin{array}{rcl}\label{roothanm}
F^{XY,sp}_{\mu\nu} &=& \braket{\varphi^X_{\mu s}|\mathbf{f}_i|\varphi_{\nu p}^Y}\\
S^{XY,sp}_{\mu\nu} &=& \braket{\varphi^X_{\mu s}|\varphi_{\nu p}^Y}
\end{array}\end{equation}


\subsection{Zweikomponentige Methoden}
Wie bereits erw?hnt wurde, handelt es sich bei der Diracgleichung um ein System gekoppelter Differentialgleichungen, die hier noch einmal aufgef?hrt werden sollen.

\begin{equation}\label{koppdirac}
\left(\begin{array}{ll}
V & c(\mathbf{\sigma}\cdot\hat{\mathbf{p}})\\
c(\mathbf{\sigma}\cdot\hat{\mathbf{p}}) & V -2mc^2\\\end{array}\right)
\begin{pmatrix}\Psi^L\\\Psi^S\end{pmatrix}
= \begin{pmatrix}\Psi^L\\\Psi^S\end{pmatrix} E
\end{equation}

Dabei ist die kleine Komponente des Spinors an die gro?en Komponente gekoppelt, wobei der verkn?pfende Operator $\hat{R}$ sei.
\begin{equation}
\Psi^S = \hat{R}\Psi^L
\end{equation}

Die Grundidee der zweikomponentigen Methoden ist, durch eine unit?re Transformation des Hamiltonoperators eine Entkopplung der L?sungen positiver und negativer Energien zu erlangen. Dadurch k?nnten nur die positiven Energien unter ausschlie?licher Verwendung der gro?en Komponente der Basis berechnet werden.

\begin{equation}
\hat{U}^\dagger\left(\begin{array}{cc}
\hat{h}_{11} & \hat{h}_{12}\\
\hat{h}_{21} & \hat{h}_{22}\\\end{array}\right) \hat{U}=
\left(\begin{array}{cc}
\hat{h}_+ & 0 \\
0         & \hat{h}_-
\end{array}\right)
\end{equation}

Dabei hat die Transformationsmatrix $\hat{U}$ folgende Struktur:
\begin{equation}
\hat{U} = \left(\begin{array}{cc}
\frac1{\sqrt{1+\hat{R}^\dagger\hat{R}}}          & \frac1{\sqrt{1+\hat{R}^\dagger\hat{R}}} \hat{R}^\dagger\\
-\frac1{\sqrt{1+\hat{R}\hat{R}^\dagger}} \hat{R} & \frac1{\sqrt{1+\hat{R}\hat{R}^\dagger}}
\end{array}\right)
\end{equation}

Da die Nichtdiagonalelemente des transformierten Hamiltonoperators 0 sein m?ssen, ergibt sich die Bedingung
\begin{equation} \label{nondigbed}
2mc^2\hat{R} = c (\mathbf{\sigma}\cdot\hat{\mathbf{p}}) + \left[V,\hat{R}\right] - \hat{R}c(\mathbf{\sigma}\cdot\hat{\mathbf{p}})\hat{R}.
\end{equation}
Im nicht-relativistischen Grenzfall ist der Kopplungsoperator $\hat{R}=\frac{(\mathbf{\sigma}\cdot\hat{\mathbf{p}})}{2mc}$ bekannt. Bei dessen Wahl als Startpunkt f?r eine iterative L?sung der Gleichung \ref{nondigbed} wird ersichtlich, dass immer h?here Ableitungen von $\hat{R}$ auftreten. Der Operator muss also angen?hert werden, wobei die beiden etabliertesten Methoden die Foldy-Wouthuysen-Methode und die Douglas-Kroll-Methode sind, die unterschiedliche Herangehensweise w?hlen. W?hrend die Foldy-Wouthuysen-Methode einen st?rungstheoretischen Ansatz verwendet, verwendet die Douglas-Kroll-Methode einen variationellen.

Ein neuerer Ansatz manipuliert nicht den Kopplungsoperator $\hat{R}$, sondern verwendet eine algebraische Herangehensweise. Unter einmaliger Verwendung des Kopplungsoperators im nicht-relativistischen Grenzfall wird zun?chst die Dirac-Gleichung \ref{koppdirac} gel?st, in der die komplette Transformation enthalten ist. Aus den L?sungen wird auf die komplette Kopplungsmatrixmatrix $\mathbf{R}$ zur?ckgeschlossen, mit deren Hilfe dann die entkoppelten Hamiltonoperatoren f?r die postivien und die negativen Energien $\hat{h}_+$ und $\hat{h}_-$ konstruiert werden k?nnen. Auf diese Weise werden Operatoren erhalten, die im Vergleich zu denen der st?rungstheoretischen Foldy-Wouthuysen-Methode f?r die gew?hlte Basis in unendlicher Ordnung genau sind, weshalb die Methode IOTC (\emph{infinite-order two-component})\cite{Ilias07} genannt wird.\\
Soll das IOTC nicht nur f?r die Beschreibung von Einteilchen-, sondern auch von Vielteilchensystemen verwendet werden, wird die sogenannte AMFI-Korrektur (\emph{atomic mean-field approximation to the spin-orbit interaction}) hinzugef?gt, die die Spin-Bahn-Kopp\-lung beschreibt und damit skalarrelativistische Ergebnisse um diesen Anteil korrigieren kann. Da bei der Berechnung der Korrektur nur die Integrale ?ber die Orbitale, die am selben Atom lokalisiert sind, ber?cksichtigt werden, sind f?r die Betrachtung von Molek?len geringe Abweichungen zum vierkomponentig ermittelten Resultat zu erwarten.\cite{Hess96}
 



\subsection{Konstruktion einer Basis f?r die Beschreibung relativistischer Probleme}\label{basis}
Wie bereits diskutiert, werden die vier Komponenten des Spinors in eine endlichdimensionale Basis atomarer Einteilchenfunktionen entwickelt. Dabei werden die Basiss?tze der gro?en und der kleinen Komponente getrennt behandelt.

\begin{equation}\begin{array}{ccc}
\phi_i^L = \begin{pmatrix}\phi^L_{i\alpha}\\\phi^L_{i\beta}\\0\\0\end{pmatrix} &\hspace{3em} &\phi_j^S = \begin{pmatrix}0\\0\\\phi^S_{j\alpha}\\\phi^S_{j\beta}\end{pmatrix}
\end{array}\end{equation}

Die einzelnen Komponenten werden als Linearkombination von skalaren atomaren Basisfunktionen $\varphi_\mu^X$ ausgedr?ckt, die durch die Entwicklungskoeffizienten $c_{\mu s}^{iX}$ gewichtet werden.

\begin{equation}\begin{array}{ccc}
\phi^L_{is} = \sum\limits_{\mu=1}^{N^L} c_{\mu s}^{iL} \varphi_\mu^L &\hspace{3em}& \phi^S_{is} = \sum\limits_{\mu=1}^{N^S} c_{\mu s}^{iL} \varphi_\mu^S
\end{array}\end{equation}

Der Ausgangspunkt f?r die Konstruktion der Basisfunktionen sind die atomzentrierten L?sungen des Wasserstoffatoms, die die Form
\begin{equation}
s_\tau^X = N_\tau^X r^{n-1}\mathrm{e}^{\alpha_\tau^X r}Y_l^{m_l}(\theta,\varphi)
\end{equation}
besitzen. Die analytische L?sung der Integrale, die bei Verwendung dieser \emph{slater-type-functions} in einer DHF-Rechnung auftreten, ist nicht m?glich, weshalb stattdessen Linearkombinationen primitiver Gau?funktionen $g_\tau^X$, die kontrahierte Gau?funktionen bilden, verwendet werden. Diese liefern zwar eine inkorrekte Beschreibung f?r das asymptotische Verhalten bei kleinen und gro?en Abst?nden ($r\rightarrow 0$, $r\rightarrow\infty$), erm?glichen jedoch die einfache analytische L?sung der Integrale. Sie sind in Kugelkoordinaten gegeben durch
\begin{equation}
g_\tau^X = N_\tau^Xr^l\mathrm{e}^{-\alpha_\tau^Xr^2}Y_l^{m_l}(\theta,\varphi)
\end{equation}
bzw. in kartesischen Koordinaten durch
\begin{equation}
g_\tau^X = N_\tau^X x^{n_x^X}y^{n_y^X}z^{n_z^X} \mathrm{e}^{-\alpha_\tau^Xr^2}.
\end{equation}
Dabei ist $N_\tau^X$ ein Normierungsfaktor und die Summe aus den Exponenten der Raumkoordinaten $n_i^X$ korrespondiert mit der Drehimpulsquantenzahl $l$.
\begin{equation}
n_x^X+n_y^X+n_z^X=l
\end{equation}
Die Kontraktion liefert dann die endg?ltigen Basisfunktionen
\begin{equation}
\varphi^X_\mu = \sum\limits_\tau d_{\mu\tau}^Xg_\tau^X,
\end{equation}
wobei der Faktor $d_{\mu\tau}^X$ die Beitr?ge der einzelnen primitiven Gau?funktionen zur kontrahierten Gau?funktion gewichtet.

Um zu verhindern, dass bei einer variationellen Rechnung L?sungen negativer Energien erhalten werden, wird die Bedingung der \emph{kinetic balance}
\begin{equation}\label{kinbal}
\phi_S = \frac1{2mc}(\sigma\cdot\hat{p})\,\phi_L
\end{equation}
 bei der Konstruktion der kleinen Komponente verwendet. Diese entspricht dem nicht-relativistischen Grenzfall der Gleichung \ref{zusklgr}. F?r die Grenzwertbetrachtung wird die Energie um $mc^2$ verschoben. Da die Energie des Elektrons $E$ wesentlich kleiner als $2mc^2$ ist, wird diese vernachl?ssigt und somit Gleichung \ref{kinbal} erhalten.
Das Einhalten dieser Beziehung zwischen gro?er und kleiner Komponente, die aus dem nicht-relativistischen Grenzfall resultiert, bewahrt bei einer Variationsrechnung vor dem \emph{variational collapse}.

Der Operator der kinetischen Energie $\alpha\cdot\hat{\pi}$ kreiert aus einer Funktion der gro?en Komponente mit einem bestimmten $l$-Wert jeweils eine Funktion der kleinen Komponente mit $l-1$ bzw. $l+1$. Daraus folgt, dass die Basis der kleinen Komponente fast doppelt so gro? ist, wie die der gro?en Komponente und dass die Anzahl der Basisfunktionen einer relativistischen Rechnung wesentlich gr??er ist, als bei einer vergleichbaren nicht-relativistischen Rechnung. Aus diesem Grund bietet sich auch die Verwendung sogenannter dualer Basiss?tze an. Dabei werden f?r die Funktionen der gro?en Komponente gerader bzw. ungerader Drehimpulsquantenzahl jeweils gleiche Exponenten verwendet, aus denen die Funktionen der kleinen Komponente erzeugt werden. Hierdurch verkleinert sich die Basis f?r die kleine Komponente, wobei allerdings die Flexibilit?t der gro?en Basis eingeschr?nkt wird.
