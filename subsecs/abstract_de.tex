\chapter*{Kurzfassung}
\thispagestyle{empty}

Der Interatomic/ Intermolecular Coulombic Decay (ICD) sowie der
Electron Transfer Mediated Decay (ETMD) sind elektronische Zerfallsprozesse,
die in einer Vielzahl von Systemen vorkommen, von kleinen
Edelgasclustern bis hin zu biologischen Systemen. Sofern schwere Atome
an diesen Prozessen beteiligt sind, können relativistische Effekte
nicht von der Betrachtung ausgeschlossen werden. Ihr Einfluss ist allerdings
bislang nicht eingehend untersucht worden.

In dieser Abhandlung werden sowohl der Einfluss der Spin-Bahn-Kopplung
als auch skalarrelativistischer Effekte auf die Kanalöffnungen bzw.
Kanalschließungen und die Zerfallsbreiten der jeweiligen Kanäle untersucht.
Hierzu wurden asymptotische Formeln für die Zerfallsbreiten des ICD und ETMD
hergeleitet, welche eine Abschätzung der Zerfallsbreiten auf Grundlage
experimenteller Daten der Einzelkomponenten erlauben.
Eine genauere Beschreibung der Zerfallsbreiten erforderte die Übertragung
der nicht-relativistischen FanoADC-Stieltjes Methode auf die relativistische
Behandlung und ihrer Implementierung im relativistischen, quantenchemischen
Programmpaket Dirac. Hiermit wurden kleine Edelgassysteme studiert.

Die experimentellen Untersuchungen dieser Zerfallsprozesse werden vorwiegend an
Edelgasclustern bestehend aus 100--2000 Atomen durchgeführt, die mit
quantenchemischen \emph{ab initio} Methoden aufgrund ihrer Größe
nicht mehr behandelt werden können.
Um eine Vergleichbarkeit herzustellen wurde der Einfluss der Clusterumgebung
auf die Zerfallsprozesse untersucht und auf dieser Grundlage eine
Methode zur Simulation von Sekundärelektronenspektren entwickelt. Diese
beruht wahlweise auf den hergeleiteten asymptotischen Formeln oder
in Abhängigkeit von der Geometrie berechneten Zerfallsbreiten. Mit Hilfe dieser
im Programm HARDRoC automatisierten Heransgehensweise wurden die konkurrierenden
Zerfallskanäle ICD und ETMD in ArXe Clustern untersucht. Darüberhinaus
bildet sie die Grundlage für eine neue Methode zur Strukturaufklärung
heteroatomarer Edelgascluster, welche am Beispiel von NeAr Clustern
erläutert wird.
