\section{\textsc{icoclus}}

Icoclus is a selection of python scripts creating xyz-coordinate
files for idealized heteronuclear clusters with a basic
icosahedral structure.
All structures contain an icosahedral core of one atom type
surrounded by atoms of the second atom type in different ways.
Hereby it has to be mentioned, that these cluster structures are not
energetically optimized but build using the scientific guess
that the atomic distances of each pair of atoms in the clusters
can be described as the sum of the van der Waals
radii.

\subsection{Construction of the Core Icosahedral Cluster}
The vertex coordinates of a regular icosahedron are given by

\begin{equation}
 ,
\end{equation}
where $a$ describes both the distance from the center to each vertex
and the length of an edge and $\phi =$.

Icosahedral clusters are build of different shells, each of them
characterized by the number of atoms per edge $c$. Starting from a
central atom, a shell with $c=2$ is added, afterwards one with $c=3$
and so on, until the desired cluster size is achieved.

Here one needs to consider, that in these packed icosahedral shells
the distance between the closest atoms within one shell is
larger than the closest distance of two atoms of neighbouring shells.
In order to achieve the overall minimum distance between two atoms
to be the sum of their van der Waals radii, we scale our vertex
distance $a$ such, that the distance between the surfaces is given by
the sum of the v.d. Waals radii.

\begin{equation}
 a' =
\end{equation}

\subsection{Complete Shells}
Around the icosahedral core cluster more complete shells consisting
of the second atom type are constructed. Hereby the distance between
the central atom and the vertices is calculated as the sum over the
contributing v. d. Waals radii. and scaled as in the case of the core
vertices.

\subsubsection{Manual of \lstinline|icoclus.py|}
In the header of the script \lstinline|icoclus.py| the following
controlling variables are defined.

\begin{lstlisting}
##################Input Variables ##################################
   atcore  = 'Ar' # atomtype of the core atoms
   atouter = 'Ne' # atomtype of the outer shells
   
   rcore =  1.88  # radius of core atoms
   router = 1.54  # radius of outer shell atoms
   
   n_core = 1     # number of atoms in the longest edge
   #n_outer = raw_input('How many layers of atoms do you want to have? ')
   #n_outer = int(n_outer)
   n_outer = 1    # number of additional shells
\end{lstlisting}

These are to be adjusted according to the desired structure.
Afterwards the script is run from the terminal.
The two commented lines are convenient if several cluster structures
of the same type but with different numbers of second atom type shells.
In this case the two commented lines need to be uncommented and the last
line to be commented out.

\subsection{Incomplete Shells}
It is also possible to generate incomplete shells around a core and
eventually complete shells of the second atom type.

\subsubsection{Manual of \lstinline|incompl_shells.py|}
In the header of \lstinline|incompl_shells.py| the following section
with the
input variables is to be found.

\begin{lstlisting}
##################Input Variables ##################################
   atcore  = 'Ar' # atomtype of the core atoms
   atouter = 'Ne' # atomtype of the outer shells
   
   rcore   = 1.88 # radius of core atoms
   router  = 1.54 # radius of outer shell atoms
   
   n_core  = 1    # number of atoms for the longest edge
   n_sec   = 2    # number of complete layers of the second atom type
   n_outer = 1    # do not change
   
   #no_surfaces = 3 + 1
   no_surfaces = raw_input('How many surfaces do you want to be covered? ')
   no_surfaces = int(no_surfaces)
\end{lstlisting}
Here \lstinline|n_outer| declares the number of shells being
partially filled. Changing it results in a different cluster class.
A nicer way to accomplish structures of this other class is to use the
script \lstinline|area_ico.py|.\\
It is important to notice, that the script is going to fail,
if the number of covered triangular surfaces is 0. In the case of
no additional incomplete shells covered being the desired structure,
one should consider using icoclus.py.


\subsection{Triangular Surfaces Covered by Layers of Atoms}
One might consider triangular surfaces covered by several layers of
atoms of the second atom type, hereby creating subsets of complete
shells as shown in figure \ref{}.

\subsubsection{Manual of \lstinline|area_ico.py|}
In the header of \lstinline|area_ico.py| the following control section
is found with the variables to be adjusted.

\begin{lstlisting}
##################Input Variables ##################################
   atcore  = 'Ar' # atomtype of the core atoms
   atouter = 'Ne' # atomtype of the outer shells
   
   rcore   =  1.88  # radius of core atoms
   router  = 1.54  # radius of outer shell atoms
   
   n_core  = 3     #number of atoms for the longest edge
   n_outer = raw_input('How many layers of atoms do you want to have? ')
   n_outer = int(n_outer)
   #n_outer = 1   # number of layers on top of the selected surfaces
   
   no_surfaces = 5 + 1 # number of triangular surfaces covered
\end{lstlisting}


\subsection{Caps}
\subsubsection{Manual of \lstinline|scatter_cap_ico.py|}

\begin{lstlisting}
##################Input Variables ##################################
   atcore  = 'Ar' # atomtype of the core atoms
   atouter = 'Ne' # atomtype of the outer shells
   
   rcore   = 1.88 # radius of core atoms
   router  = 1.54 # radius of outer shell atoms
   
   n_core  = 4    #number of atoms for the longest edge
   #n_core = raw_input('How many core layers do you want to have? ')
   #n_core = int(n_core)
   n_outer = n_core - 1 # do not change
   
   caps   = [1,5]
   n_caps = len(caps)
\end{lstlisting}

\subsubsection{Manual of \lstinline|stat_caps.py|}
\begin{lstlisting}
##################Input Variables ##################################
   atcore  = 'Ar' # atomtype of the core atoms
   atouter = 'Ne' # atomtype of the outer shells
   
   rcore   = 1.88 # radius of core atoms
   router  = 1.54 # radius of outer shell atoms
   
   n_core  = 2    #number of atoms for the longest edge
   n_outer = n_core - 1 # do not change
   
   n_caps  = 2    # number of caps
\end{lstlisting}



\subsection{Randomly Arranged Atoms Around Complete Shells}
\subsubsection{Manual of \lstinline|random_ico.py|}
\begin{lstlisting}
##################Input Variables ##################################
   atcore  = 'Ar'   # atomtype of the core atoms
   atouter = 'O'    # atomtype of the outer shells
   
   rcore   = 1.88   # radius of core atoms
   router  = 1.54   # radius of outer shell atoms
   
   n_core  = 5      #number of atoms for the longest edge
   n_sec   = 0      # number of complete shells of second atom type
   ratio   = 3.8868 #nc_atoms/no_atoms
   
   n_outer = 1      # do not change
\end{lstlisting}

