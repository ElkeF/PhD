\chapter{Resonances}
\label{chapter:resonances}

\section{General Remarks}
A short introduction to scattering theory is essential for the understanding
of resonances in quantum mechanics \cite{nolting52,Gell-Mann53,Taylor87}.

In a general scattering experiment a system $A$ approaches another
system $B$. In the beginning they are freely moving particles and do not interact
with each other. The closer these two particles get, the stronger is their interaction.
At some time $t$ they may form a compound system, which can have a long lifetime.
This meta-stable state can decay via different channels characterized
by internal degrees of freedom of the fragments. If the fragments are the
same system as before the scattering $A$ and $B$ have the same internal
energies and no electronic rearrangement
has occured, the process is called an elastic scattering process.
Otherwise, the process is called an inelastic scattering process.
Several combinations of characterizing properties of the fragments such as
quantum numbers may be possible. Each process characterized by a different
of such combinations is called a channel.

The total Hamiltonian of the system reads

\begin{equation}
 H = H_0 + V
\end{equation}
where $V$ describes the interaction of the involved systems and
$H_0$ is the Hamiltonian of the complete system without this interaction.

The wavefunctions of both the incoming systems and the outgoing
fragments inside the so-called interaction region are unknown.
However, their behaviour and characteristics at infinite times before
and after the scattering process can easily be described.
In the stationary description of the scattering prcess this requires
the functions to fulfill outgoing and incoming boundary conditions
denoted by the superscripts "+" and "-" for the initial and final states,
respectively.
These wavefunctions $\ket{E_i^{(+)}}$ and $\ket{E_f^{(-)}}$ with the correct
asymptotic behaviour
of the initial and final states at the time $t=0$ of the interaction
are constructed from the corresponding states $\ket{E_i^{(0)}}$ and
$\ket{E_f^{(0)}}$ of the system disregarding the interaction $V$.
They are eigenfunctions of $H_0$ and hence
are characterized by the energies $E_n^{(0)}$.
Their connection to the asymptotic wavefunctions is
formulated in the Lippmann-Schwinger equation

\begin{equation} \label{equation:LippmannSchwinger}
 \ket{E_n^{(\pm)}} =  \ket{E_n^{(0)}} + G_n^{(\pm)}V \ket{E_n^{(0)}}
\end{equation}

where
\begin{equation}
 G_n^{(\pm)} = \frac{1}{E_n^{(0)}-H \pm i0^+}
\end{equation}
is the Green's function with $0^+$ being an infinitesimal positive value.

The probability of finding the system with an initial state $\ket{E_i^{(+)}}$
in a distinct final state $\ket{E_f^{(-)}}$ after the scattering process
is given by the absolute square of the corresponding matrix elements
of the scattering matrix

\begin{align}
 {w}_{fi} &= \left | \braket{E_f^{(-)} | E_i^{(+)}} \right|^2\\
                  &= |S_{fi}|^2
\end{align}

These matrix elements are given by
\begin{equation}
 S_{mn} = \delta_{mn} - 2\pi i \delta \left(E_m^{(0)} - E_n^{(0)} \right)
          \braket{E_m^{(0)}| V |E_n^{(+)}}
\end{equation}

The Kronecker delta describes the elastic scattering process and the second
part describes those scattering processes, in which electronic rearrangements
are involved. This second part has poles of different kinds. For energies,
where the real part is positive and the imaginary part is negative, the poles
correspond to resonances ($E_{res}=E_R - i\frac{\Gamma}{2}$),
which hence can be understood as the maximum
response of the system to the interaction \cite{Klaiman12}. 
The real part corresponds to the
energy at which the resonance occurs and $\Gamma$ in the imaginary part
is the so-called decay width, which is connected to the lifetime
$\tau$ of the the meta-stable state

\begin{equation}
 \Gamma = \frac{\hbar}{\tau}
\end{equation}

If the process of interest is an inelastic scattering process the
above scattering matrix is reformulated using the so-called transition
operator

\begin{equation}
 T^{(n)} \equiv V \left( \mathbf{1} + G_n^{(+)} V \right)
\end{equation}

Its action on $\ket{E_n^{(0)}}$ equals the action of $V$ on the asymptotic
states $\ket{E_n^{(-)}}$
\begin{equation}
 V \ket{E_n^{(+)}} = T^{(n)} \ket{E_n^{(0)}}
\end{equation}

Hence, the scattering matrix reads
\begin{equation}
 S_{mn} = \delta_{mn} - 2\pi i \delta \left(E_m^{(0)} - E_n^{(0)} \right)
          \braket{E_m^{(0)}| T^{(n)} |E_n^{(0)}}
\end{equation}

The corresponding $\mathcal{T}$ matrix contains all information about the
inelastic scattering process and is going to be used in order
to find an expression for the calculation of the decay width.
It has poles at the resonance energies and its absolute square has the
shape of a Lorentzian with a \ac{FWHM} of $2\,0^+ = \Gamma$.

%%%%%%%%%%%%%%%%%%%%%%%%%%%%%%%%%%%%%%%%%%%%%%%%%%%%%%%%%%%%%%%%%%%%%

%Then, the two particles fly apart
%either with the same energy distribution as before (elastic scattering) or with
%redistributed energies due to internal electronic rearrangements of the particles
%(inelastic scattering). The spacial region in which the scattering occurs and
%which describes the spacial region, where the particles are not able to move freely is
%called the interaction region.
%
%Particularly interesting to the auotionization processes are those resonances
%characterized by long-lived metastable states. They belong to the group of Feshbach
%resonances \cite{Feshbach58}, where the metastable state is characterized by an
%excitation and the direct decay channels are closed. Therefore, the meta-stable
%state either has to decay into the initial state or to rearrange electronically
%and decay via decay channels of lower energy \cite{Kukulin_book}.
%Autoionization processes are special
%cases of the latter.
%
%
%
%
%
%\section{Decay Widths from a Scattering Point of View}
%For a very general description the solution of the time-dependent Schrödinger
%equation inside the interaction region is inpractical. First of all, the
%choice of the size of the interaction region will crucially influence the
%final result. It is therefore beneficial to change from solving the time-dependent
%Schrödinger equation to solving the time-independent Schrödinger equation
%with proper boundary conditions and following the time evolution of the population
%of these time-independent states. In the case of resonances
%this will lead to so-called resonance states
%with a finite lifetime, a complex energy and the correct asymptotic behaviour.
%
%This means that the aim is to express the resonance with the initial and final
%states $\ket{\phi}$ and $\ket{\chi}$ in terms of their asymptotes
%$\ket{\phi ^{(+)}}$ and $\ket{\chi ^{(-)}}$.
%
%\begin{align}
%  t \rightarrow -\infty:  \ket{\phi}  &\rightarrow \ket{\phi ^{(+)}}\\
%  t \rightarrow \infty :  \ket{\chi}  &\rightarrow \ket{\chi ^{(-)}}
%\end{align}
%Here "$+$" denotes outgoing and "$-$" denotes incoming boundary conditions.
%This means that the approaching particle has outgoing boundary conditions and the
%leaving particle has incoming boundary conditions. Introductions to the matter
%can be found in references \cite{nolting52,Gell-Mann53,Taylor87}.
%
%The Hamiltonian of the complete system including the interaction part
%is given by
%
%\begin{equation}
%  H = H_0 + V
%\end{equation}
%where $H_0$ denotes the Hamiltonian of the full system including
%bound and contiuous states excluding the interaction
%between the initial and final states at infinite times and $V$ denotes the
%operator mediating the interaction of the initial and final states.
%At asymptotic times the Schrödinger equation gives
%
%\begin{equation}
% H_0 \ket{E_n^{(0)}} = E_n^{(0)} \ket{E_n^{(0)}}
%\end{equation}
%with the energy eigenvalues $E_n^{(0)}$ and eigenstates $\ket{E_n^{0}}$ with the
%orthonormality conditions
%
%\begin{equation}
% \int \mathrm{d}n \, \ket{E_n^{(0)}} \bra{E_n^{(0)}} = \mathbf{1} \quad\quad
% \braket{E_m^{(0)} |E_n^{(0)}} = \delta_{mn}
%\end{equation}
%
%Without any interaction the initial state at time $t$ can be described as
%\begin{equation}
% \ket{\phi_0(t)} = \int \mathrm{d}n \, \mathbf{\alpha}_n e^{iE_n^{(0)}t} \ket{E_n^{(0)}}
%\end{equation}
%where $\mathbf{\alpha}_n$ is the matrix of initial state population coefficients.
%Analogously the matrix $\mathbf{\beta}_n$ is described for the final states.
%Their asymptotic time limits can be obtained from
%
%\begin{align}
% \lim_{t \to -\infty} \ket{\phi(t)} &= \lim_{t \to -\infty}
%       \int \mathrm{d}n \, \mathbf{\alpha}_n e^{iE_n^{(0)}t} \ket{E_n^{(0)}}   \\
% \lim_{t \to +\infty} \ket{\phi(t)} &= \lim_{t \to +\infty}
%       \int \mathrm{d}n \, \mathbf{\beta}_n e^{iE_n^{(0)}t} \ket{E_n^{(0)}}   
%\end{align}
%
%Outside the interaction region at infinite times the description of the
%time-dependence is most conviently expressed in the Dirac picture with both
%operators and wavefunctions being time-dependent, labelled by the subscript $D$
%
%\begin{equation}
% \ket{\phi_D(t)} = e^{iH_0t} \ket{\phi(t)}
%\end{equation}
%
%Then the time evolution from one state at time $t'$ to another state at time
%$t$ can be written using the time evolution operator $U_D(t,t')$
%
%\begin{equation}
% \ket{\phi_D(t)} = U_D(t,t')  \ket{\phi_D(t')}
%\end{equation}
%which in the Dirac picture for the complete system is given by
%
%\begin{equation}
% U_D(t,t') = e^{iH_0t} e^{-iH(t-t')} e^{iH_0t'}
%\end{equation}
%and without the interaction, i.e. $H=H_0$ is the unity operator.
%
%Then the probability amplitude to find a particle prepared as $\ket{\phi_D(t')}$
%at time $t'$ in an unperturbed energy eigenstate $\ket{E_m^{(0)}}$ at a later
%time $t$ is written as
%\begin{align}
% \braket{E_m^{(0)} | \phi_D(t)} &= \braket{E_m^{(0)}| U_D(t,t') | \phi_D(t')}\\
%  &= \int \mathrm{d} n \, \braket{E_m^{(0)}| U_D(t,t') |E_n^{(0)}}
%     \braket{E_n^{(0)}|\phi_D(t')}
%\end{align}
%
%These probability amplitudes are valid for any times $t$ and $t'$ and hence also
%for $t\rightarrow +\infty$ and $t' \rightarrow -\infty$
%\begin{align}
% \lim_{t \to +\infty} \braket{E_m^{(0)} | \phi_D(t)} &= \mathbf{\beta}_m   \\
% \lim_{t' \to -\infty} \braket{E_n^{(0)} | \phi_D(t)} &= \mathbf{\alpha}_n
%\end{align}
%
%From the latter equations an expression for the final state population coefficients
%from a given final state population can be derived as
%\begin{align}
% \mathbf{\beta}_m = \int \mathrm{d} n \,
%     \braket{E_m^{(0)}| U_D(+\infty,-\infty) |E_n^{(0)}} \mathbf{\alpha}_n
%\end{align}
%
%This can be formulated in terms of the scattering matrix $S$ as
%
%\begin{equation}
%  \mathbf{\beta}_m = \int \mathrm{d} n \, S_{mn} \mathbf{\alpha}_n
%\end{equation}
%where the scattering matrix is given by
%
%\begin{equation}
%  S_{mn} = \braket{E_m^{(0)}| U_D(+\infty,-infty) |E_n^{(0)}}
%\end{equation}
%
%It mediates the transition probability amplitudes between the non-interacting
%initial and final states. A closer look on the scattering matrix reveals another
%formulation using the asymptotic limits of the initial and final states:
%
%\begin{equation}
% S_{mn} = \lim_{t \to \infty} \lim_{t' \to -\infty}
%  \braket{E_m^{(0)}| e^{i(E_m^{(0)}-H)t} e^{-i(E_n^{(0)}-H)t'} |E_n^{(0)}}
%\end{equation}
%
%The asymptotic limit of some function $f(t)$ can be obtained by
%\begin{align}
% \lim_{t \to +\infty} f(t) &= f(\infty) \int\limits_0^\infty \mathrm{d}x \, e^{-x}\\
%   &= \lim_{\eta \to 0^+} \int\limits_0^\infty \mathrm{d}x \, e^{-x}
%           f\left( \frac{x}{\eta}\right)    \\
%   &= \lim_{\eta \to 0^+} \eta \int\limits_0^\infty \mathrm{d}t' \, e^{-\eta t} f(t')
%\end{align}
%and the analog for $t \rightarrow -\infty$ with an integral from $-\infty$ to $0$
%a changed sign in the exponential function.
%Evaluating the asymptotic limit for $t\rightarrow +\infty$ yields
%
%\begin{align}
% \lim_{t \to +\infty} e^{-i(E_m^{(0)}-H)t} \ket{E_m^{(0)}}
% &= \lim_{t \to +\infty} \eta \int\limits_0^\infty \mathrm{d}t' \,
%    e^{-i(E_m^{(0)} -H -i\eta)t'} \ket{E_m^{(0)}}\\
% &= \lim_{t \to +\infty} \frac{-\eta}{-i(E_m^{(0)}-H -i\eta)} \ket{E_m^{(0)}}\\
% &= \frac{-i0^+}{E_m^{(0)}-H -i0^+}  \ket{E_m^{(0)}}
%\end{align}
%
%and hence general for both asymptotes
%\begin{equation} \label{equation:asymptot_wf_scattering}
% \ket{E_n^{(\pm)}} = \lim_{t \to \mp\infty} e^{-i(E_n^{(0)}-H)t} \ket{E_n^{(0)}}
%   = \frac{\pm i0^+}{E_n^{(0)}-H -i0^+} \ket{E_n^{(0)}}
%\end{equation}
%
%These asymptotic wavefunctions are eigenfunctions of the Hamiltonian including
%the interaction operator $V$ and has the same energies as $H_0$
%
%\begin{equation}
% H \ket{E_n^{(\pm)}} = E_n^{(0)} \ket{E_n^{(\pm)}}
%\end{equation}
%The are normalized according to
%\begin{equation}
% \braket{E_m^{(\pm)} | E_n^{(\pm)}} = \delta_{mn}
%\end{equation}
%
%Cleverly adding zeros to equation (\ref{equation:asymptot_wf_scattering})
%yields
%\begin{align}
% \ket{E_n^{(\pm)}} &= \frac{1}{E_n^{(0)}-H \pm i0^+}
%   \left[ \left( E_n^{(0)}-H \pm i0^+ \right) + \left( H_0 + V - E_n^{(0)} \right)
%        \right] \ket{E_n^{(0)}}   \\
% &= \left[ \mathbf{1} + G_n^{(\pm)}V \right]  \ket{E_n^{(0)}}
%\end{align}
%where the Green's operator $G_n^{(\pm)} = \frac{1}{E_n^{(0)}-H \pm i0^+}$ is
%introduced. Upon these latter equations an alternative formulation
%of the Lippmann-Schwinger equation is achieved, which connects the eigenfunctions
%of $H_0$ to the asymptotic wavefunctions
%
%\begin{equation} \label{equation:LippmannSchwinger}
% \ket{E_n^{(\pm)}} =  \ket{E_n^{(0)}} + G_n^{(\pm)}V \ket{E_n^{(0)}}
%\end{equation}
%
%Therefore, the scattering matrix can now be expressed in terms of the asymptotic
%wavefunctions as
%
%\begin{equation}
% S_{mn} = \braket{E_n^{(-)} | E_n^{(+)}}
%\end{equation}
%
%Since the processes of interest are characterized by an escaping electron,
%which needs to be described with an asymptotic wavefunction, while the initial
%state is quasi-bound, the following discussion will focus on the time-evolution
%description of the final state.
%\begin{equation}
% \bra{E_m^{(\pm)}} = \bra{E_m^{(0)}} + \bra{E_m^{(0)}} V G_n^{(\mp)}
%\end{equation}
%
%Using the definition in equation (\ref{equation:LippmannSchwinger})
%for the asymptotic limits for both $t\rightarrow \infty$ and
%$t' \rightarrow - \infty$ and and subtracting them leads to 
%
%\begin{align}
%  \bra{E_m^{(-)}} - \bra{E_m^{(+)}} &= \bra{E_m^{(0)}} V (G_n^{(+)} - G_n^{(-)} ) \\
%  S_{mn} &= \delta_{mn} - \braket{E_m^{(0)}| V (G_n^{(+)} - G_n^{(-)}) |E_n^{(+)}} 
%\end{align}
%
%Evaluating the action of the Green's function on $\ket{E_n^{(+)}}$ yields
%\begin{equation}
% (G_n^{(+)} - G_n^{(-)}) \ket{E_n^{(+)}} 
% = \frac{-2i0^+}{\left( E_m^{(0)} - E_n^{(0)} \right)^2 + (0^+)^2} \ket{E_n^{(+)}}
%\end{equation}
%
%
%The achieved prefactor has the form of a function, which approximates the delta
%function with a Lorentzian shape
%\begin{equation}
% \delta(x) = \lim_{\eta \to 0^+} \frac{1}{\pi} \frac{\eta}{x^2 + \eta^2}
%\end{equation}
%
%and therefore, the scattering matrix is given by
%\begin{equation}
% S_{mn} = \delta_{mn} - 2\pi i \delta \left(E_m^{(0)} - E_n^{(0)} \right)
%          \braket{E_m^{(0)}| V |E_n^{(+)}}
%\end{equation}
%It consists of two terms. The first describes the the possibility that no scattering
%occurs and hence both initial and final state are preserved. The second term
%describes the scattering process in terms of the asymptotic limits. However,
%the desired expression should describe the scattering process in terms of the
%eigenstates of $H_0$. This is achieved by defining the transition operator
%
%\begin{equation}
% T^{(n)} \equiv V \left( \mathbf{1} + G_n^{(+)} V \right)
%\end{equation}
%
%Its action on $\ket{E_n^{(0)}}$ equals the action of $V$ on the asymptotic
%states $\ket{E_n^{(-)}}$
%\begin{equation}
% V \ket{E_n^{(-)}} = T^{(n)} \ket{E_n^{(0)}}
%\end{equation}
%
%Hence, the scattering matrix reads
%\begin{equation}
% S_{mn} = \delta_{mn} - 2\pi i \delta \left(E_m^{(0)} - E_n^{(0)} \right)
%          \braket{E_m^{(0)}| T^{(n)} |E_n^{(0)}}
%\end{equation}
%
%The corresponding $\mathcal{T}$ matrix contains all information about the
%scattering process. Its poles correspond to the resonance energies and
%the complex part of these resonance energies correspond to half the decay width.
%
%Its absolute square corresponds to the transition probability density
%which shows maxima at the resonance positions. They have a Lorentzian shape
%and the \ac{FWHM} is the decay width $\Gamma=\frac{\hbar}{\tau}$. The Lorentz
%distribution is better as known as Cauchy distribution and its shape for some
%function $f$ is given to illustrate the decay width $\Gamma$ in Figure
%\ref{figure:general_resonance}.
%
%
%\begin{figure}[h]
%  \centering
%   \begin{tikzpicture}[
          scale=0.5,>=stealth,domain=0.5:10,samples=100,
          declare function={
          gamma = 1.0;
          factor = 16.0;
          halfmax = factor * 0.15915;
          x_0 = 5.0;
          distrib(\x) = factor/3.14159 * gamma / ((\x-x_0)^2 + gamma^2);
        }]
%     \tiny
%  \draw[very thin,color=gray] (-0.1,-0.1) grid (4.9,4.9);
  \draw[->,thick] (-0.2,0) -- (10.2,0) node[right] {$E$};
  \draw[->,thick] (0,-0.2) -- (0,6.2) node[anchor=north east] {$\left. \frac{\mathrm{d}}{\mathrm{d}t}
                                                   |f_{fi}|^2 \right|_{t=0}$};
  % add ticks
  \draw [thick] (5,0) -- (5,-5pt) node [anchor=north] {$E_f$};

  \draw [color=black,domain=0:10,smooth,very thick]    plot
         (\x,{distrib(\x)});% node [anchor=south] {Cauchy distribution};
  \draw [-,very thick,diplom1] (4.0,halfmax) -- (6,halfmax)
         node [anchor=south west] {$\Gamma$};
 \end{tikzpicture}

%  \caption{Probability density function of a Cauchy distribution with a
%           maximum at $x_0$ with a height of $A=\frac{1}{\pi\gamma}$.}
%  \label{figure:general_resonance}
%\end{figure}
%
%Now, the task is to find the matrix elements of $\mathcal{T}$ given by
%\begin{equation}
% \mathcal{T} = \braket{E_m^{(-)}| V | E_n^{(0)}} = \braket{E_m^{(0)}| V |E_n^{(+)}}
%\end{equation}
%or their complex conjugates carrying the same information \cite{Feshbach_book}.



\section{Projection Method for the Evaluation of the $\mathcal{T}$ Matrix}

One of the approaches to address the treatment of multi-state systems is the
partitioning of the Hilbert space into initial and final
state subspaces and using the appropriate eigenfunctions of these subspaces for the
calculation of the decay widths. 
Several theories had been used for the description of different nuclear reactions
before they were first unified by Feshbach \cite{Feshbach58,Feshbach62,Feshbach_book}.
Somewhat later,
Fano extended this ansatz to describe excitation spectra, which
are inverse processes to Feshbach's nuclear reactions.\cite{Fano61}


In the following, Feshbach's formulation using projection
operators is applied for the case of a meta-stable decaying state,
where the initial state
is bound-like and the final states are continuum states.
Starting from the Schroedinger equation of the total system under investigation

\begin{equation}
  H \Psi = E \Psi \label{schroedinger}
\end{equation}

the projection operators $P$ and $Q$ are defined. $P$ projects the final states
or the so-called open channels out
of the total wavefunction $\Psi$ and is defined with respect to eigenstates
of the system without interaction. Hence, the initial and final state subspaces
are orthogonal ($PQ=0$).
$Q$ is analogously defined with respect to the rest of the
system as $Q = 1 - P$. Therefore after insertion into eq. (\ref{schroedinger})
\begin{equation}
  H (P+Q) \Psi = E \Psi
\end{equation}
\begin{align}
  (E - H_{PP}) P \Psi & = H_{PQ} Q \Psi \label{se_PP}\\
  (E - H_{QQ}) Q \Psi & = H_{QP} P \Psi \label{se_QQ}
\end{align}

can easily be derived with
\begin{align*}
  H_{PP} & \equiv PHP & \quad\quad H_{PQ} & \equiv PHQ\\
  H_{QP} & \equiv QHP & \quad\quad H_{QQ} & \equiv QHQ .
\end{align*}

From equation (\ref{se_QQ}) a straigthforward solution for the system excluding
the selected final states can be found.

\begin{equation}
  Q \Psi = \frac{1}{E-H_{QQ}} H_{QP} P \Psi \label{feshbach_qpsi}
\end{equation}

The latter expression holds in case of all open channels being included
in the final state description.
In case of selectively chosen open channels, which do not resemble the total
space of open channels,
$E$ is to be substituted by $E^{(-)}=E - i\eta$.

After insertion of eq. (\ref{feshbach_qpsi}) into eq. (\ref{se_PP}) one arrives at

\begin{equation}
  \mathscr{H} \,P \Psi = E \,P \Psi \label{se_ppsi}
\end{equation}

with $\mathscr{H}$ being the effective Hamiltonian of the final states.
\begin{equation}
  \mathscr{H} = H_{PP} + H_{PQ} \frac{1}{E-H_{QQ}} H_{QP}
\end{equation}

In order to solve these expressions we define $\{\Phi_n\}$ to be the solutions
of the Hamiltonian excluding the final state solutions (or the closed channels
solutions in case of all open channels being defined as final states).
These initial state functions are assumed to be bound and to fulfill the
Schrödinger equation

\begin{equation}
  (\varepsilon_n - H_{QQ}) \ket{\Phi_n} = 0 
\end{equation}

It has to be noted that these functions and energies are solutions to a stationary
problem and therefore, expressing the meta-stable, but long-lived state
in terms of these functions is an approximation. However, for systems
with long lifetimes, this approximation is reasonable.

Together with a set of continuum wavefunctions $\{\Phi(\alpha,E)\}$, they are
defined to fulfill the following orthogonality relations

\begin{align}
  \braket{\Phi_n|\Phi_n} = 1 \quad  & \quad \braket{\Phi_n|\Phi(\alpha,E)} = 0\\
  \braket{\Phi(\alpha,E)|\Phi(\alpha',E')} & = \delta(\alpha-\alpha') \delta(E-E')
\end{align}

and thereby to form an orthonormal basis. These continuum wavefunctions
are characterized
by their energy $E$ and their quantum numbers, which at this stage are both
contained
in the index $\alpha$.

\begin{equation}
  1 = \sum\limits_n \ket{\Phi_n}\bra{\Phi_n} + \int \mathrm{d}\alpha \int \mathrm{d}E
      \ket{\Phi(\alpha,E)}\bra{\Phi(\alpha,E)} \label{feshbach_1}
\end{equation}

Expanding eq. (\ref{se_ppsi}) into this complete set yields an effective
final state Hamiltonian

\begin{equation}
  \mathscr{H} = H_{PP}\, + \,
  \sum\limits_n H_{PQ} \,\frac{\ket{\Phi_n}\bra{\Phi_n}}{E-\varepsilon_n}\, H_{QP} \,+\,
  \int \mathrm{d}\alpha \int\mathrm{d}\varepsilon \,H_{PQ} \,
  \frac{\ket{\Phi(\alpha,\varepsilon)}\bra{\Phi(\alpha,\varepsilon)}}{E-\varepsilon} \, H_{QP}
\end{equation}

which is useful to split into two parts: One describing the interaction
$\mathcal{V}$ with the
initial state $\Phi_s$ with its energy $\varepsilon_s$ being in resonance with
the continuum and the rest $\mathscr{H}'$

\begin{equation}
  \mathscr{H} = \mathscr{H}' +
                H_{PQ} \,\frac{\ket{\Phi_s}\bra{\Phi_s}}{E-\varepsilon_s}\, H_{QP}
              = \mathscr{H}' + \mathcal{V}
\end{equation}

with
\begin{equation}
  \mathscr{H}' = H_{PP}\, + \,
  \sum\limits_{n\ne s} H_{PQ} \,\frac{\ket{\Phi_n}\bra{\Phi_n}}{E-\varepsilon_n}
  \, H_{QP} \,+\,
  \int \mathrm{d}\alpha \int\mathrm{d}\varepsilon \,H_{PQ} \,
  \frac{\ket{\Phi(\alpha,\varepsilon)}\bra{\Phi(\alpha,\varepsilon)}}{E-\varepsilon} \, H_{QP}
\end{equation}

This reformulation leads to the following version of eq. (\ref{se_ppsi})
\begin{equation}
  (E - \mathscr{H}')\, P \Psi =
   H_{PQ} \,\frac{\ket{\Phi_s}\bra{\Phi_s}}{E-\varepsilon_s}\, H_{QP} P \Psi %= \mathscr{V} P \Psi.
\end{equation}

$P \Psi$ has to be described in terms of the final states in the asymptotic
region, which means that the wavefunction has to satisfy the approriate
boundary conditions of the escaping particle. Therefore,
the eigenfunctions of $\mathscr{H}'$ have to fulfill incoming boundary conditions,
which is labelled by the superscript $(-)$

\begin{equation}
  (E-\mathscr{H}') \ket{\psi_f^{(-)}} = 0 \label{sol_outg}
\end{equation}

This relation can then be utilized to find a solution for $\ket{P \Psi}$
using the Lippmann-Schwinger equation (\ref{equation:LippmannSchwinger}):

\begin{equation}\label{sol_ppsi}
  \ket{P \Psi} = \ket{\psi_f^{(-)}} + \frac{1}{E^{(-)} - \mathscr{H}'}
           \frac{H_{PQ}\ket{\Phi_s}
           \braket{\Phi_s|H_{QP}|P\Psi}}{E - \varepsilon_s} 
\end{equation}

In order to solve this expression for $\ket{P \Psi}$,
eq. (\ref{sol_ppsi})
is multiplied from the left with $\bra{\Phi_s|H_{QP}}$ to give:
\begin{equation}
  \braket{\Phi_s|H_{QP}|P\Psi} = \braket{\Phi_s|H_{QP}|\Psi_f^{(-)}} -
  \frac{1}{E^{(-)} - \mathscr{H}'}
  \frac{\braket{\Phi_s|H_{QP}H_{PQ}|\Phi_s} \braket{\Phi_s|H_{QP}|P\Psi}}
       {E - \varepsilon_s}  \label{s_ppsi} 
\end {equation}

Defining the quantity

\begin{equation}
  W_{QQ} = H_{QP}\frac{1}{E^{(-)} - \mathscr{H}'}H_{PQ}
\end{equation}

and solving eq. (\ref{s_ppsi}) for

\begin{equation}
  \braket{\Phi_s|H_{QP}|P\Psi} = \frac{\braket{\Phi_s|H_{QP}|\Psi_f^{(-)}}(E-\varepsilon_s)}
{E - \varepsilon_s + \braket{\Phi_s|W_{QQ}|\Phi_s}}
\end{equation}

yields the final state description
\begin{equation}\label{}
  \ket{P \Psi} = \ket{\psi_f^{(-)}} - \frac{1}{E^{(-)} - \mathscr{H}'}
           \frac{H_{PQ}\ket{\Phi_s}
           \braket{\Phi_s|H_{QP}|\psi_f^{(-)}}}
           {E - \varepsilon_s + \braket{\Phi_s|W_{QQ}|\Phi_s}} .
\end{equation}



The complex conjugate of the above mentioned matrix of transition
amplitudes $\mathcal{T}$
is therefore given by


\begin{align}
  \mathscr{T}_{if}^* &= \braket{\psi_i^{(+)}| \mathcal{V} | P\Psi} \\
                   &= \mathscr{T}_{if}^{(P)*} + 
                     \frac{\braket{\psi_i^{(+)}|H_{PQ}|\Phi_s}
                           \braket{\Phi_s|H_{QP}|\Psi_f^{(-)}}}
                          {E-\varepsilon_s + \braket{\Phi_s|W_{QQ}|\Phi_s}}
\end{align}

where $\ket{\psi_i^{(+)}}$ in case of a full-collision process describes
the solution of the Schrödinger with outgoing
boundary conditions

\begin{equation}                                                              
  (E - \mathscr{H}') \ket{\psi_i^{(+)}} = 0
\end{equation}

Here, $\mathscr{T}_{if}^{(P)*}$ describes the non-resonant transitions
between the initial and final states. Therefore, close to   
the resonance, this part is expected to be small compared to the second part, which
describes the transition
from the initial into the final states mediated by the interaction between them.

From the transition amplitudes information about
the resonance energy can be obtained from its poles. Therefore,
$\braket{\Phi_s|W_{QQ}|\Phi_s}$ needs to be examined further.
We split it into
its real and imaginary part by introducing a delta function using 
$\lim\limits_{\varepsilon \to 0_+} \frac{1}{x \pm i\varepsilon}
 = \mathscr{P} \frac 1x \mp i\pi\delta(x)$,
where $\mathscr{P}$ is the principal part and $\delta(x)$ denotes the
Dirac delta function. \cite{Cohen-Tannoudji_3_2}

\begin{align}
  \braket{\Phi_s|W_{QQ}|\Phi_s} %& = \Delta_s(E) - i \frac{\Gamma_s(E)}{2}\\
                                & = \braket{\Phi_s|H_{QP}
                                    \frac{1}{E^{(-)} - \mathscr{H}'}H_{PQ}|\Phi_s}\\
                                & = \braket{\Phi_s|H_{QP}
                                    \frac{\mathcal{P}}{E - \mathscr{H}'}H_{PQ}|\Phi_s}
                                    - i\pi \braket{\Phi_s|H_{QP}\delta(E-\mathscr{H}')H_{PQ}
                                    |\Phi_s} \\
                                & = \Delta_s(E) - i \frac{\Gamma_s(E)}{2}
\end{align}

It consists of a real part, which can be related to an energetic shift
of the resonance energy caused by the interaction, and an imaginary part,
which is related to decay width $\Gamma$.
Inserting the latter expression into the transition matrix yields

\begin{equation}
  \mathscr{T}_{if}^* = \mathscr{T}_{if}^{(P)*} + 
                     \frac{\braket{\psi_i^{(+)}|H_{PQ}|\Phi_s}
                           \braket{\Phi_s|H_{QP}|\Psi_f^{(-)}}}
                          {E-\varepsilon_s + \Delta_s - i \frac{\Gamma}{2}}
\end{equation}

%from which now the real part of $\braket{\Phi_s|W_{QQ}|\Phi_s}$ can be interpreted 
%as an energy shift $\Delta_s(E)$ of the resonance introduced by the interaction of the initial
%with the final states and its imaginary part to carry the information about the
%decay width $\Gamma_s(E)$ and hence the lifetime.


From the imaginary part of $\braket{\Phi_s|W_{QQ}|\Phi_s}$ the decay width can
easily be deduced. Inserting a complete set
of eigenfunctions with incoming boundary conditions as defined in
eq. (\ref{sol_outg}) the decay width can be described
as a sum over the different open channel solutions for a given resonant
initial state:
\begin{align}
  \Gamma_s & = 2 \pi \braket{\Phi_s|H_{QP}\delta(E - \mathscr{H}')H_{PQ}|\Phi_s}
               \label{equation:Gamma_HE}\\
           & = 2 \pi \sum\limits_r \left | \braket{\Phi_s|H_{QP}|\psi_r^{(-)}} \right|^2
                 \label{equation:Gamma_HQP}\\
           & = 2 \pi \sum\limits_r \Gamma_{sr}(E)
\end{align}
Here it has to be remembered that $E$ is the energy of a final state, which means that
only such functions give a contribution, which have the same energy as those.

This formulation is sufficient for all cases, which are going to be discussed
in this thesis.
However, {\AA}berg \textit{et al.} have proposed a slightly different ansatz with
a different partitioning, where configurations can contribute both to the description
of the initial and final state subspace. The expression for the decay width $\Gamma$
is very similar to the above approach in equation (\ref{equation:Gamma_HE}), with more
complex conditions in the delta function \cite{Aaberg82}.
This approach is not valid for exact
eigenfunctions, but may be useful in case of mappings to $\mathcal{L}^2$ functions as
e.g., \ac{CI} used by Fano \cite{Fano61}. The resulting expression reads

\begin{equation}
  \Gamma_s = 2 \pi \left| \braket{\Phi_s|H-E|\psi_r^{(-)}} \right| ^2
\end{equation}
and is called \emph{Fano's Golden Rule}. In the following, the latter will be
the starting point
but a partitioning into orthogonal initial and final
states is always chosen, which means that
equation (\ref{equation:Gamma_HQP}) applies.
