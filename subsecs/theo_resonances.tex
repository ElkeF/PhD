\chapter{Resonances}

Classically, a resonance is the maximum response of a system to the vibrations
of at least one other
system, in which energy is either transferred from 
one system to another or converted into a different kind of energy.
In quantum mechanics analogously the maximum response 
of two states (not necessarily eigenstates of the system) 
bound or continuum states is referred to as a resonance.
One has to distinguish between two major manifolds of resonances
by being reversible or irreversible \cite{Cohen-Tannoudji_3_2}
as illustrated in figure
\ref{figure:overview_resonances}.
In reversible resonances
interaction of the enganged states can be mediated via an external field,
which causes excitations
into different electronic, vibrational and rotational states.
Irreversible resonances on the other hand are observed in the case of
metastable states decaying over time. We are going to focus on the theory of
decaying metastable states describing electronic decay processes like the
Auger process and the zoo of \ac{ICD} like processes. 

Metastable states can be divided into shape resonances and Feshbach resonances,
also called Fano resonances. In shape resonances the particle of investigation
resides in
a local minimum of the potential and decays via tunneling through some barrier.
\cite{Klaiman12}
In other words, it escapes from its former residence.
In contrast to the shape resonances, in Feshbach-Fano resonances
a bound state decays via coupling to continuum states. Such a situation might
appear in scattering experiments, where we consider two particles $A$ and $B$.
For simplicity we choose particle $B$ to be much more heavy than particle $A$ and
hence to a good approximation to be fixed in space and particle $A$
approaching it. At very large distances the two particles do not interact with
each other. Each of them is in an eigenstate of both its own local system
and the total system containing both particles. As $A$ approaches $B$, the two
particles start feeling and influencing each other which consequently leads
to changes in the eigenstates of the system and hence neither of the particles
inhabiting one of them. A metastable state is formed. Inside this
\emph{interaction region} the system
rearranges electronically and decays in our case of interest. The decay
products leave the interaction region and can subsequently be analyzed.
Likewise the decay of metastable states can be described in processes
where the reaction partners do not approach each other but the metastable
state is created inside the interaction region. This means that the initial
state is never unperturbed by the other particle. Here it is assumed, that
the meta-stable state is quasi-bound and the decay rate is determined by
the interaction of this initial state in a bound state description with
states in the continuum.

In the following sections we will start with 
discussing properties of metastable states.
We are going to get insight through the study of electron dynamics inside
the interaction region
and the description of their decay into several decay channels from an outside
view, where what happens in detail inside the interaction region is of no
concern and solely the initial and final states are taken into
account. \cite{Gell-Mann53}


\begin{figure}[h]
  \centering
  %
% Tikz tree
%
\tikzset{font=\small,
edge from parent fork down,                                         
level distance=1.75cm,                                              
every node/.style=                                                  
    {top color=white,                                               
    bottom color=diplom2!25,                                        
    rectangle,rounded corners,                                      
    minimum height=8mm,                                             
    draw=diplom2!75,
    very thick,                                                     
    drop shadow,                                                    
    align=center,                                                   
    text depth = 0pt                                                
    },                                                              
edge from parent/.style=                                            
    {draw=diplom2!50,                                                             
    thick                                                                         
    }}      

\begin{tikzpicture}[scale=1.0]

\Tree [.Resonances
        [.{reversible\\ processes}
          [.{{interaction between}\\{bound states}}
            [.{{electronic, vibrational,}\\{rotational excitations}}
%              [.{{Fermi's}\\{Golden Rule}}
%              ]
            ]
          ]
         ] 
        [.{irreversible\\processes}
           [.{{interaction of}\\{continuum states}} 
             [.{{full-collision}\\{processes}}
             ]
           ]
           [.{{interaction of quasi-bound}\\{with continuum states}} 
             [.{{half-collision}\\{processes}}
               [.{shape type}
                ]
               [.Feshbach-Fano
%                 [.{Fano's\\{Golden Rule}}
%                 ]
                ]
           ]
           ]
         ]
]

\end{tikzpicture}

  \caption{Schematic overview about different kinds of resonances and the
           calculation of their respective lifetimes.}
  \label{figure:overview_resonances}
\end{figure}





\section{Properties of Decaying Metastable States}
In this section we strictly follow the argumentation of reference
\cite{Klaiman12}, where a more detailed discussion can be found.
Consider a particle residing in a bound state of some potential, which
can be described as an $\mathcal{L}^2$ normalized eigenstate $\psi_0$
of the corresponding time-independent
Schrödinger equation. Its time evolution can be described via
the solution of the time-dependent Schrödinger equation
\begin{equation}
  \psi(x,t) = e^{-iE_0t} \psi_0(x) .
\end{equation}

Our aim is to describe the decay of a metastable state, for which we employ
a more general ansatz for the wavefunction
\begin{equation}  \label{equation:td_ansatz}
  \psi(x,t) = \Lambda(x,t) e^{iS(x,t)}   ,
\end{equation}

where $\Lambda(x,t)$ denotes the amplitude of the wavefunction
depending on the potential under investigation and
its phase is described by $S(x,t)$. It can be shown, that inside the
interaction region the phase function
decreases exponentially in time and outside the interaction region,
the wavefunction depletes over space, or in other words the
describing wavepacket is expected to travel with a constant velocity and 
momentum $k_r$ away
from the interaction region, which we here depict to be limited to the
interval $[-L,L]$. At this point we do not specify $L$ further.
This leads to the following ansatz of the
phase function

\begin{equation}    \label{equation:phase_function}
  S(|x|,t>t_0) = \begin{cases}
                   -\varepsilon t + \phi         & |x| < L \\
                   k_r |x| -\varepsilon t + \phi & |x| > L   ,
                 \end{cases}
\end{equation}
where $\phi$ is some arbitrary phase.


With this knowledge about the wavefunction we now investigate the decay
of the metastable state. We therefore investigate the probability of a particle
to be inside the interaction region $[-L,L]$ at different times $t$. This
probability is described by the integral over the density inside the interaction
region or the corresponding norm

\begin{equation}
  N_L(t) = \int\limits_{-L}^L |\psi(x,t)|^2 \mathrm d x .
\end{equation}

Calculations for the decay of a metastable state described by a superposition
of continuum eigenstates in some model potential
show the behaviour of the norm in figure \ref{figure:decay_norm}.

\begin{figure}[h]
  \centering
  \caption{}
  \label{figure:decay_norm}
\end{figure}

At the beginning of the decay the norm is mostly influenced by fast
continuum states and after a short time $t_0$ the characteristic
exponential decay of the metastable state can be observed. From the slope
of the curve, we arrive at an ansatz for the time dependent norm with
the lifetime $\tau$ and $N_L(t_0)$ being the norm at the starting time
$t_0$ of the characteristic resonance behaviour.

\begin{equation}
  N_L(t>t_0) = e^{-\frac t\tau} N_L(t_0)
\end{equation}

We can now also investigate properties of the particle inside the
interaction region, like e.g. the local energy expectation value

\begin{equation}
  \braket{\hat{H}}_L = \frac{\int\limits_{-L}^L \psi^*(x,t) \hat{H} \psi(x,t) \mathrm dx}
                       {\int\limits_{-L}^L |\psi(x,t)|^2 \mathrm dx}
\end{equation}

where we are interested in the time-dependent Hamiltonian
given by the time-dependent Schrödinger equation
\begin{equation}
  \hat{H}\psi(x,t) = i \frac{\partial}{\partial t} \psi(x,t) .
\end{equation}

From the latter we can create a set of two equations. For the first
the time-dependent Schrödinger equation is multiplied by $\psi^*$ from the
left and the second equation is the complex conjugate of the first equation.
Subtracting the second equation from the first we arrive at
\begin{equation}
  \psi^*(x,t) \hat{H} \psi(x,t) - \psi(x,t) \hat{H} \psi^*(x,t) = 
      i \frac{\partial}{\partial t} |\psi(x,t)|^2    .
\end{equation}

After integration over the interaction region we can connect the latter equation
to the time derivative of the probability of the particle to be inside the
interaction region
\begin{equation}
  \frac{\partial}{\partial t} N_L(t) = 2 \operatorname{Im} \left(
       \int\limits_{-L}^L \psi^*(x,t) \hat{H} \psi(x,t) \right)
\end{equation}

from which we by comparison have access to the resonance lifetime of the system
\begin{equation}
  \Im \left( \braket{\hat{H}}_L \right) = -\frac{1}{2\tau}   .
\end{equation}

Instead of subtracting the two equations stemming from the time-dependent
Schrödinger equation adding them yields

\begin{equation}
  \operatorname{Re} \left( \int\limits_{-L}^L \psi^*(x,t) \hat{H} \psi(x,t) \right)
 = -\operatorname{Im} \left( \int\limits_{-L}^L \psi^*(x,t)
    \frac{\partial \psi(x,t)}{\partial t} \right)   ,
\end{equation}
which with the help of the ansatz of the wavefunction \ref{equation:td_ansatz}
and the behaviour of the phase function \ref{equation:phase_function}
connects the real part of the energy expectation value inside the interaction
region with the energy of the resonance state $\varepsilon$.

\begin{equation}
  \operatorname{Re} \left( \braket{\hat{H}}_L \right)
  = -\frac{\partial S(x,t)}{\partial t}
  = \varepsilon
\end{equation}

Hence the energy of the resonance state is complex, where the real part
describes the energy of the resonance state and the imaginary part gives rise
to the lifetime of the system.
\begin{equation} \label{equation:expectation_H_interaction_region}
  \braket{\hat{H}}_L = \operatorname{Re} \left( \braket{\hat{H}}_L \right)
  + i \operatorname{Im} \left( \braket{\hat{H}}_L \right) = \varepsilon - \frac{i}{2\tau}
\end{equation}

To conclude from these general investigations, the wavefunction inside
the interaction region is bound-like while outside the interaction region,
we see particles leaving the interaction region. The probability of a
particle to be inside the interaction region decays exponentially in time
and the local energy expectation value inside the interaction region is complex,
where the imaginary part is connected to the lifetime and hence the decay
rate of the system.





\section{Decay Widths from a Scattering Point of View}
For a very general description the solution of the time-dependent Schrödinger
equation inside the interaction region is inpractical. First of all, the
choice of the size of the interaction region will crucially influence the
final result. It is therefore beneficial to change from solving the time-dependent
Schrödinger equation to solving the time-independent Schrödinger equation
with proper boundary conditions and following the time evolution of the population
of these time-independent states. In the case of resonances
this will lead to so-called resonance states
with a finite lifetime, a complex energy and the correct asymptotic behaviour.

In a full-collision scattering experiment of two particles we choose
one particle to be fixed in space and the other one to be scattered on it.
This is a good approximation if the scattered particle's mass is much smaller
than the target particle. The time of interaction is denoted by $t=0$.
At times $t<<0$ the moving particle can be well approximated to behave like
a free particle and hence a plane wave. As it approaches the target, it is
perturbed and interacts with the target. Afterwards the particle leaves the target
but might be inherently changed due to the interaction with the target particle.
For large distances and times $t>>0$ it approaches the behaviour of a free
particle again.

The wavefunctions of the initial and final states $\ket{\phi}$
and $\ket{\chi}$
at the time of the interaction
$t=0$ are obtained from their asymptotes $\ket{\phi ^{(+)}}$ 
and $\ket{\chi ^{(-)}}$ as

\begin{align}
  \ket{\phi}  &\rightarrow  \ket{\phi ^{(+)}}\\
  \ket{\chi}  &\rightarrow  \ket{\chi ^{(-)}}
\end{align}
where $+$ denotes outgoing and $-$ denotes incoming boundary conditions.
This means, that the approaching particle has outgoing boundary conditions and the
leaving particle has incoming boundary conditions.

The probability of finding the system with the initial state $\ket{\phi}$ in
the final state $\ket{\chi}$ due to the interaction at times $t\approx 0$ is given
by the overlap of the scattering states at $t=0$

\begin{align}
  w(\chi \leftarrow \phi) &= | \braket{\chi^{(-)}|\phi^{(+)}} |^2   \\
                          &= | \braket{\chi|S|\phi} | ^2    ,
\end{align}
where $S$ denotes the scattering operator, which takes care of the preparation
of the initial and final states from their asymptotes.

In the following, we will show how this probability is related to the transition
rate and the decay width for the example of an incoming particle, which is scattered
at the target particle. Our process of interest is inverse to this problem set, but
for illustrating purposes, it is easier to follow the time in positive direction.

For the calculation of decay rates we start
from the time dependent Schroedinger equation

\begin{equation}
  i \frac{\mathrm{d}}{\mathrm{d}t} \Psi(t) = (K + V) \Psi(t) ,
\end{equation}
where $K$ denotes the Hamilton operator of non-interacting colliding
particles, or in our case initial and final states. Its solution
$\Phi_i(t) = \phi_i e^{-iE_it}$ are stationary states of the system.

\begin{equation}
  i \frac{\mathrm{d}}{\mathrm{d}t} \Phi(t) = K \Phi(t)
\end{equation}

Our purpose is the description of the transition rates from an initial state
$\Phi_i(t)$ to a final eigenstate $\Phi_f(t)$ mediated by the interaction $V$ between
them. We are going to achieve this by taking the time derivative of the system's
probability $\omega_{fi}(t)$ to be in a certain final state at time t.

\begin{align}
  w_{fi}(t) &= \frac 1{N_i} |f_{fi}|^2 \\
  f_{fi}(t) &= \braket{\Phi_f(t)|\Psi_i(t)} \label{equation:scattering_overlap}\\
  N_i       &= \braket{\Psi_i(t)|\Psi_i(t)}
\end{align}

In eq. (\ref{equation:scattering_overlap}) the solution of the time-dependent 
Schrödinger equation $\Psi_i$ and not the stationary eigenstate $\Phi_i$
represents the initial state. Since our knowledge
about the initial state wave function is limited to its behaviour without the
interaction $V$, being turned on at $t=0$, we have to describe the initial state
wave function at some time in the distant past $T<0$ and propagate it until $t=0$.
Therefore the question arises, which time $T$ should be selected for this purpose.
Since no time is better than any other and the result might depend on the decision,
one averages over propagations starting at different times $T$.

\begin{equation}
  \Psi_i(t) = \frac 1\tau \int\limits_{-\tau}^0 \mathrm{d}T \,
              e^{-iH(t-T)} \, \Phi_i(T)
\end{equation}
Here $\tau$ is allowed to approach $+\infty$ at the end of the calculation.

A more convenient way to include this ansatz in the further derivation
is its Fourier transformation
\begin{equation}
  \Psi_i(t) = \varepsilon \int\limits_{-\infty}^0 \mathrm{d}T \,
              e^{\varepsilon T} e^{-iH(t-T)} \Phi_i(T)
\end{equation}

with $\varepsilon = \tau^{-1}$. Evaluating the integral, this leads to:
\begin{align}
   \Psi_i(t) &= \varepsilon \, e^{-iHt} \int\limits_{-\infty}^0 \mathrm{d}T \,
                e^{\varepsilon T} e^{i(H-E_i)T} \phi_i\\
             &= e^{-iHt} \frac{\varepsilon}{\varepsilon+i(H-E_j)} \phi_i .
\end{align}

Using the Schroedinger equation
\begin{equation}
  (H-E_i) \phi_i = V \phi_i
\end{equation}

of the whole system, one easily arrives at an expression for the initial state
at time $t=0$.
\begin{align}
  \Psi_i(0) &= \frac{\varepsilon + i(H-E_i) - i(H-E_i)}{\varepsilon + i(H-E_i)} \phi_i\\
            &= \phi_i + \frac{1}{E_i-H+i\varepsilon} V \phi_i \label{equation:in_state_0}\\
            &\approx \phi_i + \frac{1}{E_i-K+i\varepsilon} V \Psi_i(0) \label{equation:in_state_0_approx}
\end{align}

The latter equation holds for a small perturbation $V$ as can be seen by comparing
the power expansions of equations (\ref{equation:in_state_0}) and
(\ref{equation:in_state_0_approx}).\\
Since the norm $N_{fi}$ is time-independent, we now know all variables
necessary for the determination of the decay rate.
Inserting equation (\ref{equation:in_state_0_approx}) into equation
(\ref{equation:scattering_overlap}) we evaluate the overlap between the initial
and the final state.
\begin{equation}
  f_{fi}(0) = \delta_{fi} + \frac{1}{E_i-E_f+i\varepsilon} R_{fi}(\varepsilon)
\end{equation}

Here, 
\begin{equation}
  R_{fi}(\varepsilon) = \braket{\phi_f|V|\Psi_i(0)}
\end{equation}
denotes the coupling of the perturbed initial state at $t=0$ with the
final state via the interaction
operator $V$. In case of the interaction $V$ being zero, the incoming particle
is scattered at the target particle without changing the formation of a metastable
state and both particles inherit the same states before and after the collision.
This is taken care of by the $\delta$ function. Already at this stage it is evident,
that the function has
poles in the case of the energies of the initial and final state being equal
and hence the response of the system of the interaction between the initial and final
state is maximized at this point.

The time dependence of coupling between the two time-independent states is now
introduced to give:
\begin{equation}
  f_{fi}(t) = \braket{\phi_f| e^{i(E_f-H)t} |\Psi_i(0)} .
\end{equation}

Its absolute square is proportional to the probability of the system to be
in the final state $\phi_f$ and its time derivative at $t=0$ yields the
transition rate
\begin{equation}
  \left . \frac{\mathrm{d}}{\mathrm{d}t} |f_{fi}|^2 \right |_{t=0}
  = 2\delta_{fi} \operatorname{Im}R_{ii}(\varepsilon) 
    + \frac{2\varepsilon}{(E_i-E_f)^2+\varepsilon^2} |R_{fi}(\varepsilon)|^2 .
\end{equation}

It consists of two parts, the first is propotional to the probability to stay
in the initial state and the second one describes the transition into the
final state. The latter has the typical Lorentzian shape with the full width
half maximum
(FWHM) $2 \varepsilon$ as illustrated in Figure \ref{figure:general_resonance}
(further information can be found in the appendix
\ref{section:app_cauchy}). When the energies of the initial and final state
are very similar, the second part dominates the decay rate. From now on
the full width half maximum ${2\varepsilon}$ will be called the
decay width $\Gamma$.

\begin{figure}[h]
  \centering
   \begin{tikzpicture}[
          scale=0.5,>=stealth,domain=0.5:10,samples=100,
          declare function={
          gamma = 1.0;
          factor = 16.0;
          halfmax = factor * 0.15915;
          x_0 = 5.0;
          distrib(\x) = factor/3.14159 * gamma / ((\x-x_0)^2 + gamma^2);
        }]
%     \tiny
%  \draw[very thin,color=gray] (-0.1,-0.1) grid (4.9,4.9);
  \draw[->,thick] (-0.2,0) -- (10.2,0) node[right] {$E$};
  \draw[->,thick] (0,-0.2) -- (0,6.2) node[anchor=north east] {$\left. \frac{\mathrm{d}}{\mathrm{d}t}
                                                   |f_{fi}|^2 \right|_{t=0}$};
  % add ticks
  \draw [thick] (5,0) -- (5,-5pt) node [anchor=north] {$E_f$};

  \draw [color=black,domain=0:10,smooth,very thick]    plot
         (\x,{distrib(\x)});% node [anchor=south] {Cauchy distribution};
  \draw [-,very thick,diplom1] (4.0,halfmax) -- (6,halfmax)
         node [anchor=south west] {$\Gamma$};
 \end{tikzpicture}

  \caption{Probability density function of a Cauchy distribution with a
           maximum at $x_0$ with a height of $A=\frac{1}{\pi\gamma}$.}
  \label{figure:general_resonance}
\end{figure}

It can be shown, that outgoing boundary conditions, where the initial state is
constructed from its asymptotic solution, are sufficient for the
description of such a process, which reduces $f$ to the so-called $\mathcal{T}$-matrix,
which is the matrix of transition amlitudes

\begin{equation}
  \mathcal{T} = \delta_{fi} + \frac{\braket{\phi_f|V|\Psi^{(+)}}}{E_i-E_f+i\varepsilon} .
\end{equation}

Its absolute square carries the same information
as $f$. It has poles at the resonance energies and the imaginary part of
the resonance energy is capable of the information about the decay width $\Gamma$.
For a more thorough definition see \cite{Taylor87} chapters 2,3 and 8.

In short terms, the decay width $\Gamma$ is defined as the FWHM of the
the time derivative of the probability distribution to find the system in
the final state at time $t=0$. Hence it is proportional to the reaction rate
constant, which entails its mathematical treatment. 
From comparison with equation \ref{equation:expectation_H_interaction_region}
the connection between the decay width $\Gamma$ and the lifetime $\tau$ becomes
evident to be
$\Gamma=\frac \hbar \tau$ in SI units.
This means that, the larger the width is, the faster is the transition
into the final state.


\section{Resonances in systems with more than two states}

In the case of a multistate system, the calculation of the decay width is more
complex than in a system with only two states.
This difficulty is overcome by partitioning the Hilbert space into initial and final
state subspaces and using the approriate eigenfunctions of these subspaces for the
calculation of the decay widths. 
Several theories had been used for the description of different nuclear reactions
before they were first unified by Feshbach in 1958 \cite{Feshbach58,Feshbach62,Feshbach_book}.
In contrast to earlier approaches, it holds for all coupling schemes as well as
all quantum numbers. They will be taken care of in the definition of the
projection operators.
Shortly after,
Fano amplified the latter ansatz to describe excitation spectra, which
are inverse processes to Feshbach's nuclear reactions.\cite{Fano61}


In the following, we are going to use Feshbach's formulation using projection
operators for the case of a meta-stable decaying state, where the initial state
is bound-like and the final states are continuum states.
Starting from the Schroedinger equation of the total system under investigation

\begin{equation}
  H \Psi = E \Psi \label{schroedinger}
\end{equation}

the projection operators $P$ and $Q$ are defined. $P$ projects the final states
or the so-called open channels out
of the total wavefunction $\Psi$ and is defined with respect to eigenstates
of the system in the asymptotic time limit, which means long after the process
itself finished. $Q$ is analogously defined with respect to the rest of the
system as $Q = 1 - P$. Therefore after insertion to eq. (\ref{schroedinger})
\begin{equation}
  H (P+Q) \Psi = E \Psi
\end{equation}
\begin{align}
  (E - H_{PP}) P \Psi & = H_{PQ} Q \Psi \label{se_PP}\\
  (E - H_{QQ}) Q \Psi & = H_{QP} P \Psi \label{se_QQ}
\end{align}

can easily be derived with
\begin{align*}
  H_{PP} & \equiv PHP & \quad\quad H_{PQ} & \equiv PHQ\\
  H_{QP} & \equiv QHP & \quad\quad H_{QQ} & \equiv QHQ .
\end{align*}

It has to be noted, that the applied criteria for the seletion of
these subspaces affect the final results. The above selection scheme is the
most common, but not the only possible partitioning.
From equation \ref{se_QQ} a straigth-forward solution for the system excluding
the selected final states can be found.

\begin{equation}
  Q \Psi = \frac{1}{E-H_{QQ}} H_{QP} P \Psi \label{feshbach_qpsi}
\end{equation}

The latter expression holds in case of all open channels being included
in the final state description.
In case of selectively chosen open channels, which do not resemble the total
space of open channels,
$E$ is to be substituted by $E^{(-)}=E - i\varepsilon$.

After insertion of eq. \ref{feshbach_qpsi} into eq. \ref{se_PP} one arrives at

\begin{equation}
  \mathscr{H} \,P \Psi = E \,P \Psi \label{se_ppsi}
\end{equation}

with $\mathscr{H}$ being the effective Hamiltonian of the final states.
\begin{equation}
  \mathscr{H} = H_{PP} + H_{PQ} \frac{1}{E-H_{QQ}} H_{QP}
\end{equation}

In order to solve these expressions we define $\{\Phi_n\}$ to be the solutions
of the Hamiltonian excluding the final state solutions (or the closed channels
solutions in case of all open channels being defined as final states).
These initial state functions are assumed to be bound and to fulfill the
Schroedinger equation

\begin{equation}
  (\varepsilon_n - H_{QQ}) \Phi_n = 0 .
\end{equation}

This approach is not exact, because the states being bound implicate
their lifetimes to be infinite, which they are intrinsically
to the problem not supposed to be. However, for states having a long lifetime,
this approximation is reasonable.

Together with a set of continuum wavefunctions $\{\Phi(\alpha,E)\}$, they are
defined to fulfill the following orthogonality relations

\begin{align}
  \braket{\Phi_n|\Phi_n} = 1 \quad  & \quad \braket{\Phi_n|\Phi(\alpha,E)} = 0\\
  \braket{\Phi(\alpha,E)|\Phi(\alpha',E')} & = \delta(\alpha-\alpha') \delta(E-E')
\end{align}

and thereby to form an orthonormal basis. These continuums wavefunctions
are characterized
by their energy $E$ and their quantum numbers, which are at this stage embraced
to the variable $\alpha$.

\begin{equation}
  1 = \sum\limits_n \ket{\Phi_n}\bra{\Phi_n} + \int \mathrm{d}\alpha \int \mathrm{d}E
      \ket{\Phi(\alpha,E)}\bra{\Phi(\alpha,E)} \label{feshbach_1}
\end{equation}

Expanding eq. \ref{se_ppsi} into this complete set yields an effective
final state Hamiltonian of

\begin{equation}
  \mathscr{H} = H_{PP}\, + \,
  \sum\limits_n H_{PQ} \,\frac{\ket{\Phi_n}\bra{\Phi_n}}{E-\varepsilon_n}\, H_{QP} \,+\,
  \int \mathrm{d}\alpha \int\mathrm{d}E \,H_{PQ} \,
  \frac{\ket{\Phi(\alpha,E)}\bra{\Phi(\alpha,E)}}{E-\varepsilon} \, H_{QP}
\end{equation}

which is useful to split into two parts: One describing the interaction with the
initial state $\Phi_s$ with its energy $\varepsilon_s$ being in resonance with
the continuum and the rest
$\mathscr{H}'$

\begin{equation}
  \mathscr{H} = \mathscr{H}' + H_{PQ} \,\frac{\ket{\Phi_s}\bra{\Phi_s}}{E-\varepsilon_s}\, H_{QP}
\end{equation}

with
\begin{equation}
  \mathscr{H}' = H_{PP}\, + \,
  \sum\limits_{n\ne s} H_{PQ} \,\frac{\ket{\Phi_n}\bra{\Phi_n}}{E-\varepsilon_n}
  \, H_{QP} \,+\,
  \int \mathrm{d}\alpha \int\mathrm{d}E \,H_{PQ} \,
  \frac{\ket{\Phi(\alpha,E)}\bra{\Phi(\alpha,E)}}{E-\varepsilon} \, H_{QP} .
\end{equation}

This reformulation leads to the following version of eq. \ref{se_ppsi}
\begin{equation}
  (E - \mathscr{H}')\, P \Psi =
   H_{PQ} \,\frac{\ket{\Phi_s}\bra{\Phi_s}}{E-\varepsilon_s}\, H_{QP} P \Psi %= \mathscr{V} P \Psi.
\end{equation}

$P \Psi$ has to be described in means of the final states in the asymptotic
region, which means that the wavefunction has to be described by the means
of the escaped particle. Therefore
the eigenfunctions of $\mathscr{H}'$ have to fulfill incoming boundary conditions,
which is labelled by the superscript $(-)$

\begin{equation}
  (\mathscr{H}'-E) \psi_f^{(-)} = 0 \label{sol_outg} .
\end{equation}

This relation can the be utilized to find a solution for $\ket{P \Psi}$ analogous
to the approach in equation (\ref{equation:in_state_0_approx}):

\begin{equation}\label{sol_ppsi}
  P \Psi = \psi_f^{(-)} + \frac{1}{\mathscr{H}' - E^{(-)}}
           \frac{H_{PQ}\ket{\Phi_s}
           \braket{\Phi_s|H_{QP}|P\Psi}}{E - \varepsilon_s} .
\end{equation}

$\ket{P \Psi}$ depends on itself and we want it to be expressed solely in terms
of $\Psi_f^{(-)}$. Therefore 
eq. \ref{sol_ppsi}
is multiplied from the left with $\bra{\Phi_s|H_{QP}}$ to give:
\begin{equation}
  \braket{\Phi_s|H_{QP}|P\Psi} = \braket{\Phi_s|H_{QP}|\Psi_f^{(-)}} +
  \frac{1}{\mathscr{H}' - E^{(-)}}
  \frac{\braket{\Phi_s|H_{QP}H_{PQ}|\Phi_s} \braket{\Phi_s|H_{QP}|P\Psi}}
       {E - \varepsilon_s}  \label{s_ppsi} .
\end {equation}

Defining the quantity

\begin{equation}
  W_{QQ} = H_{QP}\frac{1}{\mathscr{H}' - E^{(-)}}H_{PQ}
\end{equation}

and solving eq. \ref{s_ppsi} for

\begin{equation}
  \braket{\Phi_s|H_{QP}|P\Psi} = \frac{\braket{\Phi_s|H_{QP}|\Psi_f^{(-)}}(E-\varepsilon_s)}
{E - \varepsilon_s - \braket{\Phi_s|W_{QQ}|\Phi_s}}
\end{equation}

yields the final state description
\begin{equation}\label{}
  P \Psi = \psi_f^{(-)} + \frac{1}{\mathscr{H}' - E^{(-)}}
           \frac{H_{PQ}\ket{\Phi_s}
           \braket{\Phi_s|H_{QP}|\psi_f^{(-)}}}
           {E - \varepsilon_s - \braket{\Phi_s|W_{QQ}|\Phi_s}} .
\end{equation}



The above mentioned matrix of transition amplitudes $\mathcal{T}$
is in our case given by


\begin{align}
  \mathscr{T}_{if} &= \braket{\psi_i^{(+)} | P\Psi} \\
                   &= \mathscr{T}_{if}^{(P)} + 
                     \frac{\braket{\psi_i^{(+)}|H_{PQ}|\Phi_s}
                           \braket{\Phi_s|H_{QP}|\Psi_f^{(-)}}}
                          {E-\varepsilon_s - \braket{\Phi_s|W_{QQ}|\Phi_s}}
\end{align}

where $\psi_i^{(+)}$ in case of a full-collision process describes
the solution of the Schrödinger with outgoing
boundary conditions

\begin{equation}                                                              
  (E - \mathscr{H}') \psi_i^{(+)} = 0 .                                       
\end{equation}

where $\mathscr{T}_{if}^{(P)}$ describes the transitions without any interaction
between the initial and final states taken into account. Therefore, close to   
the resonance, this part is expected to be small compared to the second part,which
describes the transition
from the initial into the final states mediated by the interaction between them.

From the transition amplitudes we can now gain information about
the resonance energy from its poles. Therefore we need to examine
$\braket{\Phi_s|W_{QQ}|\Phi_s}$ further.
We split it into
its real and imaginary part by introducing a delta function using 
$\lim\limits_{\varepsilon \to 0_+} \frac{1}{x \pm i\varepsilon}
 = \mathscr{P} \frac 1x \mp i\pi\delta(x)$,
where $\mathscr{P}$ is the principal part and $\delta(x)$ denotes the
Dirac delta function. \cite{Cohen-Tannoudji_3_2}

\begin{align}
  \braket{\Phi_s|W_{QQ}|\Phi_s} & = \Delta_s(E) - i \frac{\Gamma_s(E)}{2}\\
                                & = \braket{\Phi_s|H_{QP}
                                    \frac{1}{\mathscr{H}' - E^{(-)}}H_{PQ}|\Phi_s}\\
                                & = \braket{\Phi_s|H_{QP}
                                    \frac{\mathcal{P}}{\mathscr{H}' - E}H_{PQ}|\Phi_s}
                                    - i\pi \braket{\Phi_s|H_{QP}\delta(\mathscr{H}'-E)H_{PQ}
                                    |\Phi_s} 
\end{align}

Inserting the latter expression into the transition matrix yields

\begin{equation}
  \mathscr{T}_{if} = \mathscr{T}_{if}^{(P)} + 
                     \frac{\braket{\psi_i^{(+)}|H_{PQ}|\Phi_s}
                           \braket{\Phi_s|H_{QP}|\Psi_f^{(-)}}}
                          {E-\varepsilon_s - \Delta_s + i \frac{\Gamma}{2}}
\end{equation}

from which now the real part of $\braket{\Phi_s|W_{QQ}|\Phi_s}$ can be interpreted 
as an energy shift $\Delta_s(E)$ of the resonance introduced by the interaction of the initial
with the final states and its imaginary part to carry the information about the
decay width $\Gamma_s(E)$ and hence the lifetime.


From the imaginary part of $\braket{\Phi_s|W_{QQ}|\Phi_s}$ the decay width can
easily be concluded. Inserting a complete set
of eigenfunctions with incoming boundary conditions as defined in
eq. \ref{sol_outg} the decay width can be described
as a sum over the different open channel solutions for a given resonant initial state.
\begin{align}
  \Gamma_s & = 2 \pi \braket{\Phi_s|H_{QP}\delta(\mathscr{H}'-E)H_{PQ}|\Phi_s}   \label{equation:Gamma_HE}\\
           & = 2 \pi \sum\limits_r \left | \braket{\Phi_s|H_{QP}|\psi_r^{(-)}} \right|^2
                \\
           & = \sum\limits_r \Gamma_{sr}(E)
\end{align}
Here it has to be remembered, that $E$ is the energy of a final state, which means, that
only such functions give a contribution, which have the same energy as those.
Definitiv nochmal ansehen, intuitiv müsste hier ini,fin hin.

This formulation is sufficient for all cases, we are going to discuss in this thesis.
However, Howat \textit{et al.} have proposed a slightly different ansatz with
a different partitioning, where configurations can contribute both to the description
of the initial and final state subspace. The expression for the decay width $\Gamma$
is very similar to the above approach in equation \ref{equation:Gamma_HE}, with more
complex conditions in the delta function.\cite{Howat78} This approach is not valid for exact
eigenfunctions, but may be useful in case of mappings to $\mathcal{L}^2$ functions as
e.g. \ac{CI}. The resulting expression reads as

\begin{equation}
  \Gamma_s = 2 \pi \left| \braket{\Phi_s|H-E|\psi_r^{(-)}} \right| ^2
\end{equation}
and is called \emph{Fano's Golden Rule}.
