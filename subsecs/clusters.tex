\section{Clusters}

Clusters are aggregations of atoms or molecules connected via non-covalent
bonds such as van der Waals forces, hydrogen bonds or metallic bonding.
They are an interesting
matter because their properties lie in between the properties of on the one hand
atoms and molecules and solid state matter on the other hand.

The investigation of clusters touches the basic question, from what point
on a solid actually is a solid matter and not a very big cluster.
This question can be answered looking at several property criteria of solids,
but the criteria not necessarily allow to draw the same conclusion.

In solids atoms or molecules ideally have a fixed postion with respect to their
neighbours. These structures can be very ordered like in crystals and metals
or amorphous. In all cases the microscopic structure is characterizing for
the material and also determines its macroscopic
porperties.
The structures can differ between small clusters and the macroscopic
material. For example small rare gas clusters show an overall icosahedral
structure, whereas the solid shows a space filling fcc structure with
a distinct bondlength between every atom with its nearest neighbours.
A transition of the most favourable strucutre pattern can be utilized
for the transition between a cluster and the solid.

Solids are able to both melt and sublimate. Microscopically melting
is being defined as atoms or
molecules changing from having a fixed position within the ensemble to a
random position with respect to their neighbours, but still being part
of the clusters, whereas sublimation can be defined as atoms or molecules
moving far enough away from the rest of the cluster to not being captured
again.
Both melting and sublimation themselves can be achieved already for very
small clusters
sizes consisting of not more than two atoms. From a certain size on,
clusters melt and sublimate under the same conditions as macroscopic
solids.

Further criteria can be conductivity and magnetic susceptibility.

Research on clusters very often use noble gas clusters as workhorses for
the investigation of basic properties.
In general atoms are preferred for the investigation of clusters because
they are spherical symmetric and hence the orientation of the substituents
in the cluster is not to be investigated.
Noble gases as such are theoretically easy to describe because of their
electronic closed shell structure.
They only interact via
van der Waals forces and hence the atomic electronic states are very
localized to a specific atom and are therefore
reasonably well starting points for the description of atoms within the
clusters. 
Additionally the electronic closed shell structure leads to a very well
defined ground states which allows to neglect difficulties arising from
multi-reference states.

Experimentally rare gases clusters are beneficial because the noble gases
already are gaseous, are
thermically stable, easy to introduce into the apparatus
and the cleaning of the
apparatus after the experiment is fast and easily done compared to
e.g. metals.


\subsection{Noble Gas Clusters}
Noble gas clusters contain noble gases of either one or several types of
noble gas atoms. The several atoms can be classified based on
their position within a cluster. Atoms in the bulk are surrounded by
neighbours, whereas atoms in the surface region miss interaction
partners on at least on side as displayed in figure \ref{figure:cluster_cut} showing
a cut through a cuboctahedral cluster.

\begin{figure}[h]
 \centering
 \begin{minipage}[htb]{0.4\textwidth}
   \centering
   \begin{tikzpicture}[scale=1.0,>=stealth]
       %\tiny
       %\draw [help lines] (-3,-3) grid (5,5);
       \draw [diplom1,pattern color=diplom1,pattern=north east lines,thick]
             (0:0) circle (0.5);
       \foreach \alpha in {0,60,...,300}
         \draw [diplom1,pattern color=diplom1,pattern=north east lines,thick]
               (\alpha : 1.0) circle (0.5);
       \foreach \alpha in {0,60,...,300}
         \draw [diplom2,pattern color=diplom2,pattern=north east lines,thick]
               (\alpha : 2.0) circle (0.5);
       \foreach \alpha in {30,90,...,330}
         \draw [diplom2,pattern color=diplom2,pattern=crosshatch,thick]
               (\alpha : 1.7321) circle (0.5);
   \end{tikzpicture}
\end{minipage}
\hspace{0.5cm}
\begin{minipage}[htb]{0.4\textwidth}
%Legende
\addlegendimageintext{pattern=north east lines, pattern color=diplom1, area legend} core atom

\addlegendimageintext{pattern=north east lines, pattern color=diplom2, area legend} surface atom -- vertex

\addlegendimageintext{pattern=crosshatch, pattern color=diplom2, area legend} surface atom -- edge
\end{minipage}

 \caption{Cut through a cuboctahedral cluster illustrating the different
          kinds of atoms within a cluster. The bright blue colored atoms
          are bulk atoms whereas the dark blue atoms belong to the surface.
          Within a shell atoms can either be vertices (striped), belong to
          an edge (crosshatched) or lie within a layer (not depicted in this
          two dimensional sketch.}
 \label{figure:cluster_cut}
\end{figure}

Within a shell each atom either is a vertex atom, an atom in an edge
or in plane as shown in figure \ref{figure:cluster_cut}.

The noble clusters are held together by the London dispersion force
stemming from the interaction of induced dipole moments in the polarizable
noble gas atoms, which is a special case of the van der Waals forces.
The interaction potential between two atoms is therefore reasonably
well described by the so called
Lennard-Jones potential

\begin{equation}
  V_{LJ}(R) = 4 \varepsilon \left( \left(\frac{\sigma}{R}\right) ^{12}
              - \left(\frac{\sigma}{R}\right) ^{6} \right)
\end{equation}
where R denotes the internuclear distance, 
and $\varepsilon$ is the depth of the
potential. Due to different conventions of the formulation
of the Lennard-Jones potential $\sigma=\frac{R_{min}}{\sqrt[6]{2}}$
with $R_{min}$ being the distance of the potential minimum.

\begin{figure}[h]
 \centering
 \begin{tikzpicture}
    \begin{axis}[domain=3.2:9.0,
                 samples = 200,
                 xtick={3,5,...,9},
                 %xticklabels={$-\pi$,$-\frac \pi 2$,0,$\frac \pi 2$,$\pi$},
                 extra x ticks={3.82086},
                 extra x tick labels={$\sqrt[6]{2} \sigma$},
                 cycle list name = exotic,
%                 legend style={anchor= north west},
                 xlabel = {R [\AA]},
                 ylabel = {V [meV]}
                 ]
    \addplot+[mark = none,
             thick,
             diplom1
             ]
             {4*12.4 * ( (3.404/x)^(12) -  (3.404/x)^(6) )}; 
    \addlegendentry{Ar-Ar}
    \addplot+[mark = none,
             dashed,
             gray
             ]
             {0.0}; 
    \addplot+[mark = none,
             dashed,
             gray
             ]
             {-12.4}; 
    \draw [thick,<->] (axis cs:8.0,-12.4) --
          node [right=2pt] {$\varepsilon$} (axis cs:8.0,0.0);
    
    \end{axis}
\end{tikzpicture}

 \caption{Lennard-Jones potential of the argon dimer illustrating the
          Lennard-Jones parameters $\sigma$ and $\varepsilon$.}
 \label{figure:LJ_Ar2}
\end{figure}

It contains a repulsiv part, taking care of the description of the
Pauli repulsion due to overlapping orbitals at short distances. The
$R^{-12}$ dependency is not theoretically justified but convenient
for an analytic description. The attractive $R^{-6}$ part describes
the London dispersion interaction.
For the different noble gases the Lennard-Jones parameters are given in
table \ref{table:LJ_parameter} and the potentials are shown in
figure \ref{figure:LJ_Ar2}.

\begin{table}[htb]
 \caption{Lennard-Jones parameters for noble gas dimers used in this
          thesis.}
 \centering
 \begin{tabular}{lcc}
   \toprule
   Atom type & $\sigma$ [\unit{\AA}] & $\varepsilon$ [\unit{meV}]\\
   \midrule
   Ne-Ne     & &\\
   Ne-Ar     &&\\
   Ar-Ar     &                 3.404 & 12.4\\
   Ar-Xe     &                 3.72  & 14.4\\
   Xe-Xe     &                 3.961 & 23.1\\
   \bottomrule
 \end{tabular}
 \label{table:LJ_parameter}
\end{table}


Homonuclear noble gas clusters from the distance look like a sphere
because for minimizing
the energy of the cluster the interaction between them needs to be
maximized. Atoms at the surface of a cluster have less interaction
partners than atoms in the bulk and the three dimensional structure
with the smalles surface to volume
ratio $\frac{S}{B}$ is the sphere. 

At a closer look the structure is not able to be a perfect sphere
because of the shape of the constituing atoms. Both theoretical
calculations and mass spectrometry of rare gas mixtures show, that
for small clusters, the icosahedral structure, where icosahedral
shells are nested into each other, is energetically beneficial.
For a more detailed description on how to construct such an icosahedral
structure see \ref{}.

The number of atoms can be calculated via

\begin{equation}
  n_{atoms} = \frac{10}{3} c^3 - 5 c^2 + \frac{11}{3} c -1 .
\end{equation}

where $x$ denotes the number of shells, where the central atom is
also counted as a shell. \cite{Martin96} This way, $c$ is
identical to the number of atoms
in the edge of the outermost shell.


In contrast to this, larger clusters and the solid show a cuboctahedral
pattern, which infinitely extended leads to a \ac{fcc} structure.
In the \ac{fcc} structure every atom is symmetrically surrounded
by 12 atoms in the same distance, while in icosahedral structures, the
atomic distance between atoms of different shells are smaller than the
interatomic distance within a shell. A large number of surface atoms
will therefore induce an icosahedral structure.
The larger the clusters are, the smaller is their surface to bulk
ratio $\frac{S}{B}$ and therefore the influence of of the surface
atoms on the overall structure decreases as illustrated in
figure \ref{figure:surface_to_bulk}. From a certain size on the
preferred structure changes from icosahedral to cuboctahedral. This size
is not exactly known and a topic of ongoing research. It is well accepted,
that the transition is supposed to lie somewhere between \unit[800--3000]{atoms},
which corresponds to $c = 7 - 11$.

\begin{figure}[h]
 \centering
 \begin{tikzpicture}
    \begin{axis}[domain=2.0:13,
                 samples = 12,
                 %xtick={3,5,...,9},
                 %xticklabels={$-\pi$,$-\frac \pi 2$,0,$\frac \pi 2$,$\pi$},
                 %extra x ticks={3.82086},
                 %extra x tick labels={$\sqrt[6]{2} \sigma$},
                 cycle list name = exotic,
%                 legend style={anchor= north west},
                 ylabel = {$\frac{{surface}}{{bulk}}$},
                 xlabel = {$c$}
                 ]
    \addplot[mark = *,
             thick,
             diplom1
             ]
             {(10/3*x^3 - 5*x^2 + 11/3*x -1) /
              (10/3*(x-1)^3 - 5*(x-1)^2 + 11/3*(x-1) - 1) - 1}; 
             %{(10/3*(x-1)^3 - 5*(x-1)^2 + 11/3*(x-1) -1) /
             % ((10/3*x^3 - 5*x^2 + 11/3*x -1) - (10/3*(x-1)^3 - 5*(x-1)^2 + 11/3*(x-1) - 1))}; 
    \end{axis}
\end{tikzpicture}

 \caption{Surface to bulk ratio for icosahedral structures with $c$ complete
          shells illustrating the decreasing impact of the surface atoms.}
 \label{figure:surface_to_bulk}
\end{figure}

The structure of heteronuclear clusters is expected to be similar to the
homonuclear rare gas cluster with a core consisting of the more polarizable
atom type somehow surrounded by atoms of the other atom type, which will be discussed
in the next section. To my knowledge no more detailed information about
the structure of heteronuclear rare gas cluster is known in the
literature. A new method for the investigation of the structure is found
in this thesis.


\subsection{Creation of Noble Gas Clusters}
In order to create stable clusters, its kinetic energy $k_BT$ needs to be lower
than the sum over the cohesive enrgies in the cluster.
In the experiment, noble gas clusters are formed via expansion through
a supersonic jet. The gas flows through the jet, driven by the lower pressure
on the other end of the chamber. If the mean free path of an atom or molecule
$\lambda$ is large compared to the diameter $d$ of the nozzle, then the flow is
molecular. In case of $\lambda$ being much smaller than $d$, the atoms inside
the nozzle collide and thereby unify their velocity to the group velocity
in direction of the stream. Consequently the width of the lateral velocity
distribution is decreased and due to energy conservation the temperature
in expansion direction is decreased, which allows the formation of clusters.
In the case of the group velocity being larger than the speed of sound,
one calls this a supersonic jet.

Cluster growth starts with the creation of a dimer. Due to energy conservation
a three-particle collision is required. Two particles form a dimer and the
third particle removes kinetic energy. For all further additions of particles
a two particle collision is sufficient. The kinetic energy of the additional
atom is then used to excite vibrations within the cluster.

The likelihood for three-particle collisions is proportional to the pressure
in the nozzle squared. By increasing the pressure, therefore the creation
of Kristallisationskeime is supported. On the other hand the apparatus
is limited to the efficiency of the vacuum pump.

\subsubsection{Creation of Heteronuclear Noble Gas Clusters}
Heteronuclear noble gas clusters can either be created in two different ways:
the \emph{pick-up} method or co-expansion of the two different gases.
For the pick-up method the jet of homonuclear clusters is directed through
a volume with a high concentration of the other species. The clusters
collide with the second species which consequently is attached to the surface
of the parent cluster. Depending on the types of the species
and the experimental conditions only few atoms
of the second species stay on the surface of the parent cluster,
several atoms form a shell around the parent cluster or the second species
diffuses into the bulk of the parent cluster.

In the co-expansion method two gases are mixed before the expansion and
then led into the nozzle together. The species with the higher cohesive
energy is more likely to form dimers which then serve as Kristallkeim of
the cluster growth.



\subsection{Experimental Analysis Tools}
