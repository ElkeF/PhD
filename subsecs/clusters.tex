\section{Clusters}

Clusters are conglomerats of atoms or molecules connected via non-covalent
bonds such as van der Waals forces or metallic bonding. They are an interesting
matter because their properties lie in between the properties of on the one hand
atoms and molecules and solid state matter on the other hand.

The investigation of clusters touches the basic question, from what point
on a solid actually is a solid matter and not a very big cluster.
This question can be answered looking at several property criteria of solids,
but the criteria not necessarily allow to draw the same conclusion.

In solids atoms or molecules ideally have a fixed postion with respect to their
neighbours. These structures can be very ordered like in crystals and metals
or amorphous. In all cases the microscopic structure is characterizing for
the material and also determines its macroscopic
porperties.
The structures can differ between small clusters and the macroscopic
material. For example small rare gas clusters show an overall icosahedral
structure, whereas the solid shows a space filling fcc structure with
a distinct bondlength between every atom with its nearest neighbours.
A transition of the most favourable strucutre pattern can be utilized
for the transition between a cluster and the solid.

Solids are able to both melt and sublimate. Microscopically melting
is being defined as atoms or
molecules changing from having a fixed position within the ensemble to a
random position with respect to their neighbours, but still being part
of the clusters, whereas sublimation can be defined as atoms or molecules
moving far enough away from the rest of the cluster to not being captured
again.
Both melting and sublimation themselves can be achieved already for very
small clusters
sizes consisting of not more than two atoms. From a certain size on,
clusters melt and sublimate under the same conditions as macroscopic
solids.

Further criteria can be conductivity and magnetic susceptibility.

Research on clusters very often use noble gas clusters as workhorses for
the investigation of basic properties.
In general atoms are preferred for the investigation of clusters because
they are spherical symmetric and hence the orientation of the substituents
in the cluster is not to be investigated.
Noble gases as such are theoretically easy to describe because of their
electronic closed shell structure.
They only interact via
van der Waals forces and hence the atomic electronic states are very
localized to a specific atom and are therefore
reasonably well starting points for the description of atoms within the
clusters. 
Additionally the electronic closed shell structure leads to a very well
defined ground states which allows to neglect difficulties arising from
multi-reference states.

Experimentally rare gases clusters are beneficial because the noble gases
already are gaseous, are
thermically stable, easy to introduce into the apparatus
and the cleaning of the
apparatus after the experiment is fast and easily done compared to
e.g. metals.


\subsection{Noble Gas Clusters}

\subsection{Experimental Creation of Noble Gas Clusters}
\subsection{Experimental Analysis Tools}
