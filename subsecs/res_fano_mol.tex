\chapter{ICD Decay Widths of ArXe Obtained by the Relativistic FanoADC}
\label{chapter:fano_arxe}
\section{Computational Details}
Both, the relativistic and the non-relativistic decay width calculations
were performed using the 
FanoADC(2x) and Stieltjes routines implemented in Dirac \cite{DIRAC13}.
The relativistic results were obtained using the
Dirac-Coulomb Hamiltonian with an approximate treatment of the two-electron
integrals over the small components (LVCORR)
in four-component calculations.

Both the relativistic and the non-relativistic calculations were performed
with the cv4z basis set of Dyall \cite{dyall5p06} and
additional \ac{KBJ} \cite{Kaufmann89}
basis functions centered on the geometric mean between the argon and the xenon
atom. For this purpose five $s$, $p$ and $d$ basis functions each were used.
The active space included the valence shell of the argon and the xenon atom and
the virtual orbitals were included up to \unit[15.0]{a.u.}. 
This lead to 420 and 330 active spinors in the relativistic and non-relativistic
case, respectively.

In all cases the channels are either closed or very close to threshold.
Therefore, the pseudo-spectrum contains only very few points for energies
lower than the resonance energy. Therefore, the moments of the lower orders of
Stieltjes might not cover the resonance energy. The calculated decay width
is then obtained from extrapolation of the interpolated curves for higher energies.
In the subsets of the pseudo-spectrum for the
calculation of the partial decay widths even less or no points are given at all,
therefore the partial decay widths might be less reliable.


\section{Results}
In Figure \ref{figure:arxe_gamma_shift_fano}, the total decay widths obtained
from the FanoADC calculations and the asymptotic approximations are shown.

\begin{figure}[]
  \centering
  \begin{tikzpicture}[scale=1.0]

\begin{loglogaxis}[%scale=1.5,
             domain=3:30,
             y domain=1E-8:10,
             restrict expr to domain={y}{1E-8:15},
             xlabel={R in \AA},
             xtick={2,4,...,10,12,15,...,25},
             xticklabels={2,4,6,8,10,12,15,20,25},
             ylabel={$\Gamma(R)$ in \unit{eV}},
             %title={Parameter Fitting of NeNe and NeAr Decay Widths}
             ]

\addplot[only marks,
         mark=*,
         thick,
         diplom1
        ]
        table[
        x expr = \thisrowno{0},
        y expr = \thisrowno{1}
        ]
        {data/arxe_nrel.dat};
        \addlegendentry{nrel FanoADC};
	
\addplot[%only marks,
         mark=o,
         thick,
         diplom1
        ]
        table[
        x expr = \thisrowno{0},
        y expr = \thisrowno{1}
        ]
        {data/arxe_nrel_unshifted.dat};
        \addlegendentry{nrel unshifted};
	
%\addplot[%only marks,
%         mark=o,
%         thick,
%         diplom1
%        ]
%        table[
%        x expr = \thisrowno{0},
%        y expr = \thisrowno{1}
%        ]
%        {data/arxe_nrel_shifted.dat};
%        \addlegendentry{nrel shifted};

\addplot[only marks,
         mark=*,
         thick,
         diplom2
        ]
        table[
        x expr = \thisrowno{0},
        y expr = \thisrowno{1}
        ]
        {data/arxe_rel_total.dat};
        \addlegendentry{rel FanoADC};
	
\addplot[%only marks,
         mark=o,
         thick,
         diplom2
        ]
        table[
        x expr = \thisrowno{0},
        y expr = \thisrowno{1}
        ]
        {data/arxe_rel_unshifted.dat};
        \addlegendentry{rel unshifted};
	
%\addplot[%only marks,
%         mark=o,
%         thick,
%         diplom2
%        ]
%        table[
%        x expr = \thisrowno{0},
%        y expr = \thisrowno{1}
%        ]
%        {data/arxe_rel_shifted.dat};
%        \addlegendentry{rel shifted};

		
\end{loglogaxis}
\end{tikzpicture}

  \caption{Distance dependence of the decay widths of the heteronuclear
           ArXe dimer obtained from both relativistic and non-relativistic
           calculations using the asymptotic approximation and the FanoADC
           approach.}
  \label{figure:arxe_gamma_shift_fano}
\end{figure}

The \textit{ab initio} results of the relativistic and the non-relativistic
calculations are very close and differ by no more than a factor of 2.
However, which decay width is higher is not to be determined, because the
curves cross each other.
Since the factor of 2 is the approximate error of the calculations,
these findings are within the errors bars and no statement
about whether relativistic effects influence the total decay width can be
concluded in this case.
They both approximately show an $R^{-6}$ behaviour at distances larger than
$\approx 2R_{eq}$. Also for smaller distances the decay widths only differ
significantly at distances below \unit[5]{\AA}.

In comparison to the decay widths obtained from the asymptotic approximation the
\textit{ab initio} decay widths are higher by a factor of about 6. This behaviour
is to be expected, since the asymptotic approximation does not take any overlap
effects into account. Additionally, it relies on the truncation of a series over
$R$ and hence higher order terms are neglected as well. However, the asymptotic
approximation does serve as a lower bound to the decay width.

Another feature is the channel opening and closing. The FanoADC-Stieltjes
approach relies on an incomplete pseudo-spectrum from which points of the
decay width are constructed and later interpolated and thereby the curves are
smoothed. This smoothing prohibits the possibility to determine whether a channel
is open or closed close to threshold. Additionally, even though the resonance energy
calculated from the initial state subspace should not differ too strongly from the
initial state's single ionization potential, small errors might be introduced by
the partitioning of the Hamiltonian. Additionally, the \ac{ADC}(2x) method as such
is not exact and might contribute to the error. All these factors necessitate
an interpretation of the scientist.
Possibilities to treat the channel opening and closing are to calculate the
channel opening distances with the ad hoc approach of equations (\ref{equation:E_in}
-- \ref{equation:E_sec}) or to determine the channel opening distance by a set of
single and double ionization calculations.


The total decay width of the relativistic FanoADC calculation can be devided
into partial decay widths of the following channels already presented in
section \ref{section:icd_geom}: Ar3p$_{3/2}^{-1}$Xe5p$_{3/2}^{-1}$,
Ar3p$_{1/2}^{-1}$Xe5p$_{3/2}^{-1}$, Ar3p$_{3/2}^{-1}$Xe5p$_{1/2}^{-1}$ and
Ar3p$_{1/2}^{-1}$Xe5p$_{1/2}^{-1}$. The sum of the partial decay widths should
equal the total decay width and from the calculations only small deviations
of up to \unit[10]{\%} were observed. If this is not the case, the
partial decay widths are meaningsless. The calculated decay widths are renormalized
to the total decay width and illustrated in
Figure \ref{figure:arxe_rel_fano_partial}.

\begin{figure}[]
  \centering
  \begin{tikzpicture}[scale=1.0]

\begin{loglogaxis}[%scale=1.5,
             domain=3:30,
             y domain=1E-8:10,
             restrict expr to domain={y}{1E-8:15},
             xlabel={R in \AA},
             xtick={2,4,...,10,15,...,25},
             xticklabels={2,4,6,8,10,15,20,25},
             ylabel={$\Gamma(R)$ in \unit{eV}},
             cycle list name = exotic
             %title={Parameter Fitting of NeNe and NeAr Decay Widths}
             ]

\addplot[only marks,
         black,
         mark=*,
         thick,
        ]
        table[
        x expr = \thisrowno{0},
        y expr = \thisrowno{1}
        ]
        {data/arxe_rel_total.dat};
        \addlegendentry{rel FanoADC};
	
\addplot+[only marks,
         mark=x,
         thick
        ]
        table[
        x expr = \thisrowno{0},
        y expr = \thisrowno{1}
        ]
        {data/arxe_rel_partial.dat};
        \addlegendentry{Ar3p$_{3/2}^{-1}$Xe5p$_{3/2}^{-1}$};
	
\addplot+[only marks,
         mark=x,
         thick
        ]
        table[
        x expr = \thisrowno{0},
        y expr = \thisrowno{2}
        ]
        {data/arxe_rel_partial.dat};
        \addlegendentry{Ar3p$_{1/2}^{-1}$Xe5p$_{3/2}^{-1}$};
	
\addplot+[only marks,
         mark=x,
         thick
        ]
        table[
        x expr = \thisrowno{0},
        y expr = \thisrowno{3}
        ]
        {data/arxe_rel_partial.dat};
        \addlegendentry{Ar3p$_{3/2}^{-1}$Xe5p$_{1/2}^{-1}$};
	
\addplot+[only marks,
         mark=x,
         thick
        ]
        table[
        x expr = \thisrowno{0},
        y expr = \thisrowno{4}
        ]
        {data/arxe_rel_partial.dat};
        \addlegendentry{Ar3p$_{1/2}^{-1}$Xe5p$_{1/2}^{-1}$};
	

\end{loglogaxis}
\end{tikzpicture}

  \caption{Renormalized partial decay widths calculated with the relativistic
           FanoADC of the ArXe dimer compared to each other and the total decay
           width.}
  \label{figure:arxe_rel_fano_partial}
\end{figure}

The actual decay width in ascending
order are given by the channels above.
These findings can be explained by two factors already discussed in the preceeding
section about the dependence of the decay with on the quantum numbers of the
initial and final states. The ratios of the channels between different decay widths
mainly depend on the ratio of the two different ionization cross sections
$\sigma_{3/2}$ and $\sigma_{1/2}$ and the ratio between the radiative lifetimes
$\chi = \frac{\tau_{1/2}}{\tau_{3/2}}$.

The ratio $\frac{\sigma_{3/2}}{\sigma_{1/2}}=1.6$ can be measured experimentally
\cite{Krause81} and for large distances the ratio of the partial decay widths should
behave accordingly. The ratios of the decay widths obtained from the
FanoADC-Stieltjes calculations are compared to the this experimental ratio in
Figure \ref{figure:arxe_rel_fano_ratios_sigma}.
\begin{figure}[]
  \centering
  \begin{tikzpicture}

\begin{axis}[
             domain=3:24,
             restrict expr to domain={x}{3:24},
             xlabel={R in \AA},
             ylabel={$\frac{\Gamma(Ar_{1/2})}{\Gamma(Ar_{3/2})}$},
             %xtick={2,4,...,24},
             %xticklabels={2,4,6,8,10,15,20,25},
             cycle list name = exotic
            ]

\addplot[
         black,
         no marks,
         thick
         ]
        {1.6};
        \addlegendentry{$\frac{\sigma_{3/2}}{\sigma_{1/2}}$};

\addplot+[
         ]
        table[
        x expr = \thisrowno{0},
        y expr = \thisrowno{1} / \thisrowno{3}
        ]
        {data/arxe_rel_partial.dat};
        \addlegendentry{Ar3p$_{3/2}^{-1}$};

\addplot+[
         ]
        table[
        x expr = \thisrowno{0},
        y expr = \thisrowno{2} / \thisrowno{4}
        ]
        {data/arxe_rel_partial.dat};
        \addlegendentry{Ar3p$_{1/2}^{-1}$};

\end{axis}

\end{tikzpicture}

  \caption{Partial decay width ratios for decay channels of the same final
           state on the initially ionized atom compared to the experimentally
           oberserved ratio between the ionization cross sections of the xenon
           atom. The latter value provides the asymptotic limit.}
  \label{figure:arxe_rel_fano_ratios_sigma}
\end{figure}

The agreement of the decay width ratios to the ratios of the ionization cross
sections are reasonable for distances between 5 and \unit[10]{\AA}. At shorter
distances the ratio is much higher than expected. But especially at small distances
overlap effects and higher order terms of the expansion over $R$ might play
a non-negligible role. At larger distances the decay width ratios should converge
to the asymptotic limit. However, they obviously do not and hence I conclude
that these partial decay widths and perhaps also the total decay width calculations
with the chosen technical parameters and the current implementation are not
reliable. This behaviour is observed for FanoADC-Stieltjes calculations
\cite{Kolorenc_private,Stumpf_private} in most systems.

Equation (\ref{equation:gamma_ratio_arxetype}) provides the asymptotic limit
for the ratios of the decay widths with different quantum numbers at the initially
ionized atom. Approximated from the atomic ionization energies of argon and the
experimentally determined value of $\chi = 2.05$ \cite{Jans97}
the asymptotic ratio of these partial decay widths is given by
$\frac{\Gamma_{1/2}}{\Gamma_{3/2}} = 0.52$. This number is compared to
the results from the FanoADC calculation in Figure
\ref{figure:arxe_rel_fano_ratios}.

\begin{figure}[]
  \centering
  \begin{tikzpicture}

\begin{axis}[
             domain=3:24,
             restrict expr to domain={x}{3:24},
             xlabel={R in \AA},
             ylabel={$\frac{\Gamma(Ar_{1/2})}{\Gamma(Ar_{3/2})}$},
             %xtick={2,4,...,24},
             %xticklabels={2,4,6,8,10,15,20,25},
             cycle list name = exotic
            ]

\addplot[
         black,
         no marks,
         thick
         ]
        {0.5212};
        \addlegendentry{$\frac{\Gamma_{1/2}}{\Gamma_{3/2}}$};

\addplot+[
         ]
        table[
        x expr = \thisrowno{0},
        y expr = \thisrowno{2} / \thisrowno{1}
        ]
        {data/arxe_rel_partial.dat};
        \addlegendentry{Xe5p$_{3/2}^{-1}$};

\addplot+[
         ]
        table[
        x expr = \thisrowno{0},
        y expr = \thisrowno{4} / \thisrowno{3}
        ]
        {data/arxe_rel_partial.dat};
        \addlegendentry{Xe5p$_{1/2}^{-1}$};

\end{axis}

\end{tikzpicture}

  \caption{Partial decay width ratios compared to the asymptotic limit provided
           by equation (\ref{equation:gamma_ratio_arxetype}).}
  \label{figure:arxe_rel_fano_ratios}
\end{figure}

As for the ratio between the ionization cross sections the ratios of the partial decay
widths for medium interatomic distances between \unit[5]{\AA} and \unit[15]{\AA}
seem to be reasonably well described. At shorter distances the ratio is higher
than expected from the asymptotic limit. At larger distances the description
should be expected to be better, but it is worse for the Xe5p$_{3/2}^{-1}$ case.

Therefore, the partial decay width calculations in the range between distances
up to \unit[10]{\AA} can be assumed to be trustworthy, while the
partial decay widths for larger distances would need a more elaborate treatment.



\section{Possible Improvements}
The currently implemented projectors for the partial decay widths have already
previously been shown not to be trustworthy in all cases. Therefore, a more elaborate
definition of the channel projection operators might improve the results.
Additionally, a larger basis set might improve the results. A further augmentation
would increase the basis set size in the energy regions of the emitted electron
and further ghost atoms located between the argon and the xenon atom might prevent
the unexpected behaviour at distances larger than \unit[10]{\AA}, if the
drawback of the current setup is the loss of overlap between the basis functions.
However, at the current state of implementation the computational limit is reached
and larger basis sets can not be investigated with the computer clusters of the
group.


