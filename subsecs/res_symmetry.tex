\chapter{Dependence of \ac{ICD} Decay Widths on the Quantum Numbers}

In the working equation of the asymptotic \ac{ICD} decay widths in the
$jj$-coupling
picture (\ref{reltheolifetime_exp})
\begin{align}
 \Gamma_\beta =& \frac{2\pi}{R^6} \sum\limits_{M_A'} \, B_{M_A'-M_A}^2 \, \left| \left(
\begin{array}{ccc}
J_A'  & 1        & J_A\\
-M_A' & M_A'-M_A & M_A
\end{array}\right) \right|^2
 (2J_A+1)\frac{3c^4 \sigma^{(B)}(\omega_{vp})}{16\pi^2\omega_{vp}^4\tau_A} .
\end{align}

the total angular momenta of the initial and final states enter explicitely.
Therefore, the question arises, how the ratio between the decay widths behaves
both for the total angular momenta as well as their projections within the
same total angular momentum. In the LS-coupling scheme only one L-state is possible,
which is why an investigation of the behaviour of ratios is senseless in this
case. However, the ratios of the different projections of this angular momentum
can be obtained and have been shown to be \cite{Gokhberg10_1}

\begin{equation}
  \frac{\Gamma_0}{\Gamma_{\pm 1}} = 4  .
\end{equation}

The orbital with $L=1$ and $M_L=0$ corresponds to a p-orbital aligned along
the internuclear axis, whereas for $M_{\pm 1}$ the orbitals are aligned perpendicular
to the internuclear axis. In the picture of a classical oscillating dipole it is
to be expected to have a much more efficient induction of another dipole
in direction of the oscillation. The ratio of the decay widths therefore
corresponds to the expectations.

In a relativistic treatment including the $jj$-coupling scheme,
where the wavefunction
is a four component spinor with the elements being linear combinations
of LS-coupling functions, and different density functions the behaviour is
not evident a priori.

\section{Total Angular Momentum}
First we are going to assume a process
from a non-degenerate initial state with $J=\frac 12$ into different
p-type final state configurations
in the initially ionized atom for an \ac{ICD} process. Afterwards we are going
to assume different initial states of p-type orbitals with an s-type
character in the final state of the initially ionized atom.
However the classification is more accurate in terms of the order of
ionization energies of the split states. In the discussion we assume
$SIP_{3/2}<SIP_{1/2}$, which is not necessarily true for all atoms. E.g.,
the order of the ionization potentials of the calcium 3p orbitals are
switched.

In equation (\ref{reltheolifetime_exp}) most of the entities are independent
of the total angular momenta and therefore the ratio between the decay widths
reduces to

\begin{align}
  \frac{\Gamma_{1/2}}{\Gamma_{3/2}}
  &= \frac{P_{1/2}}{P_{3/2}} \, \frac{\sigma^B(\omega_{vp1/2})}{\omega_{vp1/2}^4 \tau_{1/2}}
     \,\frac{\omega_{vp3/2}^4 \tau_{3/2}}{\sigma^B(\omega_{vp3/2})}     
     \, \frac{2J_{A1/2}+1}{2J_{A3/2}+1}     \\
  &\approx \frac{P_{1/2}}{P_{3/2}} \,
        \frac{\omega_{vp3/2}^5}{\omega_{vp1/2}^5}  \,\frac{\tau_{3/2}}{\tau_{1/2}}
     \, \frac{2J_{A1/2}+1}{2J_{A3/2}+1}   \label{equation:GammaJ_ratio}\\
  &= \frac{P_{1/2}}{P_{3/2}} \, \frac{\omega_{vp3/2}^5}{\omega_{vp1/2}^5}
     \,  \frac{2J_{A1/2}+1}{2J_{A3/2}+1} \,\chi
\end{align}
where the sum of the products of the absolute square of
Wigner's 3j-symbol and $B_{M_A'-M_A}$
are grouped to the variable $P$.
The ionization cross section is proportional to the inverse of the energy
of the virtual photon $\sigma{\omega}\propto \frac 1 \omega$ and the ratio between
the lifetimes of the two different total angular momenta
$\chi= \frac{\tau_{3/2}}{\tau_{1/2}}$ is experimentally
accessible for atoms. Hence, the ratio of the decay widths does not depend
on properties of the atom $B$ explicitly.


\begin{table}[h]
 \centering
 \begin{tabular}{c|cccc}
  \toprule
  \backslashbox{$M_A$}{$M_A'$} & $\frac 32$             & $\frac 12$                   & $-\frac 12$            & $-\frac 32$\\
  \midrule
  $\frac 12$                   & $\frac 12\,\,\,^{(1)}$ & $-\sqrt{\frac 16}\,\,^{(0)}$ & $\sqrt{\frac 1{12}}\,\,^{(-1)}$ & --\\
  $-\frac 12$                  & --                     & $-\sqrt{\frac 1{12}}\,\,^{(1)}$ & $\sqrt{\frac 16}\,\,^{(0)}$  & $-\frac 12\,\,\,^{(-1)}$\\
  \bottomrule
 \end{tabular}
 \caption{Evaluation of Wigner's $3j$-symbols for $J=\frac 32$ in the argon-xenon
          dimer. The numbers in brackets denote the difference of the projections
          of the angular momenta $M_A'-M_A$.}
 \label{table:wignerarxe3}
\end{table}

\begin{table}[h]
 \centering
 \begin{tabular}{c|cc}
  \toprule
  \backslashbox{$M_A$}{$M_A'$} & $\frac 12$                   & $-\frac 12$\\
  \midrule
  $\frac 12$                   & $\sqrt{\frac 16}\,\,^{(0)}$ & $-\sqrt{\frac 1{3}}\,\,^{(-1)}$\\
  $-\frac 12$                  & $-\sqrt{\frac 1{3}}\,\,^{(1)}$ & $\sqrt{\frac 16}\,\,^{(0)}$\\
  \bottomrule
 \end{tabular}
 \caption{Evaluation of Wigner $3j$-symbols for $J=\frac 12$ in the argon-xenon
          dimer. The numbers in brackets denote the difference of the projections
          of the total angular momenta $M_A'-M_A$.}
 \label{table:wignerarxe1}
\end{table}


\subsection{One Initial State and Several Final States}
For one given initial state the total angular momentum is defined and hence
$J_{A3/2} = J_{A1/2}$. The energies of the virtual photon transferred between
the units of the \ac{ICD} differ for the two final state configurations of
the initially ionized atom as

\begin{align}
  \omega_{vp1/2} &= SIP_{in} - SIP_{fin1/2}  \\
  \omega_{vp3/2} &= SIP_{in} - SIP_{fin3/2}
\end{align}

In case of the \ac{SIP} of the $J=\frac 32$ state being lower than the \ac{SIP}
of the $J=\frac 12$ state, the difference in energy of the two virtual
photons is given by the positive spin-orbit coupling constant $a$.

\begin{align}
  \omega_{vp3/2} &= \omega_{vp1/2} + SIP_{fin1/2} - SIP_{fin3/2} \\
  \omega_{vp3/2} &= \omega_{vp1/2} + a
\end{align}

$P_{1/2}$ and $P_{3/2}$ are evaluated using the
expressions for the 3j-symbols given in Tables
\ref{table:wignerarxe3} and \ref{table:wignerarxe1} and $B_0=-2$, $B_{\pm 1}=1$
to give 1 in both cases.
From these considerations it follows that the ratio between the decay
widths of $J_A'=\frac 32$ and $J_A'=\frac 12$ is given by


\begin{align}
  \frac{\Gamma_{1/2}}{\Gamma_{3/2}}
  &= \frac{\omega_{vp3/2}^5}{\omega_{vp1/2}^5} \frac{2J_{A1/2}+1}{2J_{A3/2}+1} \,
     \frac{1}{\chi}\\
  &= \frac{(\omega_{vp1/2} +a)^5}{\omega_{vp1/2}^5} \frac{1}{\chi}
     \label{equation:gamma_ratio_arxetype}
\end{align}

In the non-relativistic limit the spin-orbit coupling constant $a$ is zero and
hence the ratio is given by the ratio of the two different lifetimes. Without
relativistic effects this ratio $\chi$ is determined purely by the degeneracy
of the states and therefore $\chi_{nrel} = 2$.
In the relativistic case $a>0$ and a splitting additional to the degeneray of
the final states configurations is to be observed. Furthermore $\chi$ varies
with the strength of the spin-orbit coupling. For increasing spin-orbit coupling
constant $\chi$ decreases.
In the neighbourhood of other atoms, the spin-orbit coupling constant is not
a real constant \cite{Freudenberg_Forsch}. The latter minor differences will further not be taken
into account.




\subsection{Several Initial States and One Final State}
Normally, the starting point for such gedankenexperiments is a given initial
state. However, decay widths from different initial states with the same
final state can also be compared. It has to be kept in mind, that
the sum of the resulting decay widths would not resemble a total decay width.

Consider the initially ionized atom to be an alkaline earth metal atom
being ionized from the
p-level. In this case the initial state can both be $J_A = \frac 32$ and
$J_A = \frac12$. The final state configuration being a vacancy in the
4s shell is defined by $J_A'= \frac 12$. Hence, in this section the decay
widths of two different initial states with the same final state are compared.
The virtual photon energies would then be defined as

\begin{align}
  \omega_{vp1/2} &= SIP_{in1/2} - SIP_{fin}  \\
  \omega_{vp3/2} &= SIP_{in3/2} - SIP_{fin}  \\
  \omega_{vp3/2} &= \omega_{vp1/2} + SIP_{fin3/2} - SIP_{fin1/2} \\
  \omega_{vp3/2} &= \omega_{vp1/2} - a 
\end{align}

Notice, that in this case the sign in front of the spin-orbit coupling
constant $a$ is different from the one for one specific initial state.
The last of the above equations only hold in the case of
$SIP_{fin1/2} > SIP_{fin3/2}$.

Considering the degeneracies of the states
$\frac{2J_{A1/2}+1}{2J_{A3/2}+1} = \frac 12$.
The evaluation of $\frac{P_{1/2}}{P_{3/2}}$ is possible for all kinds
of initial states with a specific value for the projection $M_A$, but
the focus of this section lies on the influence of the total
angular momentum without the consideration of their projections, here
the channels are defined with respect to $M_A'$ instead of
$M_A$. In this case $\frac{P_{1/2}}{P_{3/2}} = 1$.
Hence the ratio of the decay widths can be evaluated to be

\begin{align}
  \frac{\Gamma_{1/2}}{\Gamma_{3/2}}
  &= \frac{P_{1/2}}{P_{3/2}}
     \,\frac{\omega_{vp3/2}^5}{\omega_{vp1/2}^5}
     \,\frac{2J_{A1/2}+1}{2J_{A3/2}+1} \, \frac{1}{\chi}\\
  &= \frac 12 \frac{(\omega_{vp1/2} -a)^5}{\omega_{vp1/2}^5} \frac{1}{\chi}
\end{align}


\section{Projection of the Total Angular Momentum}
Again consider the ICD from a defined initial state with $J_A=\frac 12$ and
$M_A=\frac 12$. Within the final state specification of $J_A'=\frac 32$, several
projections $M_A'= +\frac 32, +\frac 12, -\frac 12$ are possible. In the asymptotic
picture, these three states are obviously degenerate, are described by the same
total angular momentum and have the same radiation lifetimes.

Hence, in equation (\ref{equation:GammaJ_ratio}) the only difference relies on $P$,
which has to be remembered to be a sum over the different possible projections
and orientation factors $B$. Evaluating the ratios for the different final state
projections yields

\begin{align}
  \frac{\Gamma_{+1/2}}{\Gamma_{+3/2}} &= \frac{P_{+1/2}}{P_{+3/2}} = \frac 83  \\
  \frac{\Gamma_{+1/2}}{\Gamma_{-1/2}} &= \frac{P_{+1/2}}{P_{-1/2}} = 8   .
\end{align}

In the non-relativistic investigation, the higher efficiency of the \ac{ICD} process
from a $z$-orbital ($M_L=0$) was explained by the spacial orientation of the
oscillating dipol along the internuclear axis.
However, in the relativistic investigation above, the spacial orientation
can not alone be responsible for the efficiency, because the relativistic
probability density of both $M_A'= +\frac 12$ and $M_A'= -\frac 12$ are aligned
along the internuclear axis. Still, the decay width into a final state characterized
by $M_A'= \frac 12$ is eight times as fast as the decay into the final state
with $M_A'= -\frac 12$. I therefore conclude that in addition to the spacial
orientation of the dipole, the phase of the final state wavefunction is
important as well.

An analogous treatment for the final state
$J_A'=\frac 12$ with $M_A'= +\frac 12, -\frac 12$
leads to
\begin{equation}
  \frac{\Gamma_{+1/2}}{\Gamma_{-1/2}} = \frac{P_{+1/2}}{P_{-1/2}} = 2  .
\end{equation}
Also in this case a difference is to be observed, even though both the initial
and final states' probability density are spherical symmetric.

%\begin{equation}
%\end{equation}
%
%\begin{equation}
%\end{equation}
%
%\begin{equation}
%\end{equation}
%
%\begin{equation}
%\end{equation}
%
%\begin{equation}
%\end{equation}
%
%\begin{equation}
%\end{equation}
%
%\begin{equation}
%\end{equation}
%
%\begin{equation}
%\end{equation}
%
