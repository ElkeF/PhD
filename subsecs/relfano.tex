\section{Relativistic FanoADC}

The FanoADC code used in this thesis is implemented in Dirac \cite{DIRAC13}
and based on the symmetry handling and calculation of matrix elements of
the relativistic \ac{ADC} and Lanczos code of Pernpointner
\cite{Pernpointner04_1,Pernpointner04_2}.

In this FanoADC code the partitioning by population is applied. Hence, the
three subspace matrices of the initial states, the interaction and the final
states can be obtained from resorting the available matrix elements. The
selection of the final state subspace is based on manually chosen $2h$
configurations.
Since the \ac{ADC} matrix can have dimensions
of $10^5\times 10^5$, they are not kept
in memory but stored to disk. A resorting from reading in these matrix elements
would be inefficient and therefore, the matrices are build correctly
sorted. The heart of the sorting is an array, in which the positions within
a $2h1p$ vector are stored, which correspond to final state configurations
of different channels called \verb|finalpos|.

Due to the hermiticity of the \ac{ADC} matrix, it is sufficient to store the
lower triangle and the diagonal, which are build separately. First, the
1h/1h block is constructed as in the \verb|reladc| code, since all
configurations contribute to the initial state subspace.

As in the \verb|reladc| code, the matrix is build columnwise.
In the
coupling block, this leads to an easy sorting by leaving out all those matrix
elements, for which the row index corresponds to a final state position within
a column.
At the end of the column, the skipped matrix elements are added to the
interaction part.

In the satellite block, both row and column indices can correspond to a
final state position. Still, the FanoADC matrices are constructed using
the column indices of the \verb|reladc| code. For those columns corresponding
to the intial state subspace, the sorting is obtained as in the coupling block
explained above. However, those matrix elements,
which stem from a column, which corresponds to a final state configuration,
those matrix elements, which mediate the interaction with the initial state
subspace need to be correctly sorted into the interaction matrix. Thereby,
the final state subspace matrix elements are left out. For the
interaction matrix elements, the indices are first shifted correctly and
then switched and the matrix elements are complex conjugated, due to the
hermiticity of the \ac{ADC} matrix.

The final state subspace of the satellite block is constructed calling
those matrix elements, corresponding to the final state positions stored
in \verb|finalpos|.

After the contruction of the three matrices, the final state matrix is
diagonalized by full diagonalization. Then, the initial state matrix
is diagonalized using the Lanczos routine. After the first Lanczos run,
a set of short eigenvectors with an overlap higher than a given threshold
with the initial state spinor and those with the same energies
is chosen for the calculation of long eigenvectors.
After calculation of these long eigenvectors, they are merged into linearly
independent vectors. From each of these long eigenvectors the interaction
is calculated using the interaction matrix and the final state vectors.

The pseudo-spectrum which afterwards enters the Stieltjes claculation
is obtained from the final state eigenenergies and the corresponding interaction
matrix elements.

The manual can be found as part of the Dirac manual from the release of 2014
on \cite{DIRAC13}.


