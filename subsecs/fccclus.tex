\section{\textsc{fccclus}}

Clusters with an \ac{fcc} structure have a dodecahedral shape. It consists
of a central atom surrounded by shells of other atoms. All distances 
between atoms and their nearest neighbours both within one shell and atoms
of different shells are equal.

They can be constructed starting from a central atom, which is surrounded by
12 atoms. Of these 12 atoms, six are positioned in a regular hexagon around
the central atom. Additional three are positioned in a triangle above the
plane defined by the hexagon and three below. Thereby, the triangles are rotated
by \unit[30]{\degree} with respect to the hexagon and \unit[60]{\degree}
with respect to each other.

Hence, the positions of the corners of every shell are with $i=0,1,\dots 6$
and $j=0,1,2$ given by

\begin{equation}
 \begin{pmatrix}
  a \cos \left(\frac\pi3 i \right)\\
  a \sin \left(\frac\pi3 i \right)\\
  0
 \end{pmatrix}
 \quad\quad
 \begin{pmatrix}
  -\frac{a}{\sqrt{3}} \sin \left(\frac{2\pi}{3} j \right)\\
   \frac{a}{\sqrt{3}} \cos \left(\frac{2\pi}{3} j \right)\\
  \sqrt{\frac23} a
 \end{pmatrix}
 \quad\quad
 \begin{pmatrix}
   \frac{a}{\sqrt{3}} \sin \left(\frac{2\pi}{3} j \right)\\
  -\frac{a}{\sqrt{3}} \cos \left(\frac{2\pi}{3} j \right)\\
  -\sqrt{\frac23} a
 \end{pmatrix}
\end{equation}

for the hexagon, the three atoms above and the three atoms below the plane.
Here, $a$ denotes the length of an edge, and at the same time from the central
atom to the vertex, defined by the sum of van der Waals
radii.

From the positions of these vertices, the atomic positions of atoms in the
edges and together with the latter, the coordinates of atoms in the triangular
and square planes are constructed.

For heteronuclear clusters $a$ ist determined by the sum over the van der Waals
radii in seen from the central atom to the vertex to be contructed.
