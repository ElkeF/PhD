\section{Autoionization Processes of Ionized Species}
Vacancies in the core or inner-valence region of atoms and molecules
can decay via photon emission, in case of molecules via coupling to
the nuclear motion or via several autoionization processes. All these
processes compete and hence the fastest are observed. In the auotionization
processes
the vacancy is filled with an electron from an outer shell and the excess
energy is simultaneously transferred to another electron, which consequently
is emitted. Hence the final state is characterized by a doubly ionized state
and an electron in the continuum. These processes can
occur when two criteria are fulfilled, the energy and the coupling criterion.
The energy criterion is fulfilled, when the doubly ionized system in the
final state is of lower energy than the singly ionized initial state.

The different auotionization processes can be classified by the initial
vacancy, where the
vacancy filling electron originates and where the secondary electron is
emitted from.

\subsection{Auger Process}
The Auger process, independently discovered by xyz Auger and Lise Meitner
\cite{}, is the longest known of the presented autoionization processes. Here
the initial ionization mostly resides in the core region of an atom. The vacancy
is then filled by another, energetically higher electron of the same atom, the
energy is simultaneously transferred to yet another electron of the same atom and
emitted as shown in figure \ref{figure:auger_process}.

\begin{figure}[h]
 \centering
%    \begin{tikzpicture}[scale=0.5,>=stealth]
       \tiny
%       \draw [help lines] (-3,-3) grid (20,5);
       \coordinate(ground)   at (0,0) {};
       \draw [very thick] (ground) circle (1.2);
       \draw [very thick] (ground) circle (0.5);
       \shade [ball color=white,path fading=fade inside] (ground) circle (1.5);
       \coordinate (in1)  at (60:0.5) {};
       \coordinate (in2)  at (240:0.5) {};
       \draw[->,thick,diplom2] (in1) -- (2,3);
       \foreach \alpha in {30,90,...,330}
         \shade [ball color=red] (\alpha :1.2) circle (0.2);
       \shade [ball color=red] (in1) circle (0.2);
       \shade [ball color=red] (in2) circle (0.2);
       \draw [->,decorate,decoration=snake,diplom1,thick] (-2,3) --
              (in1.north west);
%       \shade [ball color=white,path fading=fade inside] (ground) circle (1.5);

       \draw[->,very thick] (2.5,0) -- (3.5,0);

    \begin{scope}[xshift=6cm]
       \coordinate(ground)   at (0,0) {};
       \draw [very thick] (ground) circle (1.2);
       \draw [very thick] (ground) circle (0.5);
       \shade [ball color=white,path fading=fade inside] (ground) circle (1.5);
       \coordinate (in1)  at (60:0.5) {};
       \coordinate (in2)  at (240:0.5) {};
       \foreach \alpha in {30,90,150,...,330}
         \shade [ball color=red] (\alpha :1.2) circle (0.2);
       \draw [densely dashed,red,thick] (in1) circle (0.2);
       \shade [ball color=red] (in2) circle (0.2);
       \draw[->,thick,diplom2] (30:1.2) .. controls (0.7,0.8) .. (in1);
       \draw [->,decorate,decoration=snake,segment length=2mm,diplom1,thick]
              (0.5,0.5) -- (330:1.2);
       \draw[->,thick,diplom2] (330:1.2) -- (3,3);
       \node at (3.2,3.3) {e$_{Auger}$};

       \draw[->,very thick] (2.5,0) -- (3.5,0);

    \end{scope}

    \begin{scope}[xshift=12cm]
       \coordinate(ground)   at (0,0) {};
       \draw [very thick] (ground) circle (1.2);
       \draw [very thick] (ground) circle (0.5);
       \shade [ball color=white,path fading=fade inside] (ground) circle (1.5);
       \coordinate (in1)  at (60:0.5) {};
       \coordinate (in2)  at (240:0.5) {};
       \foreach \alpha in {90,150,...,300}
         \shade [ball color=red] (\alpha :1.2) circle (0.2);
       \shade [ball color=red] (in1) circle (0.2);
       \shade [ball color=red] (in2) circle (0.2);
       \draw [densely dashed,red,thick] (30:1.2) circle (0.2);
       \draw [densely dashed,red,thick] (330:1.2) circle (0.2);
       \shade [ball color=red] (3.2,3.3) circle (0.2);
    \end{scope}
   \end{tikzpicture}

 \caption{}
 \label{figure:auger_process}
\end{figure}


Since both the vacancy filling electron and the emitted electron reside
on the same atom as the initial vacancy, neither energy nor electrons need
to be transferred through space, this process is ultrafast with lifetimes
in the atto- to femtosecond region.

In order to fulfill the energy criterion, the initial ionization has to
be of quite high energy, which is why the process is rarely observed after
ionization from the inner-valence region and therefore the possibility to
observe other autoionization processes.

\subsection{Interatomic Coulombic Decay (ICD)}
The Interatomic/ Intermolecular Coulombic Decay (ICD) was predicted
theoretically 1997 by L. S. Cederbaum \cite{Cederbaum97}
and later verified experimentally by Marburger et al. \cite{Marbuger03}.

Here the initial vacancy is filled by an electron of the same atom and simultaneously,
in contrast to the Auger process, the atom interacts with the surroundings by
transferring the excess energy to another atom, which finally gets ionized. The
positive charges on different atomic sites repell each other and therefore undergo
Coulomb explosion as shown in figure \ref{figure:icd_process}.

\begin{figure}
\end{figure}

For the following sections we introduce a new nomenclature. The combinations
of all involved atoms and molecules is referred to as the (total) system, whereas
each of them is named a unit. One or a conglomerat of these units, within
the electron fills the vacancy is called a subsystem ($S_1$) as well as the units
forming the 
subsystem emitting the seconday electron $S_2$.

Due to the energy transfer via a virtual photon between to subsystems
the ICD has a lifetime of femto- to pico-seconds.
The initial vacancy of observed ICD processes are normally in the inner-valence,
but are not limited to these cases. Nevertheless, one might rarely observe
an ICD after core ionization, because the Auger process would then be energetically
allowed and outrule the slower ICD.



\subsection{Electron Transfer Mediated Decay Processes}
In the \ac{ETMD} processes.

\subsection{Resonant ICD (RICD)}
ETI excitation transfer ionization
\subsection{Interatomic Coulombic Electron Capture (ICEC)}
