\chapter{Relativistic Effects in Autoionization Processes}

For the auotionization processes, the influence of relativistic effects
can be grouped into three different parts:

\begin{enumerate}
  \item Spin-orbit coupling
  \item Different selection rules for electronic transitions
  \item scalar-relativistic effects.
\end{enumerate}

The spin-orbit coupling leads to an energetic splitting of non-relativistically
degenerate states with the total quantum number $J= L \pm \frac 12$.
For electronic decay processes with a defined initial state, the energies
of the involved final state orbitals depend on these total angular momenta.
For a doubly charged final state, two holes are involved. In the non-relativistic
description each of them would have a definite energy. However, due to the
spin-orbit splitting, each of those two different holes are energetically split
to give in total up to four energetically differing final state configurations.
These distinguishable channels will show as different peaks in a secondary electron
spectrum.
In case of an \ac{ICD} or \ac{ETMD}3, where the two charges in the final state are
seperated by some distance, each of these channels will open at different
geometries as will be shown in chapter \ref{chapter_geom}.

The selections rules for electronic transitions in the $jj$-coupling scheme
is $\Delta J = 0, \pm 1$. 
In the non-relativistic
formulation of $LS$-coupling, an electronic transition between two states
of same parity, such as $s$ and $d$, would be forbidden,
which is known as Laporte rule.
Due to the spin-orbit coupling, an $s$-orbital
has $J=0+\frac 12=\frac 12$. It is spherical symmetric. A $d$-orbital with
$L=2$ can both have the
total angular momentum $J=\frac 32$ and $J=\frac 52$. 
While the probability density of the state with $J=\frac 52$
more or less inhibits the shape of a
non-relativistic $d$-orbital and hence gerade symmetry, the overall
shape of the probability density of the $J=\frac 32$ state is closer to the
shape of the probability density of a non-relativistic $p$-orbital with
and inhibits ungerade symmetry. Hence the transition
between an $s_{1/2}$ and a $d_{3/2}$ orbital is allowed.

Scalar-relativistic effects play an important role in the absolute energies
of the ionization potentials by increasing the ionization energy out of
$d$- and $f$-orbitals and decreasing the ionization potential for $s$- and
$p$-orbitals. This shift might lead to an opening or closing of some channel.
However, until now such case is not known.


