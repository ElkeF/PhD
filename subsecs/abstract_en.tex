\chapter*{Abstract}
\thispagestyle{empty}
The Interatomic/ Intermolecular Coulombic Decay (ICD) as well as the
Electron Transfer Mediated Decay (ETMD) are electronic decay
processes, which occur in a multidude of systems ranging from
noble gas dimers to biological systems.
If heavy atoms are involved in these processes, relativistic effects
cannot be neglected. However, their influence has so far not
been investigated thoroughly.

In this thesis, the influence of the spin-orbit coupling as well as
scalar-relativistic effects on openings and closings of decay channels as well
as on the corresponding decay widths are studied.
For this purpose, asymptotic expressions for the decay widths of both
ICD and ETMD are derived. They allow for analytic studies of basic
properties and estimations of the decay widths based on properties
of the constituting atoms or molecules of the total system.
A more precise description of the decay widths required the transfer of
the non-relativistically known FanoADC-Stieltjes method to the relativistic
regime and its implementation into the relativistic quantum chemical
programme package Dirac. Using this method, small noble gas systems
are investigated. 

Experimentally, these decay processes are usually studied in noble gas
clusters consisting of 100 -- 2000 atoms. These clusters are to
large to be treated with \emph{ab initio} methods. In order to allow
for a comparison of theoretical and experimental results, the influence
of the cluster environment on the secondary electron spectra are
investigated. These findings are used for the development of a method
for the decay width estimation of clusters based on the asymptotic
expressions or calculated decay widths for a multitude of geometries.
This method was implemented as the programme HARDRoC and
is used for the investigation of the two competing processes ICD and ETMD in
ArXe clusters. Additionally, it is the foundation of a new
structure determination method of heteronuclear noble gas clusters,
which is exemplarily explained for NeAr clusters.
