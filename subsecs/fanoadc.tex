\section{FanoADC}
In the FanoADC, we use the discrete ADC Hamiltonian in the \ac{ISR}
basis for the construction of the pseudo-spectrum of the decay width
$\Gamma$.
Therefore
Due to multiple possible final states, we employ the procedure
of Feshbach using projection operators for the partitioning of the Hamiltonian
into an initial and a final state subspace.
The processes of our interest have a singly ionized initial state and a
doubly charged final state with an additional electron in the continuum.
Therefore, the final state configurations are chosen from
the class of $2h1p$ states, where the $2h$ part is assumed to
describe the doubly ionized final state and the particle maps the continuum electron.
The initial state in a zeroth order approximation can be described as a $1h$ state.

For the partitioning two different methods are known in the literature:
the selection by $2h$ contributions, which correspond to final states of
the process\cite{Averbukh05}, or by energy\cite{}.
The first is easy to implement while the second automatically
enables the correct calculation of decay widths in system with lower than
spherical symmetry, but which have inversion symmetry.


\subsection{Partitioning by Population}
In the actual partitioning, all $2h1p$ with the $2h$ part corresponding to
a final state configuration are chosen to be part of the final states.
All $1h$ configurations and those $2h1p$ configurations not corresponding
to a possible final state configuration are used for a better description
of the initial state. This leads to a resorted \ac{ADC} matrix as shown in
figure \ref{figure:fano_matsort}, where the subspace of the initial state
is denoted by $\mathbf{M}$, the final state subspace by $\mathbf{N}$ and
the interaction coupling those two subsets is named $\mathbf{W}$.

\begin{figure}[h]
  \centering
  \begin{tikzpicture}[scale=1.0]
    \footnotesize
    %\draw [help lines] (-1,0) grid (23,5);
    \draw [thick] (0,0) rectangle (3,3);
    \draw [thick] (0,2) rectangle (1,3);
    \draw [thick] (1,0) rectangle (3,2);
    \node (1h1) at (0.5,3.5) {1h};
    \node (1h2) at (-0.5,2.5) {1h};
    \node (2h1) at (2.0,3.5) {2h1p};
    \node (2h2) at (-0.5,1.0) {2h1p};

    \draw[->, very thick] (3.5,1.5) -- node[above] {sorting} (5.5,1.5);

  \begin{scope}[xshift=7cm]
    \draw [thick] (0,0) rectangle (3,3);
    \draw [thick] (0,2) rectangle (1,3);
    \draw [thick] (1,0) rectangle (3,2);
    \node (1h1) at (0.5,3.5) {1h};
    \node (1h2) at (-0.5,2.5) {1h};
    \node (2h1) at (2.0,3.5) {2h1p};
    \node (2h2) at (-0.5,1.0) {2h1p};
    \filldraw [orange,opacity=0.7] (0,1) rectangle (2,3);
    \filldraw [diplom2,opacity=0.5] (2,0) rectangle (3,1);
    \node at (1,2) {\textcolor{white}{$\mathbf{M}$}};
    \node at (2.5,0.5) {\textcolor{white}{$\mathbf{N}$}};
    \node at (1,0.5) {$\mathbf{W}^T$};
    \node at (2.5,2.0) {$\mathbf{W}$};
  \end{scope}  
\end{tikzpicture}

  \caption{}
  \label{figure:fano_matsort}
\end{figure}

The separate diagonalization of the initial and final state subspaces
$\mathbf{M}$ and $\mathbf{N}$ yield
the corresponding eigenvectors and eigenvalues on the diagonal of the
matrices $\mathbf{\Lambda}$ and $\mathbf{\Omega}$

\begin{align}
  \mathbf{\Lambda} &= \mathbf{I}^T \mathbf{M} \mathbf{I}  \\
  \mathbf{\Omega}  &= \mathbf{F}^T \mathbf{N} \mathbf{F} 
\end{align}

and the Hamiltonian is represented in the basis of their eigenstates
as illustrated in
figure \ref{figure:fano_bastrans} with $\mathbf{V} = \mathbf{I}^T \mathbf{W} \mathbf{F}$
being the interaction part
in this new basis.

\begin{figure}[h]
  \centering
     \begin{tikzpicture}[scale=0.7,>=stealth]
       \footnotesize
%       \draw [help lines] (-1,0) grid (23,5);
       \draw [thick] (0,0) rectangle (3,3);
       \draw [thick] (0,1) rectangle (2,3);
       \draw [thick] (2,0) rectangle (3,1);
       \node at (1,2) {$\mathbf{I}^T$};
       \node at (2.5,0.5) {$\mathbf{F}^T$};
       \node at (1,0.5) {$\mathbf{0}$};
       \node at (2.5,2.0) {$\mathbf{0}$};

     \begin{scope}[xshift=4cm]
       \draw [thick] (0,0) rectangle (3,3);
       \draw [thick] (0,2) rectangle (1,3);
       \draw [thick] (1,0) rectangle (3,2);
       \node (1h1) at (0.5,3.5) {1h};
       \node (1h2) at (-0.5,2.5) {1h};
       \node (2h1) at (2.0,3.5) {2h1p};
       \node (2h2) at (-0.5,1.0) {2h1p};
       \filldraw [orange,opacity=0.7] (0,1) rectangle (2,3);
       \filldraw [diplom2,opacity=0.5] (2,0) rectangle (3,1);
       \node at (1,2) {\textcolor{white}{$\mathbf{M}$}};
       \node at (2.5,0.5) {\textcolor{white}{$\mathbf{N}$}};
       \node at (1,0.5) {$\mathbf{W}^T$};
       \node at (2.5,2.0) {$\mathbf{W}$};
     \end{scope}  
     
     \begin{scope}[xshift=8cm]
       \draw [thick] (0,0) rectangle (3,3);
       \draw [thick] (0,1) rectangle (2,3);
       \draw [thick] (2,0) rectangle (3,1);
       \node at (1,2) {$\mathbf{I}$};
       \node at (2.5,0.5) {$\mathbf{F}$};
       \node at (1,0.5) {$\mathbf{0}$};
       \node at (2.5,2.0) {$\mathbf{0}$};
     \end{scope}

       \draw [->,very thick] (11.5,1.5) -- (12.5,1.5);

     \begin{scope}[xshift=13cm]
       \draw [thick] (0,0) rectangle (3,3);
       \draw [] (0,1) rectangle (2,3);
       \draw [] (2,0) rectangle (3,1);
       \draw [diplom1,opacity=0.5,ultra thick] (2,1.5) -- (3,1.5);
       \draw [diplom2,opacity=0.5,ultra thick] (2,1) -- (3,0);
       \node at (1,2) {\textcolor{diplom1}{${\mathbf{\Lambda}}$}};
       \node at (2.5,0.5) {\textcolor{diplom2}{$\mathbf{{\Omega}}$}};
       \node at (1,0.5) {$\mathbf{V}^T $};
       \node at (2.5,2.0) {$\mathbf{V}$};

       \node (moments) at (6.2,1.5) []
              {$\textcolor{diplom1}{\mathbf{V}_{i,q}}=
              \braket{\phi_i|\hat{H}|\psi_{q}^{2h1p}}$};
       \node (energy) at (5.4,0.5) []
              {$\textcolor{diplom2}{\bar{\omega}_{q}} =  E_q^{2h1p}$};
       \draw [->,ultra thick,gray] (3.2,1.5) -- (moments);
       \draw [->,ultra thick,gray] (3.2,0.5) -- (energy);
     \end{scope}

   \end{tikzpicture}

  \caption{}
  \label{figure:fano_bastrans}
\end{figure}
The desired pseudo-spectrum is then obtained from the final state energies
$\bar{\omega}_\beta$ and the interaction part in the basis of the initial and
final eigenstates for a specific initial state choice $i$ and possibly a choice
of channel $\beta$
$V_{i,\beta} = \braket{\phi |\hat{H}| \psi_{\beta, 2h1p}}$.



\subsection{Partitioning by Energy}
A partitioning by energy requires first a calculation of the double ionization
spectrum with the same conditions as regarded for the later decay width
calculation. From the results, the number of open channel contributions
for each symmetry are determined manually. Then, the satellite block
of the \ac{ADC} matrix is constructed such, that for each particle a
small $2h1p$ block is to be found on the diagonal. These blocks are diagonalized.
It is assumed, that the resulting order of the $2h$ configurations is the
same for all particles and equals the one from the former double ionization
potential calculation. These \textit{precorrelated} final states are then
partitioned with respect to energy into the the initial or the final state
subspace. In practice, rather the coefficients which allow for the construction
of the precorrelated final states from the uncorrelated matrix elements are stored.
Finally the \ac{ADC} matrix is constructed using the precorrelated final states.

From this point on, the calculation is the same as in the case of the partitioning
based on population.


\section{How to Obtain Trustworthy Results}
\subsection{Choice of Basis Set and Active Space}
A good basis set for a decay width calculation has to both
describe the initial state of the process reasonably well and to mimic
the outgoing secondary electron. The first requires a correlated basis set, since
the initial state is ionized in the inner- or subvalence region.
The latter requires the basis set to be:


\begin{enumerate}
 \item of high density
 \item including diffuse functions
 \item uncontracted
\end{enumerate}
Since in the Stieltjes approach the Chebyshev polynomials are constructed from
the pseudo-spectrum, the denser this pseudo-spectrum is, the better can the
description of the decay width be expected to be. A Stieltjes calculation starting
from a pseudo-spectrum with less than four points with a significant contribution
in the energy range of interest can not be expected to give reasonable results.

\begin{figure}[ht]
  \centering
  \begin{tikzpicture}[font=\footnotesize]

\begin{axis}[
             scale=1.0,
             ytick={0,0.05,...,0.15},
             yticklabels={0,0.05,...,0.15},
             xlabel={Energy in a.u.},
             ylabel=$\Gamma$ in a.u.,
             title={Comparison of Contracted and Uncontracted cc-pVQZ}
%             domain=1.72:1.76,
%             y domain=1.16:1.20
             ]

	
\addplot[color=diplom1,
         mark=*,
         ycomb
        ]
        table[
        x expr=\thisrowno{0} - 31.88,
        y expr=\thisrowno{1}*2*3.14159265359
        ] {data/uncontract/cc-pVQZ};
        \addlegendentry{uncontracted}
\addplot[color=black,
         mark=*,
         ycomb
        ]
        table[
        x expr=\thisrowno{0} -31.88,
        y expr=\thisrowno{1}*2*3.14159265359
        ] {data/contract/cc-pVQZ};
        \addlegendentry{contracted}

\end{axis}
\end{tikzpicture}

  \caption{Shifted pseudo-spectra for different basis sets with $E_r=0$
           for the Auger process following an ionization of the neon 1s.
           Comparison of contracted and uncontracted cc-pVQZ.}
  \label{figure:bases_pseudospectra}
\end{figure}

A contraction of the basis set reduces its flexibility and optimizes the
description of the electron cloud close to the nuclei. Since in the decay width
calculation one aims to mimic the delocalized outgoing electron by
$\mathcal{L}^2$ functions, a contraction leads to a very narrow pseudo-spectrum
in energy regions which are of no interest and is therefore counterproductive.

\begin{figure}[ht]
  \centering
  \begin{tikzpicture}[font=\footnotesize]

\begin{axis}[
             scale=1.0,
             %ytick={0,0.05,...,0.15},
             %yticklabels={0,0.05,...,0.15},
             xlabel={Energy in a.u.},
             ylabel=$\Gamma$ in a.u.,
             title={Pseudo-Spectra of Different Basis Set Groups}
%             domain=1.72:1.76,
%             y domain=1.16:1.20
             ]

	
\addplot[color=diplom2,
         mark=*,
         ycomb
        ]
        table[
        x expr=\thisrowno{0},
        y expr=\thisrowno{1}*2*3.14159265359
        ] {data/uncontract/cc-pCV6Z};
        \addlegendentry{cc-pCV6Z}
\addplot[color=diplom1,
         mark=*,
         ycomb
        ]
        table[
        x expr=\thisrowno{0} -31.88,
        y expr=\thisrowno{1}*2*3.14159265359
        ] {data/uncontract/aug-cc-pV6Z};
        \addlegendentry{aug-cc-pV6Z}
\addplot[color=black,
         mark=*,
         ycomb
        ]
        table[
        x expr=\thisrowno{0} - 31.88,
        y expr=\thisrowno{1}*2*3.14159265359
        ] {data/uncontract/cc-pV6Z};
        \addlegendentry{cc-pV6Z}

\end{axis}
\end{tikzpicture}

  \caption{Shifted pseudo-spectra for different basis sets with $E_r=0$
           for the Auger process following an ionization of the neon 1s.
           Different basis set classes.}
  \label{figure:bases_pseudospectra}
\end{figure}

An example is given in figure \ref{figure:bases_pseudospectra} for some selected
basis sets for the Auger following an ionization of the Ne1s. The resonance energy
about \unit[32]{a.u.}. It can easily be seen, that the smaller double and
quadruple zeta basis sets do not cover the range of the resonance energy in case
of the contracted basis sets. Additionally it can be seen that the pseudo-spectrum
of the uncontracted basis is much broader and more dense close to the resonance
energy. The contracted cc-pV6Z basis set covers the energy range quite well, which
is not to be expected in general.

\begin{figure}[ht]
  \centering
  \begin{tikzpicture}[font=\footnotesize]

\begin{axis}[view=
             scale=1.0,
             ytick={0,0.05,...,0.15},
             yticklabels={0,0.05,...,0.15},
             xlabel={Energy in a.u.},
             ylabel=$\Gamma$ in a.u.,
             title={With KBJ Exponents}
%             domain=1.72:1.76,
%             y domain=1.16:1.20
             ]

	
\addplot[color=red,
         mark=*,
         ycomb
        ]
        table[
        x expr=\thisrowno{0},
        y expr=\thisrowno{1}*2*3.14159265359
        ] {data/uncontract/cc-pCV6Z_jungen_s};
        \addlegendentry{jungen\_s}
\addplot[color=orange,
         mark=*,
         ycomb
        ]
        table[
        x expr=\thisrowno{0},
        y expr=\thisrowno{1}*2*3.14159265359
        ] {data/uncontract/cc-pCV6Z_jungen_p};
        \addlegendentry{jungen\_p}
\addplot[color=orange!50,
         mark=*,
         ycomb
        ]
        table[
        x expr=\thisrowno{0},
        y expr=\thisrowno{1}*2*3.14159265359
        ] {data/uncontract/cc-pCV6Z_jungen_d};
        \addlegendentry{jungen\_d}
\addplot[color=blue!50,
         mark=*,
         ycomb
        ]
        table[
        x expr=\thisrowno{0},
        y expr=\thisrowno{1}*2*3.14159265359
        ] {data/uncontract/cc-pCV6Z_jungen_spd};
        \addlegendentry{jungen\_spd}
\addplot[mark=*,
         color=black,
         ycomb
        ]
        table[
        x expr=\thisrowno{0},
        y expr=\thisrowno{1}*2*3.14159265359
        ] {data/uncontract/cc-pCV6Z};
        \addlegendentry{cc-pCV6Z}

\end{axis}
\end{tikzpicture}

  \caption{Pseudo-spectra for the Auger process following the ionization
           out of the Ne1s for the aug-cc-pV6Z basis set with additional
           \ac{KBJ} basis functions of the s, p and d type.}
  \label{figure:bases_jungen}
\end{figure}

In order to mimic the outgoing electron with $\mathcal{L}^2$ functions, the
exponents of the basis set have to include the energy range of the expected
kinetic energy of the electron. Preferably, the basis set is very dense in
this energy region. Additionally, and especially for processes close to threshold
with slow
secondary electrons, the basis set should contain many diffuse and hence
delocalized functions. They allow to cover the spacial range of the interaction
region and hence states with the electron being disrupted from the nuclei.
For this purpose additional basis functions of the \ac{KBJ}\cite{Kaufmann89}
type are used at the atomic positions or in case of molecules on ghost atoms
positioned between the atoms. Their effect on the density of the pseudo-spectrum
is illustrated in figure \ref{figure:bases_jungen}. There, the contributions of the
different anuglar momentum functions are shown. Each class adds some significant
points to the pseudo-spectrum and hence, in combination, a good and dense
pseudo-spectrum is created.

Additionally the active space needs to include the orbitals of the initial state
shell as well as a reasonably large number of positive energy solutions. These
positive energy solutions describe the outgoing electron an hence the kinetic energy
of the electron has to be included. Furthermore, since the Stieltjes procedure is
based on an interpolation, energies higher than the kinetic energy of the outgoing
electron are as well needed. From experience, the positive energy of the active
space should at least cover the kinetic energy of the outgoing electron plus
\unit[10]{a.u.}.



\subsection{Quality Check Using the Decay Width Profile}
Errors and instabilities of the FanoADC-Stieltjes calculation can be observed
in the energy dependence of the calculated decay width $\Gamma(E)$.
As already shown in figure \ref{figure:stieltjes_density}, the optimum
interpolation result is a smooth, monotonically decreasing function.
The decay width for the given resonance energy $\Gamma(E_r)$ can in this case easy be
evaluated.

The error of the calculated decay width is estimated by the decay widths
close to the resonance energy.
There is no rigorous definition
of how exactly to choose this energy range in the literature. 
In general it is estimated
by the quality of the initial state description in the initial
state subspace compared to the one of the full Hamiltonian and the
manual inspection of the density of points obtained
from the FanoADC-Stieltjes calculation. If
the obtained density function has a low curvature at the resonance energy,
the error of the decay width is normally very small, while for resonance
energies, where the decay width has a high curvature, the error can be of the same
order as the obtained decay width.

In addition to numerical instabilities introduced by the Stieltjes procedure
discussed in section \ref{},
two further characteristic pattern might be observed in a calculated
decay width: channel openings and interactions of the initial state
with Rydberg states.


\begin{figure}[h]
  \centering
  \begin{tikzpicture}[scale=1.0]

\begin{axis}[%scale=0.8,
             domain=0.0:5.5,
             samples = 200,
             %ticks=none,
             xtick={3,7},
             xticklabels={$E_{r,1}$,$E_{r,2}$},
             ytick=empty,
             %legend style={anchor= north west}
             xlabel={$E$},
             ylabel={$\Gamma(E)$},
             legend pos = north west,
             ]

  \node[anchor=west] at (axis cs:-0.7,4.3) {\small channel opening};
  \addplot+[mark = none,
           thick,
           diplom1,
           domain=0:2
           ]
           {1.5/(x+0.7)^2 + 0.3};
  \addplot+[mark = none,
           thick,
           diplom1,
           domain=2:4
           ]
           {1.5/(x+0.7)^2 + 0.3 + 1.5/(x-1.3)^2 + 0.3};
  \addplot+[mark = none,
           thick,
           diplom1,
           domain=4:8
           ]
           {1.5/(x+0.7)^2 + 0.3 + 1.5/(x-1.3)^2 + 0.3 + 1.5/(x-3.3)^2 + 0.3};
  \draw[diplom1,thick] (axis cs:2,0.5058) --
                 (axis cs:2,3.8670);
  \draw[diplom1,thick] (axis cs:4,0.8737) --
                 (axis cs:4,4.2349);

  \draw[dashed] (axis cs:3,\pgfkeysvalueof{/pgfplots/ymin}) --
                 (axis cs:3,\pgfkeysvalueof{/pgfplots/ymax});
%  \node[pin={[pin distance=0.2cm]180:{\small $E_{r,1}$}}]
%     at (axis cs:3,4.3) {};

  \draw[dashed] (axis cs:7,\pgfkeysvalueof{/pgfplots/ymin}) --
                 (axis cs:7,\pgfkeysvalueof{/pgfplots/ymax});
%  \node[pin={[pin distance=0.2cm]180:{\small $E_{r,2}$}}]
%     at (axis cs:7,4.3) {};
\end{axis}
\end{tikzpicture}

  \caption{Schematic illustration of a decay width profile with three
           channels. At the resonant energy $E_{r,1}$, the first two channels
           are open and at the resonance energy $E_{r,2}$ all three channels
           are open.}
  \label{figure:prof_channel_opening}
\end{figure}
In case of channel openings, at low energies the decay width is decreasing, but
at a certain energy another channel opens and the decay width of both are
added as shown in figure \ref{figure:prof_channel_opening}. In case of the
resonance energy being higher than all the channel opening energies as for
$E_{r,2}$, the
selection of the final state subspace can be expected to be correct
and the initial
and final state description of the partitioned Hamiltonian to be reasonable.

In case of the resonance energy being lower than the highest channel
opening energy like for $E_{r,1}$ a closed channel is treated as a final
state. If such a result is obtained, one should carefully check the selection
of the final state configurations. However, it is still possible that the
partitioning of the Hamiltonian yields such descriptions of the initial and
final states that the channel opening and closing is not reproduced in the
decay width. In these cases, the results are not trustworthy.

Rydberg states are bound states of the systems and hence do not contribute
to the decay. However, an interaction of the initial state with those
Rydberg states is possible and leads to a density function as shown in
figure \ref{figure:prof_rydberg}.

\begin{figure}[h]
  \centering
  \begin{tikzpicture}[scale=1.0]

\begin{axis}[%scale=0.8,
             domain=0.0:5.5,
             samples = 200,
             ticks=none,
             %xtick={3,7},
             %xticklabels={$E_{r,1}$,$E_{r,2}$},
             %ytick=empty,
             %legend style={anchor= north west}
             xlabel={$E$},
             ylabel={$\Gamma(E)$},
             legend pos = north west,
             ]

  \addplot+[mark = none,
           thick,
           diplom2,
           domain=1:3
           ]
           {1.5/(x+0.7)^2 + 0.3 + 1.7E-2/ ((x-2)^2 + 0.1^2)};
  \addplot+[mark = none,
           thick,
           diplom2,
           domain=3:5
           ]
           {1.5/(x+0.7)^2 + 0.3 + 1.4E-2/ ((x-4)^2 + 0.1^2)};
  \addplot+[mark = none,
           thick,
           diplom1,
           domain=0:6
           ]
           {1.5/(x+0.7)^2 + 0.3};

  \node[anchor=west,fill=white] at (axis cs:-0.4,3.4) {\small Rydberg states};

\end{axis}
\end{tikzpicture}

  \caption{Schematic illustration of a decay width profile with couplings
           of the initial state to Rydberg states.}
  \label{figure:prof_rydberg}
\end{figure}
Here the density function is in principle a monotonically decreasing function,
but has additional peaks at certain energies. They can be distinguished from
channel openings by observing the behaviour after the peak. In case of
interactions with Rydberg states, the decay width is the same as if the interaction
would not take place, while in case of a channel opening, the decay widths
of the two channels are added to give a total decay width.
If the resonance energy coincides with the position of a Rydberg interaction,
the Rydberg interaction is not to be taken into account in the determination
of the decay width.

In a real calculation, these two features are more diffuse than in the
illustrations above in
figures \ref{figure:prof_channel_opening}
and \ref{figure:prof_rydberg}, because the Stieltjes procedure smoothes
discontinuities as observed in the case of channel openings and sharp peaks
as for the Rydberg states.
In the Stieltjes procedure such density functions are more accurately described
with higher order of Stieltjes being based on higher order polynomials
with more degrees of freedom.
An observed phenomenon related to this is
a splitting of the decay width described by either low or high orders of Stieltjes.
In these cases, one has to decide based on the knowledge of all other
parameters, which of these two curves might be more trustworthy.

Close to threshold, the determination of the decay width is difficult, since
the resonance energies are approximations and not exact. Therefore, the
energy range spans a wide distribution of different decay widths. Furthermore
the broadened decay width being based on points obtained from the Stieltjes
procedure might not cover the maximum of the peak and hence the theoretical
maximum of the decay width.
