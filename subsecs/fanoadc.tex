\section{FanoADC}
In the FanoADC, we use the discrete ADC Hamiltonian in the \ac{ISR}
basis for the construction of the pseudo-spectrum of the decay width
$\Gamma$.
Therefore
Due to multiple possible final states, we employ the procedure
of Feshbach using projection operators for the partitioning of the Hamiltonian
into an initial and a final state subspace.
The processes of our interest have a singly ionized initial state and a
doubly charged final state with an additional electron in the continuum.
Therefore, the final state configurations are chosen from
the class of $2h1p$ states, where the $2h$ part is assumed to
describe the doubly ionized final state and the particle maps the continuum electron.
The initial state in a zeroth order approximation can be described as a $1h$ state.

In the actual partitioning, all $2h1p$ with the $2h$ part corresponding to
a final state configuration are chosen to be part of the final states.
All $1h$ configurations and those $2h1p$ configurations not corresponding
to a possible final state configuration are used for a better description
of the initial state. This leads to a resorted \ac{ADC} matrix as shown in
figure \ref{figure:fano_matsort}, where the subspace of the initial state
is denoted by $\mathbf{M}$, the final state subspace by $\mathbf{N}$ and
the interaction coupling those two subsets is named $\mathbf{W}$.

\begin{figure}[h]
  \centering
  \begin{tikzpicture}[scale=1.0]
    \footnotesize
    %\draw [help lines] (-1,0) grid (23,5);
    \draw [thick] (0,0) rectangle (3,3);
    \draw [thick] (0,2) rectangle (1,3);
    \draw [thick] (1,0) rectangle (3,2);
    \node (1h1) at (0.5,3.5) {1h};
    \node (1h2) at (-0.5,2.5) {1h};
    \node (2h1) at (2.0,3.5) {2h1p};
    \node (2h2) at (-0.5,1.0) {2h1p};

    \draw[->, very thick] (3.5,1.5) -- node[above] {sorting} (5.5,1.5);

  \begin{scope}[xshift=7cm]
    \draw [thick] (0,0) rectangle (3,3);
    \draw [thick] (0,2) rectangle (1,3);
    \draw [thick] (1,0) rectangle (3,2);
    \node (1h1) at (0.5,3.5) {1h};
    \node (1h2) at (-0.5,2.5) {1h};
    \node (2h1) at (2.0,3.5) {2h1p};
    \node (2h2) at (-0.5,1.0) {2h1p};
    \filldraw [orange,opacity=0.7] (0,1) rectangle (2,3);
    \filldraw [diplom2,opacity=0.5] (2,0) rectangle (3,1);
    \node at (1,2) {\textcolor{white}{$\mathbf{M}$}};
    \node at (2.5,0.5) {\textcolor{white}{$\mathbf{N}$}};
    \node at (1,0.5) {$\mathbf{W}^T$};
    \node at (2.5,2.0) {$\mathbf{W}$};
  \end{scope}  
\end{tikzpicture}

  \caption{}
  \label{figure:fano_matsort}
\end{figure}

The separate diagonalization of the initial and final state subspaces
$\mathbf{M}$ and $\mathbf{N}$ yield
the corresponding eigenvectors and eigenvalues on the diagonal of the
matrices $\mathbf{\Lambda}$ and $\mathbf{\Omega}$

\begin{align}
  \mathbf{\Lambda} &= \mathbf{I}^T \mathbf{M} \mathbf{I}  \\
  \mathbf{\Omega}  &= \mathbf{F}^T \mathbf{N} \mathbf{F} 
\end{align}

and the Hamiltonian is represented in the basis of their eigenstates
as illustrated in
figure \ref{figure:fano_bastrans} with $\mathbf{V} = \mathbf{I}^T \mathbf{W} \mathbf{F}$
being the interaction part
in this new basis.

\begin{figure}[h]
  \centering
     \begin{tikzpicture}[scale=0.7,>=stealth]
       \footnotesize
%       \draw [help lines] (-1,0) grid (23,5);
       \draw [thick] (0,0) rectangle (3,3);
       \draw [thick] (0,1) rectangle (2,3);
       \draw [thick] (2,0) rectangle (3,1);
       \node at (1,2) {$\mathbf{I}^T$};
       \node at (2.5,0.5) {$\mathbf{F}^T$};
       \node at (1,0.5) {$\mathbf{0}$};
       \node at (2.5,2.0) {$\mathbf{0}$};

     \begin{scope}[xshift=4cm]
       \draw [thick] (0,0) rectangle (3,3);
       \draw [thick] (0,2) rectangle (1,3);
       \draw [thick] (1,0) rectangle (3,2);
       \node (1h1) at (0.5,3.5) {1h};
       \node (1h2) at (-0.5,2.5) {1h};
       \node (2h1) at (2.0,3.5) {2h1p};
       \node (2h2) at (-0.5,1.0) {2h1p};
       \filldraw [diplom3,opacity=0.5] (0,1) rectangle (2,3);
       \filldraw [diplom2,opacity=0.5] (2,0) rectangle (3,1);
       \node at (1,2) {\textcolor{white}{$\mathbf{M}$}};
       \node at (2.5,0.5) {\textcolor{white}{$\mathbf{N}$}};
       \node at (1,0.5) {$\mathbf{W}^T$};
       \node at (2.5,2.0) {$\mathbf{W}$};
     \end{scope}  
     
     \begin{scope}[xshift=8cm]
       \draw [thick] (0,0) rectangle (3,3);
       \draw [thick] (0,1) rectangle (2,3);
       \draw [thick] (2,0) rectangle (3,1);
       \node at (1,2) {$\mathbf{I}$};
       \node at (2.5,0.5) {$\mathbf{F}$};
       \node at (1,0.5) {$\mathbf{0}$};
       \node at (2.5,2.0) {$\mathbf{0}$};
     \end{scope}

       \draw [->,very thick] (11.5,1.5) -- (12.5,1.5);

     \begin{scope}[xshift=13cm]
       \draw [thick] (0,0) rectangle (3,3);
       \draw [] (0,1) rectangle (2,3);
       \draw [] (2,0) rectangle (3,1);
       \draw [diplom1,opacity=0.5,ultra thick] (2,1.5) -- (3,1.5);
       \draw [diplom2,opacity=0.5,ultra thick] (2,1) -- (3,0);
       \node at (1,2) {\textcolor{diplom1}{${\Lambda}$}};
       \node at (2.5,0.5) {\textcolor{diplom2}{${\Omega}$}};
       \node at (1,0.5) {$\mathbf{V}^T $};
       \node at (2.5,2.0) {$\mathbf{V}$};

       \node (moments) at (6.2,1.5) []
              {$\textcolor{diplom1}{\mathbf{V}_{i,\beta}}=
              \braket{\phi|\hat{H}|\psi_{\beta,2h1p}}$};
       \node (energy) at (5.4,0.5) []
              {$\textcolor{diplom2}{\bar{\omega}_{\beta}} =  E_{2h1p}$};
       \draw [->,ultra thick,gray] (3.2,1.5) -- (moments);
       \draw [->,ultra thick,gray] (3.2,0.5) -- (energy);
     \end{scope}

   \end{tikzpicture}

  \caption{}
  \label{figure:fano_bastrans}
\end{figure}
The desired pseudo-spectrum is then obtained from the final state energies
$\bar{\omega}_\beta$ and the interaction part in the basis of the initial and
final eigenstates for a specific initial state choice $i$ and possibly a choice
of channel $\beta$
$V_{i,\beta} = \braket{\phi |\hat{H}| \psi_{\beta, 2h1p}}$.

