\section{HARDRoC --- Hunting Asymptotic Relativistic Decay Rates of Clusters}

The program calculates the \ac{ICD} and \ac{ETMD}3 decay widths
of given cluster structures. Thereby, it is possible to choose between
the evaluation of the asymptotic expressions of equations
(\ref{reltheolifetime_exp}) and (\ref{reltheolifetimeetmd_exp}) using
atomic data and the evaluation of a function obtained from fitting
\emph{ab initio} results to $\Gamma(R) = c_{pre} e^{-c_{exp}} + \frac{c_6}{R^6}$
for the ICD.

The required input information is:
\begin{itemize}
 \item xyz-file
 \item desired decay process(es)
 \item evaluation type
 \item ICD based in atomic properties:
       \begin{itemize}
        \item \ac{SIP}s of the initial and final state atoms for each channel and
              eventual energetic shifts caused by the cluster environment
        \item radiative lifetime $\tau$ of the initially ionized state
        \item optionally, the ratio $\chi=\frac{\tau_{1/2}}{\tau_{3/2}}$
        \item ratio of ionization cross sections
       \end{itemize}
 \item ICD based on \emph{ab initio} results:
       \begin{itemize}
        \item \ac{SIP}s of the initial and final state atoms for each channel and
              eventual energetic shifts caused by the cluster environment
        \item parameters of the function fitted to the decay width $c_pre$,
              $c_{exp}$ and $c_6$
       \end{itemize}
 \item ETMD: 
       \begin{itemize}
        \item \ac{SIP}s of the initial and final state atoms for each channel and
              eventual energetic shifts caused by the cluster environment
        \item ratio of ionization cross sections
       \end{itemize}
\end{itemize}

Additionally the total ionization cross section is required. Since it is
energy dependent, the experimental data of references \cite{}
parameters of the function
$\sigma(\omega) = a\omega^2 + b\omega + c + \frac{d}{\omega}$ were fitted
to the experimental data of references \cite{West76,West78}
for the noble gases and incorporated in the program. This function is evaluated
when needed.

The transition dipole moments required for the ETMD3 calculation were
calculated with Dirac using a zyx basis set. Since, they depend on the
internuclear distance, they were fitted to a function and the coresponding
parameters were incoprporated in the program and are evaluated for the
different distances.

The algorithm proceeds in the following way:
\begin{enumerate}
 \item decompose the cluster structure into pairs or triples depending
       on the required decay process
 \item create statistics of equivalent pair or triple geometries
 \item for each decay channel and for each non-equivalent geometry
       determine energetic accessibility of the decay channel
 \item evaluate the corresponding decay width in case of energetic accessibility
 \item normalize the decay width to one initially ionized atom and multiply
       it by the occurence of the pair or triple geometry of the cluster
 \item for each channel write the geometry parameters, secondary electron energies
       and the decay widths to an output file
 \item for each channel sum over the obtained decay widths
 \item sum the decay widths of the different channels to give the total decay
       width of all channels
\end{enumerate}

The computationally most expensive step in the above procedure is the
sorting procedure in order to obtain a shorter array consisting of the
non-equivalent geometries and the corresponding weights. It scales like
$\frac{N(N_{neq}-1)}{2}$ with $N$ being the total number of pairs or triples
and $N_{neq}$ being the number of non-equivalent pairs or triples of the cluster.
In the worst case $N=N_{neq}$.
For the \ac{ICD}, $N=N_{in}\cdot N_{fin}$, where $N_{in}$
denotes the number of atoms of the atomtype, which is initially ionized
and $N_{fin}$ denotes the number of potential decay partners. For the
\ac{ETMD}3, $N=N_{in} \cdot N_{fin1} \cdot N_{fin2}$, where $N_{fin1,2}$
denote the number of atoms of the atom type of electron donor and electron
emitter. For clusters consisting of several hundrets or thousands of atoms, these
numbers are huge, especially for the ETMD. Since the transition dipole moments
in equation (\ref{reltheolifetimeetmd_exp}) decrease exponentially with the
internuclear distance, triples with large $Q$ have a negligible contribution
to the total decay width. Therefore, an additional parameter for the maximum
allowed electron transfer distance $Q$ can be introduced,
which reduces the number of triples before the sorting dramatically for large
clusters.

Despite its computational cost, this sorting step is highly regarded for
all following steps.
Since mostly, highly symmetric structures are investigated, the length of
the array and hence the number of operations can be decreased up to a
factor of about 50. Also the written output is shortened by the same factor
and has an improved readability.

The manual is to be found in reference \cite{HARDRoC}.
