\subsection{Moment Problem}

The moments $S(k)$ of a real and continuous function $f(\omega)$ are defined
as

\begin{equation}
  S(k) = \int\limits_a^b \omega^k f(\omega) d\omega \quad\quad k=0,1,\dots  .
\end{equation}

In case of $f(\omega)$ being a probability density function, it is connected
to the probability distribution function $F(\omega)$ via
\begin{equation}
  F(\omega) = f(\omega){d\omega} .
\end{equation}

The probability density function is completely determined by the manifold
of moments. Therefore, when all moments are known, the probability density
function can be calculated from the moments. In the present case $f(\omega)$
is the decay width $\Gamma(E)$, but the theory is also applicable and very often
used for the description of cross sections.

In practice, all moments are never available unless the moments can be
calculated analytically. Therefore, one has to approximately solve the reduced
moment problem, since the density function is not completely defined.
In this case the $2r$ moments are

\begin{equation}
  S(k) = \int\limits_a^b \omega^k f(\omega) d\omega \quad\quad k=0,1,...,2r-1
\end{equation}





\subsection{Gaussian Quadrature}

\begin{figure}[h]
  \centering
  \begin{tikzpicture}
    \begin{axis}[%scale=0.8,
                 domain=-1.0:1.0,
                 samples = 200,
                 %xtick={-3.14159,-1.57089,...,3.14159},
                 %xticklabels={$-\pi$,$-\frac \pi 2$,0,$\frac \pi 2$,$\pi$},
                 cycle list name = exotic,
                 legend style={anchor= north west},
                 legend cell align = left
                 ]
     \addplot+[domain=-1:-0.965925826289+0.135517335117/2,
              diplom1,
              mark = none,
              %forget plot,
              pattern = north east lines,
              pattern color = diplom1
              ]
              {0.0987789349866} \closedcycle;
     \addlegendentry{approximation of the integral}
     \addplot+[domain=-0.965925826289+0.135517335117/2:-0.707106781187+0.370240244847/2,
              diplom1,
              mark = none,
              forget plot,
              pattern = north east lines,
              pattern color = diplom1
              ]
              {0.646446609407} \closedcycle;
     \addplot+[domain=-0.707106781187+0.370240244847/2:-0.258819045103+0.505757579964/2,
              diplom1,
              mark = none,
              forget plot,
              pattern = north east lines,
              pattern color = diplom1
              ]
              {0.982662411472} \closedcycle;
     \addplot+[domain=-0.258819045103+0.505757579964/2:0.258819045103+0.505757579964/2,
              diplom1,
              mark = none,
              forget plot,
              pattern = north east lines,
              pattern color = diplom1
              ]
              {1.017337588535} \closedcycle;
     \addplot+[domain=0.258819045103+0.505757579964/2:0.708718898022+0.370240244847/2,
              diplom1,
              mark = none,
              forget plot,
              pattern = north east lines,
              pattern color = diplom1
              ]
              {1.353553390592} \closedcycle;
     \addplot+[domain=0.708718898022+0.370240244847/2:1.0,
              diplom1,
              mark = none,
              forget plot,
              pattern = north east lines,
              pattern color = diplom1
              ]
              {1.901221065013} \closedcycle;
     \addplot [diplom2, thick]
              {x^3 + 1};
     \addlegendentry{$f(x)= x^3 + 1$}
     %\addplot [diplom3, thick]
     %         {x^4/4 + x};
     \addplot [only marks,mark=o,thick]
       coordinates {
                   ( 0.965925826289, 1.901221065013 )
                   ( 0.707106781187, 1.353553390592 )
                   ( 0.258819045103, 1.017337588535 )
                   (-0.258819045103, 0.982662411472 )
                   (-0.707106781187, 0.646446609407 )
                   (-0.965925826289, 0.0987789349866)
                   };
     \addlegendentry{$f_i(x_i)$}
    \end{axis}
\end{tikzpicture}

  \caption{}
  \label{}
\end{figure}

\begin{equation}
a
\end{equation}



\subsection{Stieltjes Imaging}

\begin{equation}
  \int\limits_a^b Q_n(1/\omega) \, Q_m(1/\omega) f(\omega) d\omega = N_n \delta_{nm}
\end{equation}

\begin{equation}
  N_n = \int\limits_a^b \left[ Q_n(1/\omega) \right]^2 f(\omega) d\omega
\end{equation}

\begin{equation}
  Q_n(1/\omega) = \frac{1}{\omega - a_n} Q_{n-1}(1/\omega) - b_{n-1} Q_{n-2}(1/\omega)
\end{equation}

\begin{align}
  a_n     &= \frac{1}{b_0b_1\cdots b_{n-1}}
             \int (1/\omega)^n Q_{n-1}(1/\omega) f(\omega) d\omega
             - \sum\limits_{l=1}^{n-1} a_l \\
  b_{n-1} &= \frac{1}{b_0b_1\cdots b_{n-2}}
             \int (1/\omega)^{n-1} Q_{n-1}(1/\omega) f(\omega) d\omega
\end{align}

\begin{align}
  a_n     &= \frac{1}{b_0b_1\cdots b_{n-1}}
             \sum\limits_{i=1}^N
               (1/\bar{\omega}_i)^n Q_{n-1}(1/\bar{\omega_i}) \bar{f}_i
             - \sum\limits_{l=1}^{n-1} a_l \\
  b_{n-1} &= \frac{1}{b_0b_1\cdots b_{n-2}}
             \sum\limits_{i=1}^N
               (1/\bar{\omega}_i)^{n-1} Q_{n-1}(1/\bar{\omega}_i) \bar{f}_i
\end{align}

\begin{equation}
  Q_n(1/\bar{\omega}_i) = \frac{1}{\bar{\omega}_i - a_n} Q_{n-1}(1/\bar{\omega}_i)
                          - b_{n-1} Q_{n-2}(1/\bar{\omega}_i)
\end{equation}

\begin{equation}
  Q_0(1/\bar{\omega}_i) = 1 \quad\quad Q_1(1/\bar{\omega}_i) = (1/\bar{\omega}_i) - a_1
\end{equation}


As discussed above in section \ref{section:}, given the weights, the polynomials
can be calculated. In the Stieltjes imaging, starting from the polynomials
defined by the discrete pseudo-spectrum, one wants to achieve the weight function.
Also here the ideal abscissae are the roots of the highest order polynomial
for each moment $k$

\begin{equation}
  Q_n(1/\omega_i) = 0 \quad\quad i = 1,2,\dots ,n
\end{equation}



\begin{equation}
  f_i = \left[ \sum\limits_{m=0}^{n-1} \frac{Q_m^2(1/\omega_i)}{N_m} \right]^{-1}
\end{equation}


\begin{equation}
  Q_n(1/\omega) = (-1)^n \sqrt{N_n} R_n(1/\omega)
\end{equation}

\begin{equation}
  (1/\omega)R_{n-1}(1/\omega) = - \sqrt{b_n}R_n(1/\omega) + a_nR_{n-1}(1/\omega)
                                - \sqrt{b_{n-1}} R_{n-2}(1/\omega)
\end{equation}

\begin{equation}
  (1/\omega)R_0(1/\omega) = - \sqrt{b_1}R_1(1/\omega) + a_1 R_0(1/\omega)
\end{equation}

\begin{equation}
 \begin{split}
 \begin{pmatrix}
a_1        & -\sqrt{b_1}&            &                &             &          \\
-\sqrt{b_1}& a_2        & -\sqrt{b_2}&                &             &          \\
           & -\sqrt{b_2}& a_3        & -\sqrt{b_3}    &             &          \\
           &            & \ddots     & \ddots         & \ddots      &          \\
           &            &            & -\sqrt{b_{n-2}}& a_{n-1}     & -\sqrt{b_{n-1}}\\
           &            &            &                & -\sqrt{b_{n-1}}& a_n   
 \end{pmatrix}
 \begin{pmatrix}
  R_0(1/\omega)\\
  R_1(1/\omega)\\
  R_2(1/\omega)\\
  \vdots\\
  R_{n-2}(1/\omega)\\
  R_{n-1}(1/\omega)
 \end{pmatrix}         \\
 = (1/\omega)
 \begin{pmatrix}
  R_0(1/\omega)\\
  R_1(1/\omega)\\
  R_2(1/\omega)\\
  \vdots\\
  R_{n-2}(1/\omega)\\
  R_{n-1}(1/\omega)
 \end{pmatrix}
 -
 \begin{pmatrix}
  0\\
  0\\
  0\\
  \vdots\\
  0\\
  -\sqrt{b_n} R_{n}(1/\omega)
 \end{pmatrix}
 \end{split}
\end{equation}



\begin{equation}
  1 = f_i \sum\limits_{m=0}^{n-1} R_m^2 (1/\omega_i) = \mathbf{u_i} \cdot \mathbf{u_i}
\end{equation}

\begin{equation}
  f_i = N_0 u_{0i}^2
\end{equation}

\begin{equation}
  S(-k) = \sum\limits_{i=1}^n f_i (1/\omega_i)^k \quad\quad k=0,1,\dots,2n-1
\end{equation}


\begin{equation}
  F^{(n)}(\omega_i - 0) \le F^{(n+1)}(\omega_i - 0) \le F(\omega_i)
  \le F^{(n+1)}(\omega_i + 0) \le F^{(n)}(\omega_i + 0)
\end{equation}


\begin{figure}[h]
  \centering
  %NeAr at 3.42 AA, 6th order of stieltjes
\begin{tikzpicture}
    \begin{axis}[%scale=0.8,
                 domain=-1.0:1.0,
                 samples = 200,
                 %xtick={-3.14159,-1.57089,...,3.14159},
                 %xticklabels={$-\pi$,$-\frac \pi 2$,0,$\frac \pi 2$,$\pi$},
                 cycle list name = exotic,
                 legend style={anchor= north west},
                 legend cell align = left,
                 xlabel= {$E$ [a.u.]},
                 ylabel= {$F(E)$}
                 ]
     \addplot+[domain=-1:-0.42789448546292908,
              diplom1,
              mark = none,
%              forget plot,
              pattern = north east lines,
              pattern color = diplom1
              ]
              {0.0} \closedcycle;
     \addlegendentry{$\sum\limits_{j=1}^{i} f_j \quad \omega_i < \omega < \omega_{i+1}$}
     \addlegendimage{empty legend}
     \addlegendentry{}
     \addplot+[domain=-0.42789448546292908:-0.25969072398345394,
              diplom1,
              mark = none,
              forget plot,
              pattern = north east lines,
              pattern color = diplom1
              ]
              {0.000790314507333} \closedcycle;
     \addplot+[domain=-0.25969072398345394:0.11385089684626037,
              diplom1,
              mark = none,
              forget plot,
              pattern = north east lines,
              pattern color = diplom1
              ]
              {0.00156501308103} \closedcycle;
     \addplot+[domain=0.11385089684626037:1.4305018410002852,
              diplom1,
              mark = none,
              forget plot,
              pattern = north east lines,
              pattern color = diplom1
              ]
              {0.00208878898366} \closedcycle;
     \addplot+[domain=1.4305018410002852:5.1882229137192883,
              diplom1,
              mark = none,
              forget plot,
              pattern = north east lines,
              pattern color = diplom1
              ]
              {0.0022929712885} \closedcycle;
     \addplot+[domain=5.1882229137192883:6,
              diplom1,
              mark = none,
              forget plot,
              pattern = north east lines,
              pattern color = diplom1
              ]
              {0.00241086375443} \closedcycle;
%     \addplot [diplom2, thick,
%               domain=-0.5:6]
%              %{0.001300 * ln(x+2.594992)};
%              {0.001049 * sqrt(x+1.374436)};
%     \addlegendentry{$F(x)= \frac 13 x^3 + x^2 + 2x$}
     \addplot [samples=200,mark=*,thick,smooth,diplom2]
       coordinates {
                   (-0.42789448546292908, 0.0003951572536665)
                   (-0.25969072398345394, 0.0011776637941815)
                   ( 0.11385089684626037, 0.001826901032345)
                   ( 1.4305018410002852, 0.00219088013608)
                   ( 5.1882229137192883, 0.002351917521465)
                   };
     \addlegendentry{$\approx F(E_i)$}
    \end{axis}
\end{tikzpicture}

  \caption{}
  \label{}
\end{figure}



\begin{equation}
  F^{(n)} (\omega) =
  \begin{cases}
    0                                & \omega < \omega_1\\
    \sum\limits_{j=1}^{i} f_j        & \omega_i < \omega < \omega_{i+1}\\
    \sum\limits_{j=1}^{i} f_j = S(0) & \omega_n < \omega
  \end{cases}
\end{equation}

\begin{equation}
  F^{(n)} (\omega_i) = \frac 12 \left[ F^{(n)} (\omega_i - 0)
                       + F^{(n)} (\omega_i+0) \right]
\end{equation}

\begin{equation}
  f^{(n)} (\omega) =
  \begin{cases}
    \frac 12 \frac{f_1}{\omega_1}    & \omega < \omega_1\\
    \frac 12 \frac{f_{i+1} + f_i}{\omega_{i+1} - \omega_i}
                                     & \omega_i < \omega < \omega_{i+1}\\
    0                                & \omega_n < \omega
  \end{cases}
\end{equation}

%\begin{equation}
%
%\end{equation}
%
%\begin{equation}
%
%\end{equation}
%
%\begin{equation}
%
%\end{equation}
%
%\begin{equation}
%
%\end{equation}
%
%\begin{equation}
%
%\end{equation}
%
%\begin{equation}
%
%\end{equation}
%
%\begin{equation}
%
%\end{equation}
%
%\begin{equation}
%
%\end{equation}
%
%\begin{equation}
%
%\end{equation}
%
%\begin{equation}
%
%\end{equation}
