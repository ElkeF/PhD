\subsection{Moment Problem}

The moments $S(k)$ of a real and continuous function $f(\omega)$ are defined
as

\begin{equation}
  S(k) = \int\limits_a^b \omega^k f(\omega) d\omega \quad\quad k=0,1,\dots  .
\end{equation}

In case of $f(\omega)$ being a probability density function, it is connected
to the probability distribution function $F(\omega)$ via
\begin{equation}
  F(\omega) = f(\omega){d\omega} .
\end{equation}

The probability density function is completely determined by the manifold
of moments. Therefore, when all moments are known, the probability density
function can be calculated from the moments. In the present case $f(\omega)$
is the decay width $\Gamma(E)$, but the theory is also applicable and very often
used for the description of cross sections. Its pseudo-spectrum has the same
mathematical properties as the pseuso-spectrum of the decay width. Therefore,
the knowledge obtained in the description of cross sections can be adopted to
the description for the decay widths.

In practice, all moments are never available unless the moments can be
calculated analytically. Therefore, one has to approximately solve the reduced
moment problem, since the density function is not completely defined.
In this case the $2r$ moments are

\begin{equation}
  S(k) = \int\limits_a^b \omega^k f(\omega) d\omega \quad\quad k=0,1,...,2r-1
\end{equation}





\subsection{Gaussian Quadrature}
The gaussian quadrature is a numerical method for integration. By approximating
the function to be integrated $g(x) = \rho(x) f(x)$ to be a product of a
positive definite weight
function $\rho (x)$ and a continuous and bounded function $f(x)$.

The evaluation of the integral is then desired to be obtained as

\begin{equation}
  \int\limits_a^b \rho(x) f(x) dx \approx \sum\limits_{i=1}^n \omega_i f(x_i)
\end{equation}

, where the weigths $\omega_i$ and the abcissae $x_i$ are to be determined
analytically, if possible, or otherwise in an optimal way. This leads the abcissae
to be unequally spaced unlike in the basic
integration schemes using the trapezoidal rule.

$f(x)$ can be expressed as a polynomial. For certain weight functions and boundaries
of integration, the
polynomials can be determined analytically.
It can furthermore be shown, that the roots (zeros) of the highest order polynomial
describing $f(x)$ give the optimal abcissae $x_i$.

In case of
$\omega(x)= \frac{1}{\sqrt{1-x^2}}$ and the condition

\begin{equation}
  \int\limits_{-1}^{1} \rho(x) Q_n(x) Q_m(x) = N_n \delta_{nm}
\end{equation}

, where $N_n$ denotes the normalization factor,
the solution to the polynomials are the so-called Chebyshev polynomials, with

\begin{equation}
  x_{i,n} = \cos \left( \frac{2i-1}{2n} \pi \right)
  \quad\quad \omega_{i,n} = \frac \pi n .
\end{equation}

This means, that the integration of an arbitrary function $h(x)$ within
the interval $[-1,1]$ can be integrated as

\begin{equation}
  \int\limits_{-1}^1 h(x) dx = \int\limits_{-1}^1 \omega(x) \sqrt{1-x^2} h(x) dx
  \approx \frac \pi n \sum\limits_{i_1}^n h(x_i) \sqrt{1-x_i^2}
\end{equation}

\begin{figure}[h]
  \centering
  \begin{tikzpicture}
    \begin{axis}[%scale=0.8,
                 domain=-1.0:1.0,
                 samples = 200,
                 %xtick={-3.14159,-1.57089,...,3.14159},
                 %xticklabels={$-\pi$,$-\frac \pi 2$,0,$\frac \pi 2$,$\pi$},
                 cycle list name = exotic,
                 legend style={anchor= north west},
                 legend cell align = left
                 ]
     \addplot+[domain=-1:-0.965925826289+0.135517335117/2,
              diplom1,
              mark = none,
              %forget plot,
              pattern = north east lines,
              pattern color = diplom1
              ]
              {0.0987789349866} \closedcycle;
     \addlegendentry{approximation of the integral}
     \addplot+[domain=-0.965925826289+0.135517335117/2:-0.707106781187+0.370240244847/2,
              diplom1,
              mark = none,
              forget plot,
              pattern = north east lines,
              pattern color = diplom1
              ]
              {0.646446609407} \closedcycle;
     \addplot+[domain=-0.707106781187+0.370240244847/2:-0.258819045103+0.505757579964/2,
              diplom1,
              mark = none,
              forget plot,
              pattern = north east lines,
              pattern color = diplom1
              ]
              {0.982662411472} \closedcycle;
     \addplot+[domain=-0.258819045103+0.505757579964/2:0.258819045103+0.505757579964/2,
              diplom1,
              mark = none,
              forget plot,
              pattern = north east lines,
              pattern color = diplom1
              ]
              {1.017337588535} \closedcycle;
     \addplot+[domain=0.258819045103+0.505757579964/2:0.708718898022+0.370240244847/2,
              diplom1,
              mark = none,
              forget plot,
              pattern = north east lines,
              pattern color = diplom1
              ]
              {1.353553390592} \closedcycle;
     \addplot+[domain=0.708718898022+0.370240244847/2:1.0,
              diplom1,
              mark = none,
              forget plot,
              pattern = north east lines,
              pattern color = diplom1
              ]
              {1.901221065013} \closedcycle;
     \addplot [diplom2, thick]
              {x^3 + 1};
     \addlegendentry{$f(x)= x^3 + 1$}
     %\addplot [diplom3, thick]
     %         {x^4/4 + x};
     \addplot [only marks,mark=o,thick]
       coordinates {
                   ( 0.965925826289, 1.901221065013 )
                   ( 0.707106781187, 1.353553390592 )
                   ( 0.258819045103, 1.017337588535 )
                   (-0.258819045103, 0.982662411472 )
                   (-0.707106781187, 0.646446609407 )
                   (-0.965925826289, 0.0987789349866)
                   };
     \addlegendentry{$f_i(x_i)$}
    \end{axis}
\end{tikzpicture}

  \caption{Integration by Gauss-Chebyshev quadrature of the function
           $f(x)=x^3 + 1$ (dark blue). The integral (light blue) is approximately
           obtained by summation
           over all product of optimal abcissae and weights $x_if(x_i)$ (circles).}
  \label{figure:gaussian_quadrature}
\end{figure}

An example for such an integration is shown in figure \ref{figure:gaussian_quadrature}
for $h(x) = x^3 + 1$. The dark blue curve shows $h(x)$, the points are the
calculated $f_i$ at the abcissae $x_i$ and the light blue hatched areas are the
approximations to the integral for the certain areas.

Vice versa this means, that if $\omega(x) = \frac {1}{\sqrt{1-x^2}}$ reasonably
well describes a weight function or probability density function of interest,
the corresponding probability density function can be obtained, if on some basis,
Chebyshev polynomials
can be constructed.


\subsection{Combination of Moment Theory with Gaussian Quadrature}
The procedure for the calculation of cross sections combining moment
theory and gaussian quadrature has been investigated thoroughly. In this section
we follow the argumentation of Müller-Plathe \cite{}, from which the
\verb|stieltjes| routine has been written by Averbukh and which is used in
combination with the FanoADC implemented in Dirac.

For the ionization cross sections it has been shown, that the moment with
$k>2$ diverge and hence are useless for the evaluation of the probability
density function $f(\omega)$. Therefore the inverse moment $S(-k)$ are investigated
instead.

\begin{equation}
  S(-k) = \int\limits_a^b \left( \frac{1}{\omega} \right) ^k f(\omega) d\omega
\end{equation}

Each order of moment can be assigned to a set of Chebyshev polynomials
$Q_n (1/\omega) = \sum\limits_{i=0}^n Q_n^{i}\left( \frac{1}{\omega} \right)^{i}$,
of order $0$ -- $r$. They are orthogonal with respect to the weight function
to be determined $f(\omega)$.

\begin{equation}
  \int\limits_a^b Q_n(1/\omega) \, Q_m(1/\omega) f(\omega) d\omega = N_n \delta_{nm}
\end{equation}

They are normalized such, that the coefficient of the highest power polynomial
equals 1.

\begin{equation}
  N_n = \int\limits_a^b \left[ Q_n(1/\omega) \right]^2 f(\omega) d\omega
\end{equation}

Chebyshev polynomials in general can be constructed from an recursion formula
\begin{equation}
  Q_n(1/\omega) = \frac{1}{\omega - a_n} Q_{n-1}(1/\omega) - b_{n-1} Q_{n-2}(1/\omega)
\end{equation}

, so that all polynomials can be constructed if $Q_0$ and $Q_1$ are known.
From these recursion relations, expressions for the recursion coefficients
$a_n$ and $b_n$ can be obtained.

\begin{align}
  a_n     &= \frac{1}{b_0b_1\cdots b_{n-1}}
             \int (1/\omega)^n Q_{n-1}(1/\omega) f(\omega) d\omega
             - \sum\limits_{l=1}^{n-1} a_l  \label{equation:an_cont}\\
  b_{n-1} &= \frac{1}{b_0b_1\cdots b_{n-2}}
             \int (1/\omega)^{n-1} Q_{n-1}(1/\omega) f(\omega) d\omega \label{equation:bn_cont}
\end{align}

By expansion of the integral in equations \ref{equation:an_cont} and
\ref{equation:bn_cont} into a sum over moments obtained from the pseudo-spectra,
approximate expressions can be obtained for the recursion coefficients
depending on the energies $\bar{\omega}_i$, here the inverse abcissae,
and decay widths or the weights $\bar{f}_i$ of the
pseudo-spectrum.

\begin{align}
  a_n     &= \frac{1}{b_0b_1\cdots b_{n-1}}
             \sum\limits_{i=1}^N
               (1/\bar{\omega}_i)^n Q_{n-1}(1/\bar{\omega_i}) \bar{f}_i
             - \sum\limits_{l=1}^{n-1} a_l \\
  b_{n-1} &= \frac{1}{b_0b_1\cdots b_{n-2}}
             \sum\limits_{i=1}^N
               (1/\bar{\omega}_i)^{n-1} Q_{n-1}(1/\bar{\omega}_i) \bar{f}_i
\end{align}

Hence the recursion relation now reads as
\begin{equation}
  Q_n(1/\bar{\omega}_i) = \frac{1}{\bar{\omega}_i - a_n} Q_{n-1}(1/\bar{\omega}_i)
                          - b_{n-1} Q_{n-2}(1/\bar{\omega}_i)
\end{equation}

with
\begin{equation}
  Q_0(1/\bar{\omega}_i) = 1 \quad\quad Q_1(1/\bar{\omega}_i) = (1/\bar{\omega}_i) - a_1
\end{equation}
as starting points of the determination of the polynomials.

As discussed above in section \ref{section:}, given the weights, the polynomials
can be calculated. In the Stieltjes imaging, starting from the polynomials
defined by the discrete pseudo-spectrum, one wants to achieve the weight function.
Also here the ideal abscissae are the roots of the highest order polynomial
for each moment $k$

\begin{equation}
  Q_n(1/\omega_i) = 0 \quad\quad i = 1,2,\dots ,n .
\end{equation}

The connection between the weights with the polynomials is given by

\begin{equation}
  f_i = \left[ \sum\limits_{m=0}^{n-1} \frac{Q_m^2(1/\omega_i)}{N_m} \right]^{-1} .
\end{equation}

In order to obtain the roots of the highest order polynomial, in principle
any programm for root detection can be used. However, the problem can
be reformulated in the means of general polynomials $R_n(1/\omega)$ connected
to the obtained Chebyshev polynomials
\begin{equation}
  Q_n(1/\omega) = (-1)^n \sqrt{N_n} R_n(1/\omega)
\end{equation}

with the recursion formulas
\begin{equation}
  (1/\omega)R_{n-1}(1/\omega) = - \sqrt{b_n}R_n(1/\omega) + a_nR_{n-1}(1/\omega)
                                - \sqrt{b_{n-1}} R_{n-2}(1/\omega)
\end{equation}
and
\begin{equation}
  (1/\omega)R_0(1/\omega) = - \sqrt{b_1}R_1(1/\omega) + a_1 R_0(1/\omega) .
\end{equation}

Now the roots can be determined of the following equation disregarding the
last vector by solving the eigenvalue problem.

\begin{equation}
 \begin{split}
 \begin{pmatrix}
a_1        & -\sqrt{b_1}&            &                &             &          \\
-\sqrt{b_1}& a_2        & -\sqrt{b_2}&                &             &          \\
           & -\sqrt{b_2}& a_3        & -\sqrt{b_3}    &             &          \\
           &            & \ddots     & \ddots         & \ddots      &          \\
           &            &            & -\sqrt{b_{n-2}}& a_{n-1}     & -\sqrt{b_{n-1}}\\
           &            &            &                & -\sqrt{b_{n-1}}& a_n   
 \end{pmatrix}
 \begin{pmatrix}
  R_0(1/\omega)\\
  R_1(1/\omega)\\
  R_2(1/\omega)\\
  \vdots\\
  R_{n-2}(1/\omega)\\
  R_{n-1}(1/\omega)
 \end{pmatrix}         \\
 = (1/\omega)
 \begin{pmatrix}
  R_0(1/\omega)\\
  R_1(1/\omega)\\
  R_2(1/\omega)\\
  \vdots\\
  R_{n-2}(1/\omega)\\
  R_{n-1}(1/\omega)
 \end{pmatrix}
 -
 \begin{pmatrix}
  0\\
  0\\
  0\\
  \vdots\\
  0\\
  -\sqrt{b_n} R_{n}(1/\omega)
 \end{pmatrix}
 \end{split}
\end{equation}

Therefore the solution is simplyfied to a matrix diagonalization
of the coefficients matrix. Its eigenvalues are the roots of the polynomial
and hence the wanted abcissae. The eigenfunctions are normalized to 1 and therefore
have to be renormalized according to

\begin{equation}
  1 = f_i \sum\limits_{m=0}^{n-1} R_m^2 (1/\omega_i) = \mathbf{u_i} \cdot \mathbf{u_i}
\end{equation}
from equation \ref{} to give

\begin{equation}
  f_i = N_0 u_{0i}^2
\end{equation}

for the weights.

\begin{equation}
  S(-k) = \sum\limits_{i=1}^n f_i (1/\omega_i)^k \quad\quad k=0,1,\dots,2n-1
\end{equation}





\subsection{Stieltjes Imaging}
Having obtained the abcissae and weights, the probability distribution function
$F(\omega)$ can be approximated. For this purpose the so-called Stieltjes imaging
employed, where

\begin{equation}
  F^{(n)} (\omega) =
  \begin{cases}
    0                                & \omega < \omega_1\\
    \sum\limits_{j=1}^{i} f_j        & \omega_i < \omega < \omega_{i+1}\\
    \sum\limits_{j=1}^{i} f_j = S(0) & \omega_n < \omega 
  \end{cases}
\end{equation}
which is illustrated in figure \ref{figure:stieltjes_imaging} for a sixth
order stieltjes procedure using a pseudo-spectrum for the NeAr ICD in a
Stieltjes histogram.


\begin{figure}[h]
  \centering
  %NeAr at 3.42 AA, 6th order of stieltjes
\begin{tikzpicture}
    \begin{axis}[%scale=0.8,
                 domain=-1.0:1.0,
                 samples = 200,
                 %xtick={-3.14159,-1.57089,...,3.14159},
                 %xticklabels={$-\pi$,$-\frac \pi 2$,0,$\frac \pi 2$,$\pi$},
                 cycle list name = exotic,
                 legend style={anchor= north west},
                 legend cell align = left,
                 xlabel= {$E$ [a.u.]},
                 ylabel= {$F(E)$}
                 ]
     \addplot+[domain=-1:-0.42789448546292908,
              diplom1,
              mark = none,
%              forget plot,
              pattern = north east lines,
              pattern color = diplom1
              ]
              {0.0} \closedcycle;
     \addlegendentry{$\sum\limits_{j=1}^{i} f_j \quad \omega_i < \omega < \omega_{i+1}$}
     \addlegendimage{empty legend}
     \addlegendentry{}
     \addplot+[domain=-0.42789448546292908:-0.25969072398345394,
              diplom1,
              mark = none,
              forget plot,
              pattern = north east lines,
              pattern color = diplom1
              ]
              {0.000790314507333} \closedcycle;
     \addplot+[domain=-0.25969072398345394:0.11385089684626037,
              diplom1,
              mark = none,
              forget plot,
              pattern = north east lines,
              pattern color = diplom1
              ]
              {0.00156501308103} \closedcycle;
     \addplot+[domain=0.11385089684626037:1.4305018410002852,
              diplom1,
              mark = none,
              forget plot,
              pattern = north east lines,
              pattern color = diplom1
              ]
              {0.00208878898366} \closedcycle;
     \addplot+[domain=1.4305018410002852:5.1882229137192883,
              diplom1,
              mark = none,
              forget plot,
              pattern = north east lines,
              pattern color = diplom1
              ]
              {0.0022929712885} \closedcycle;
     \addplot+[domain=5.1882229137192883:6,
              diplom1,
              mark = none,
              forget plot,
              pattern = north east lines,
              pattern color = diplom1
              ]
              {0.00241086375443} \closedcycle;
%     \addplot [diplom2, thick,
%               domain=-0.5:6]
%              %{0.001300 * ln(x+2.594992)};
%              {0.001049 * sqrt(x+1.374436)};
%     \addlegendentry{$F(x)= \frac 13 x^3 + x^2 + 2x$}
     \addplot [samples=200,mark=*,thick,smooth,diplom2]
       coordinates {
                   (-0.42789448546292908, 0.0003951572536665)
                   (-0.25969072398345394, 0.0011776637941815)
                   ( 0.11385089684626037, 0.001826901032345)
                   ( 1.4305018410002852, 0.00219088013608)
                   ( 5.1882229137192883, 0.002351917521465)
                   };
     \addlegendentry{$\approx F(E_i)$}
    \end{axis}
\end{tikzpicture}

  \caption{Stieltjes histogram of a sixth order integration from
           an NeAr ICD pseudo-spectrum (light blue). At the abcissae $\omega_i$,
           the histogram provides lower and upper bounds for the actual
           values. The mean of these two bounds (dark blue) normally is a good
           approximation of the distribution function $F(E)$ at this point.}
  \label{figure:stieltjes_imaging}
\end{figure}

This procedure is based on the so-called Chebyshev inequalities

\begin{equation} \label{equation:Chebyshev_inequalities}
  F^{(n)}(\omega_i - 0) \le F^{(n+1)}(\omega_i - 0) \le F(\omega_i)
  \le F^{(n+1)}(\omega_i + 0) \le F^{(n)}(\omega_i + 0).
\end{equation}

This means, that the distribution functions obtained from the Chebyshev
polynomials approaching the abcissae $\omega_i$ from below and above
give lower and upper bounds to the actual value of the distribution
function at this particular point $F(\omega_i)$. In fact, the mean of these
two values normally is a very good approximation to the exact value.

\begin{equation}
  F^{(n)} (\omega_i) = \frac 12 \left[ F^{(n)} (\omega_i - 0)
                       + F^{(n)} (\omega_i+0) \right]
\end{equation}

This distribution function is then numerically differenciated via

\begin{equation}
  f^{(n)} (\omega) =
  \begin{cases}
    \frac 12 \frac{f_1}{\omega_1}    & \omega < \omega_1\\
    \frac 12 \frac{f_{i+1} + f_i}{\omega_{i+1} - \omega_i}
                                     & \omega_i < \omega < \omega_{i+1}\\
    0                                & \omega_n < \omega
  \end{cases}
\end{equation}

to give points of the desired density function $f(\omega)$, which are
subsequently interpolated. In the routine of Averbukh, a spline interpolation
is used for this purpose.

\subsection{Quality and Stability of the Results}
The abcissae of the ploynomials constructed from different orders
of moments intersect each other. Therefore, the combination of all abcissae
and weights for the interpolation is beneficial.

As can be seen from the Chebyshev inequalities in
equation \ref{equation:Chebyshev_inequalities}, the higher the highest
power of the polynomials and hence the larger the degrees of freedom are,
the closer the lower and upper bounds get to the exact result.

Unfortunately, the moment problem is ill-conditioned and by introducing
the polynomials, the moment problem as such get well-conditioned but instead
the construction of the polynomials from the pseudo-spectrum is ill-conditioned.
As can be seen from the Chebyshev inequalities, one would like to go to as
high orders as possible, to get more accurate results. But in the construction
of the recursion coefficients \ref{equation:an_disc}, two very large numbers
are subtracted from each other. This is known to be numerically instable.
And the higher the order of the moment is, the bigger
are these numbers and hence the introduced error. Therefore the number
of moments to be successfully employed for the approximation of the
density function is limited by the quality of the orthogonality of
the corresponding set of constructed polynomials.
