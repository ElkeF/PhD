\begin{appendix}

\chapter{Mathematical Appendix}
\section{Cauchy distribution} \label{section:app_cauchy}
The Cauchy distribution is a continuous probability distribution
with the probability density function

\begin{equation}
  f(x;x_0,\gamma) = \frac 1\pi \frac{\gamma}{(x-x_0)^2 + \gamma^2}
\end{equation}
where $x_0$ denotes the location parameter of the peak and
$2 \gamma$ is the full width half maximum (FWHM).\\
Its height or amplitude is $A = \frac{1}{\pi\gamma}$.
In most cases the
Cauchy distribution is normalized to 1. However, it is easy to show,
that $\gamma$ is completely
independent of the normalization and more generally of any prefactor
as long as the structure of the denominator is conserved.
Other names of the Cauchy distribution are Lorentz distribution or
Lorentzian function.


\begin{figure}[h]
  \centering
  \input{pics/cauchy_pgf}
  \caption{Probability density function of a Cauchy distribution with a
           maximum at $x_0$ with a height of $A=\frac{1}{\pi\gamma}$.}
  \label{figure:cauchy_distribution}
\end{figure}


In physics very often the following three parameter Lorentzian function

\begin{equation}
  f(x;x_0,\gamma,A) = A \frac{\gamma^2}{(x-x_0)^2 + \gamma^2}
\end{equation}
is used. As stated above and easily to show, this reformulation does
not effect the value of the FWHM $2 \gamma$.



\end{appendix}
