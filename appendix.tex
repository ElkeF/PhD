\begin{appendix}

%\chapter{Mathematical Appendix}
%\section{Cauchy distribution} \label{section:app_cauchy}
%The Cauchy distribution is a continuous probability distribution
%with the probability density function
%
%\begin{equation}
%  f(x;x_0,\gamma) = \frac 1\pi \frac{\gamma}{(x-x_0)^2 + \gamma^2}
%\end{equation}
%where $x_0$ denotes the location parameter of the peak and
%$2 \gamma$ is the full width half maximum (FWHM).\\
%Its height or amplitude is $A = \frac{1}{\pi\gamma}$.
%In most cases the
%Cauchy distribution is normalized to 1. However, it is easy to show,
%that $\gamma$ is completely
%independent of the normalization and more generally of any prefactor
%as long as the structure of the denominator is conserved.
%Other names of the Cauchy distribution are Lorentz distribution or
%Lorentzian function.
%
%
%\begin{figure}[h]
%  \centering
%  \input{pics/cauchy_pgf}
%  \caption{Probability density function of a Cauchy distribution with a
%           maximum at $x_0$ with a height of $A=\frac{1}{\pi\gamma}$.}
%  \label{figure:cauchy_distribution}
%\end{figure}
%
%
%In physics very often the following three parameter Lorentzian function
%
%\begin{equation}
%  f(x;x_0,\gamma,A) = A \frac{\gamma^2}{(x-x_0)^2 + \gamma^2}
%\end{equation}
%is used. As stated above and easily to show, this reformulation does
%not effect the value of the FWHM $2 \gamma$.



\chapter{Properties of Noble Gas Atoms}

\begin{table}[h!]
 \caption{Atomic, experimental ionization energies.
          The non-relativistic ionization energies of the
          $np$ were estimated by the weighted average of the experimental
          ionization energies of the $np_{3/2}$ and $np_{1/2}$.}
 \centering
 \begin{tabular}{lcrcrcr}
  \bottomrule
     & \multicolumn{2}{c}{$SIP(np_{3/2})$} & \multicolumn{2}{c}{$SIP(np_{1/2})$} & \multicolumn{2}{c}{$SIP(ns_{1/2})$} \\
  \midrule
   Ne& \unit[21.5645]{eV} & \cite{NIST2014} & \unit[21.6613]{eV} & \cite{NIST2014} & \unit[48.475]{eV} & \cite{NIST2014}\\
   Ar& \unit[15.7596]{eV} & \cite{NIST2014} & \unit[15.9371]{eV} & \cite{NIST2014} & \unit[29.239]{eV} & \cite{NIST2014}\\
   Xe& \unit[12.1298]{eV} & \cite{NIST2014} & \unit[13.4363]{eV} & \cite{NIST2014} & \unit[23.397]{eV} & \cite{Lauer99} \\
  \midrule
   Ne$_{nrel}$ & \multicolumn{4}{c}{\unit[21.5968]{eV}} & \unit[48.475]{eV} &   \\
   Ar$_{nrel}$ & \multicolumn{4}{c}{\unit[15.8188]{eV}} & \unit[29.239]{eV} &   \\
   Xe$_{nrel}$ & \multicolumn{4}{c}{\unit[12.5652]{eV}} & \unit[23.397]{eV} &   \\
  \bottomrule
 \end{tabular}
 \label{table:noble_atom_ionization}
\end{table}

\begin{table}[h!]
 \caption{Lifetimes and relative ionization cross sections of
          noble gas atoms.}
 \centering
 \begin{tabular}{lcrcrcr}
  \toprule
      & \multicolumn{2}{c}{$\tau(ns_{1/2})$} & \multicolumn{2}{c}{$\chi=\frac{\tau_{1/2}}{\tau_{3/2}}$} & \multicolumn{2}{c}{$\frac{\sigma_{3/2}}{\sigma_{1/2}}$} \\
  \midrule
   Ne & \unit[1.429]{ns} & \cite{Lauer99} & 2.04 & \cite{Jans97} & 2.0   & \\
   Ar & \unit[4.684]{ns} & \cite{Lauer99} & 2.05 & \cite{Jans97} & 1.875 & \cite{Codling80} \\
   Xe & \unit[35.93]{ns} & \cite{Lauer99} & 8.40 & \cite{Luyken72} & 1.6 & \cite{Krause81} \\
  \bottomrule
 \end{tabular}
 \label{table:noble_atom_properties}
\end{table}

\begin{table}[h!]
 \caption{Shift of atomic ionization energies due to a cluster environment
          \cite{Feifel04}. All values are given in eV.}
 \centering
 \begin{tabular}{lcccc}
  \toprule
       & {$\Delta(np_{3/2})$} & {$\Delta(np_{1/2})$} & \multicolumn{2}{c}{$\Delta(ns_{1/2})$} \\
       &         &         & bulk    & surface \\
  \midrule
   Ar  & -1.0    & -0.4 & -0.636  & -0.313  \\
   Xe  & -1.3    & -0.9 & -0.756  & -0.346  \\
  \bottomrule
 \end{tabular}
 \label{table:cluster_shifts}
\end{table}







\chapter{NeAr Cluster Structure Agreement Plots}
%\section{Complete Shells}
\begin{figure}[h]
    \centering
    \input{pics/near_clusters/schale02}
    \caption{Complete neon shells with $c=2$.}
    \label{compl02}
\end{figure}

\begin{figure}
    \centering
    \input{pics/near_clusters/schale03}
    \caption{Complete neon shells with $c=3$.}
    \label{compl03}
\end{figure}

\begin{figure}[h]
    \centering
    \input{pics/near_clusters/schale04}
    \caption{Complete neon shells with $c=4$.}
    \label{compl04}
\end{figure}

%\clearpage


%\section{Incomplete Shells Surrounding Complete Shells}
%\subsection{No Complete Neon Shells}
\begin{figure}[h]
    \centering
    \input{pics/near_clusters/in-shell00-core02}
    \caption{Incomplete shells with $c=2$ and no complete neon shell.}
    \label{incompl00-core02}
\end{figure}


%\subsection{One Complete Neon Shell}
\begin{figure}[h]
    \centering
    \input{pics/near_clusters/in-shell01-core02}
    \caption{Incomplete shells with $c=2$ and one complete neon shell.}
    \label{incompl01-core02}
\end{figure}

\begin{figure}
    \centering
    \input{pics/near_clusters/in-shell01-core03}
    \caption{Incomplete shells with $c=3$ and one complete neon shell.}
    \label{incompl01-core03}
\end{figure}

\begin{figure}[h]
    \centering
    \input{pics/near_clusters/in-shell01-core04}
    \caption{Incomplete shells with $c=4$ and one complete neon shell.}
    \label{incompl00-core01}
\end{figure}

\begin{figure}
    \centering
    \input{pics/near_clusters/in-shell01-core05}
    \caption{Incomplete shells with $c=5$ and one complete neon shell.}
    \label{incompl01-core05}
\end{figure}

%\clearpage


%\section{Randomly Arranged Neon Atoms around Complete Shells}
\begin{figure}[h]
    \centering
    \input{pics/near_clusters/random-core02}
    \caption{Random arrangements with $c=2$.}
    \label{random-core02}
\end{figure}

\begin{figure}
    \centering
    \input{pics/near_clusters/random-core03}
    \caption{Random arrangements with $c=3$.}
    \label{random-core03}
\end{figure}

\begin{figure}[h]
    \centering
    \input{pics/near_clusters/random-core04}
    \caption{Random arrangements with $c=4$.}
    \label{random-core04}
\end{figure}


%Programmes
\input{programs}

\end{appendix}
