\begin{appendix}

\chapter{Mathematical Appendix}
\section{Cauchy distribution} \label{section:app_cauchy}
The Cauchy distribution is a continuous probability distribution
with the probability density function

\begin{equation}
  f(x;x_0,\gamma) = \frac 1\pi \frac{\gamma}{(x-x_0)^2 + \gamma^2}
\end{equation}
where $x_0$ denotes the location parameter of the peak and
$2 \gamma$ is the full width half maximum (FWHM).\\
Its height or amplitude is $A = \frac{1}{\pi\gamma}$.
In most cases the
Cauchy distribution is normalized to 1. However, it is easy to show,
that $\gamma$ is completely
independent of the normalization and more generally of any prefactor
as long as the structure of the denominator is conserved.
Other names of the Cauchy distribution are Lorentz distribution or
Lorentzian function.


\begin{figure}[h]
  \centering
   \begin{tikzpicture}[
          scale=1.0,>=stealth,domain=0.5:10,samples=100,
          declare function={
          gamma = 1.0;
          factor = 16.0;
          halfmax = factor * 0.15915;
          x_0 = 5.0;
          distrib(\x) = factor/3.14159 * gamma / ((\x-x_0)^2 + gamma^2);
        }]
%     \tiny
%  \draw[very thin,color=gray] (-0.1,-0.1) grid (4.9,4.9);
  \draw[->,thick] (-0.2,0) -- (10.2,0) node[right] {$x$};
  \draw[->,thick] (0,-0.2) -- (0,7.2) node[above] {$f(x,x_0,\gamma)$};
  % add ticks
  \draw [thick] (5,0) -- (5,-5pt) node [anchor=north] {$x_0$};

  \draw [color=black,domain=0:10,smooth,very thick]    plot
         (\x,{distrib(\x)});% node [anchor=south] {Cauchy distribution};
  \draw [-,very thick,diplom1] (4.0,halfmax) -- (6,halfmax)
         node [anchor=south west] {$2\gamma$};
 \end{tikzpicture}

  \caption{Probability density function of a Cauchy distribution with a
           maximum at $x_0$ with a height of $A=\frac{1}{\pi\gamma}$.}
  \label{figure:cauchy_distribution}
\end{figure}


In physics very often the following three parameter Lorentzian function

\begin{equation}
  f(x;x_0,\gamma,A) = A \frac{\gamma^2}{(x-x_0)^2 + \gamma^2}
\end{equation}
is used. As stated above and easily to show, this reformulation does
not effect the value of the FWHM $2 \gamma$.


\chapter{NeAr Cluster Structure Agreement Plots}
\section{Complete Shells}
\begin{figure}[!h]
    \centering
    
                \begin{tikzpicture}%[scale=.55]
                    \begin{axis}[
                        use units,
                        x unit=\%,
                        y unit=\%,
                        xmin=0,
                        xmax=100,
                        %axis x line=bottom,
%                       axis x discontinuity=parallel,
                        ymin=0,
                        ymax=100,
                        %axis y line=left,
                        samples=1000,
                        schale=1.0,
                        legend style={draw=none,font=\tiny},
                        legend cell align=center,
                        legend pos=south east,
                        axis line style={-}
                        ]
                    \addplot[
                        only marks,
                        mark=10-pointed star,
                        color=blue!50!black,
                        error bars,
                        y dir=both,
                        y explicit
                        ]
                        table[
                        x expr=\thisrowno{0},
                        y expr=\thisrowno{1}
                        %y error expr=\thisrowno{2}
                        ]
                        {data/near_clusters/schale02.csv};
                        \addlegendentry{1 - 5 layers}
                    \filldraw [black,thin,opacity=0.15] (axis cs:37,56) rectangle (axis cs:57,70);
\filldraw [green,thin,opacity=0.25] (axis cs:30,51) rectangle (axis cs:40,71);
\filldraw [blue,thin,opacity=0.25] (axis cs:17,33) rectangle (axis cs:25,43);
\filldraw [cyan,thin,opacity=0.15] (axis cs:16,47) rectangle (axis cs:24,59);
\filldraw [red,thin,opacity=0.15] (axis cs:6,25) rectangle (axis cs:10,31);

\addplot[
            only marks,
            mark=+,
            clip mode=individual,
            color=black,
            error bars,
            x dir=both,
            x explicit,
            y dir=both,
            y explicit
            ]
            coordinates{
            (47,63) +-  (10,7)
            (35,61) +-  (5,10)
            (21,38) +-  (4,5)
            (20,53) +-  (4,6)
            (8,28) +-  (2,3)
            };
            \addlegendentry{experimental data}

                    \end{axis}
                \end{tikzpicture}

    \caption{Complete shells with $c=2$.}
    \label{compl02}
\end{figure}

\begin{figure}
    \centering
    
                \begin{tikzpicture}%[scale=.55]
                    \begin{axis}[
                        use units,
                        x unit=\%,
                        y unit=\%,
                        xmin=0,
                        xmax=100,
                        %axis x line=bottom,
%                       axis x discontinuity=parallel,
                        ymin=0,
                        ymax=100,
                        %axis y line=left,
                        samples=1000,
                        schale=1.0,
                        legend style={draw=none,font=\tiny},
                        legend cell align=center,
                        legend pos=south east,
                        axis line style={-}
                        ]
                    \addplot[
                        only marks,
                        mark=10-pointed star,
                        color=blue!50!black,
                        error bars,
                        y dir=both,
                        y explicit
                        ]
                        table[
                        x expr=\thisrowno{0},
                        y expr=\thisrowno{1}
                        %y error expr=\thisrowno{2}
                        ]
                        {data/near_clusters/schale03.csv};
                        \addlegendentry{1 - 5 layers}
                    \filldraw [black,thin,opacity=0.15] (axis cs:37,56) rectangle (axis cs:57,70);
\filldraw [green,thin,opacity=0.25] (axis cs:30,51) rectangle (axis cs:40,71);
\filldraw [blue,thin,opacity=0.25] (axis cs:17,33) rectangle (axis cs:25,43);
\filldraw [cyan,thin,opacity=0.15] (axis cs:16,47) rectangle (axis cs:24,59);
\filldraw [red,thin,opacity=0.15] (axis cs:6,25) rectangle (axis cs:10,31);

\addplot[
            only marks,
            mark=+,
            clip mode=individual,
            color=black,
            error bars,
            x dir=both,
            x explicit,
            y dir=both,
            y explicit
            ]
            coordinates{
            (47,63) +-  (10,7)
            (35,61) +-  (5,10)
            (21,38) +-  (4,5)
            (20,53) +-  (4,6)
            (8,28) +-  (2,3)
            };
            \addlegendentry{experimental data}

                    \end{axis}
                \end{tikzpicture}

    \caption{Complete shells with $c=3$.}
    \label{compl03}
\end{figure}

\begin{figure}[!h]
    \centering
    
                \begin{tikzpicture}%[scale=.55]
                    \begin{axis}[
                        use units,
                        x unit=\%,
                        y unit=\%,
                        xmin=0,
                        xmax=100,
                        %axis x line=bottom,
%                       axis x discontinuity=parallel,
                        ymin=0,
                        ymax=100,
                        %axis y line=left,
                        samples=1000,
                        scale=0.55,
                        legend style={draw=none,font=\tiny},
                        legend cell align=center,
                        legend pos=south east,
                        axis line style={-}
                        ]
                    \addplot[
                        only marks,
                        mark=10-pointed star,
                        color=blue!50!black,
                        error bars,
                        y dir=both,
                        y explicit
                        ]
                        table[
                        x expr=\thisrowno{0},
                        y expr=\thisrowno{1}
                        %y error expr=\thisrowno{2}
                        ]
                        {data/near_clusters/schale04.csv};
                        \addlegendentry{1 - 5 layers}
                    \filldraw [black,thin,opacity=0.15] (axis cs:37,56) rectangle (axis cs:57,70);
\filldraw [green,thin,opacity=0.25] (axis cs:30,51) rectangle (axis cs:40,71);
\filldraw [blue,thin,opacity=0.25] (axis cs:17,33) rectangle (axis cs:25,43);
\filldraw [cyan,thin,opacity=0.15] (axis cs:16,47) rectangle (axis cs:24,59);
\filldraw [red,thin,opacity=0.15] (axis cs:6,25) rectangle (axis cs:10,31);

\addplot[
            only marks,
            mark=+,
            clip mode=individual,
            color=black,
            error bars,
            x dir=both,
            x explicit,
            y dir=both,
            y explicit
            ]
            coordinates{
            (47,63) +-  (10,7)
            (35,61) +-  (5,10)
            (21,38) +-  (4,5)
            (20,53) +-  (4,6)
            (8,28) +-  (2,3)
            };
            \addlegendentry{experimental data}

                    \end{axis}
                \end{tikzpicture}

    \caption{Complete shells with $c=4$.}
    \label{compl04}
\end{figure}



\section{Incomplete Shells Surrounding Complete Shells}
\subsection{No Complete Neon Shells}
\begin{figure}[h]
    \centering
    
\begin{tikzpicture}%[scale=1]
                    \begin{axis}[
                        use units,
                        x unit=\%,
                        y unit=\%,
                        xmin=0,
                        xmax=100,
                        %axis x line=bottom,
%                       axis x discontinuity=parallel,
                        ymin=0,
                        ymax=100,
                        %axis y line=left,
                        samples=1000,
                        scale=0.55,
                        legend style={draw=none,font=\tiny},
                        legend cell align=center,
                        legend pos=south east,
                        axis line style={-}
                        ]
                    \addplot[
                        only marks,
                        mark=10-pointed star,
                        color=blue!50!black,
                        error bars,
                        y dir=both,
                        y explicit
                        ]
                        table[
                        x expr=\thisrowno{0},
                        y expr=\thisrowno{1}
                        %y error expr=\thisrowno{2}
                        ]
                        {data/near_clusters/in-shell00-core02.csv};
                    \addlegendentry{calculations}
                    \filldraw [black,thin,opacity=0.15] (axis cs:37,56) rectangle (axis cs:57,70);
\filldraw [green,thin,opacity=0.25] (axis cs:30,51) rectangle (axis cs:40,71);
\filldraw [blue,thin,opacity=0.25] (axis cs:17,33) rectangle (axis cs:25,43);
\filldraw [cyan,thin,opacity=0.15] (axis cs:16,47) rectangle (axis cs:24,59);
\filldraw [red,thin,opacity=0.15] (axis cs:6,25) rectangle (axis cs:10,31);

\addplot[
            only marks,
            mark=+,
            clip mode=individual,
            color=black,
            error bars,
            x dir=both,
            x explicit,
            y dir=both,
            y explicit
            ]
            coordinates{
            (47,63) +-  (10,7)
            (35,61) +-  (5,10)
            (21,38) +-  (4,5)
            (20,53) +-  (4,6)
            (8,28) +-  (2,3)
            };
            \addlegendentry{experimental data}

                    \end{axis}
\end{tikzpicture}

    \caption{Incomplete shells with $c=2$ and no complete neon shell.}
    \label{incompl00-core02}
\end{figure}


\subsection{One Complete Neon Shell}
\begin{figure}[h]
    \centering
    
\begin{tikzpicture}%[scale=1]
                    \begin{axis}[
                        use units,
                        x unit=\%,
                        y unit=\%,
                        xmin=0,
                        xmax=100,
                        %axis x line=bottom,
%                       axis x discontinuity=parallel,
                        ymin=0,
                        ymax=100,
                        %axis y line=left,
                        samples=1000,
                        scale=0.55,
                        legend style={draw=none},
                        legend cell align=center,
                        legend pos=south east,
                        axis line style={-}
                        ]
                    \addplot[
                        only marks,
                        mark=10-pointed star,
                        color=blue!50!black,
                        error bars,
                        y dir=both,
                        y explicit
                        ]
                        table[
                        x expr=\thisrowno{0},
                        y expr=\thisrowno{1}
                        %y error expr=\thisrowno{2}
                        ]
                        {data/near_clusters/in-shell01-core02.csv};
                    \addlegendentry{calculations}
                    \filldraw [black,thin,opacity=0.15] (axis cs:37,56) rectangle (axis cs:57,70);
\filldraw [green,thin,opacity=0.25] (axis cs:30,51) rectangle (axis cs:40,71);
\filldraw [blue,thin,opacity=0.25] (axis cs:17,33) rectangle (axis cs:25,43);
\filldraw [cyan,thin,opacity=0.15] (axis cs:16,47) rectangle (axis cs:24,59);
\filldraw [red,thin,opacity=0.15] (axis cs:6,25) rectangle (axis cs:10,31);

\addplot[
            only marks,
            mark=+,
            clip mode=individual,
            color=black,
            error bars,
            x dir=both,
            x explicit,
            y dir=both,
            y explicit
            ]
            coordinates{
            (47,63) +-  (10,7)
            (35,61) +-  (5,10)
            (21,38) +-  (4,5)
            (20,53) +-  (4,6)
            (8,28) +-  (2,3)
            };
            \addlegendentry{experimental data}

%                    \node[pin={[pin distance=1.0cm]110:{\includegraphics[scale=0.12]{near_set4.png}}}]
%                      at (axis cs:20,49) {};
%                    \node[pin={[pin distance=1.0cm]358:{\includegraphics[scale=0.12]{near_set5.png}}}]
%                      at (axis cs:10,20) {};
                    \end{axis}
\end{tikzpicture}

    \caption{Incomplete shells with $c=2$ and one complete neon shell.}
    \label{incompl01-core02}
\end{figure}

\begin{figure}
    \centering
    
\begin{tikzpicture}%[scale=1]
                    \begin{axis}[
                        use units,
                        x unit=\%,
                        y unit=\%,
                        xmin=0,
                        xmax=100,
                        %axis x line=bottom,
%                       axis x discontinuity=parallel,
                        ymin=0,
                        ymax=100,
                        %axis y line=left,
                        samples=1000,
                        schale=1.0,
                        legend style={draw=none,font=\tiny},
                        legend cell align=center,
                        legend pos=south east,
                        axis line style={-}
                        ]
                    \addplot[
                        only marks,
                        mark=10-pointed star,
                        color=blue!50!black,
                        error bars,
                        y dir=both,
                        y explicit
                        ]
                        table[
                        x expr=\thisrowno{0},
                        y expr=\thisrowno{1}
                        %y error expr=\thisrowno{2}
                        ]
                        {data/near_clusters/in-shell01-core03.csv};
                    \addlegendentry{calculations}
                    \filldraw [black,thin,opacity=0.15] (axis cs:37,56) rectangle (axis cs:57,70);
\filldraw [green,thin,opacity=0.25] (axis cs:30,51) rectangle (axis cs:40,71);
\filldraw [blue,thin,opacity=0.25] (axis cs:17,33) rectangle (axis cs:25,43);
\filldraw [cyan,thin,opacity=0.15] (axis cs:16,47) rectangle (axis cs:24,59);
\filldraw [red,thin,opacity=0.15] (axis cs:6,25) rectangle (axis cs:10,31);

\addplot[
            only marks,
            mark=+,
            clip mode=individual,
            color=black,
            error bars,
            x dir=both,
            x explicit,
            y dir=both,
            y explicit
            ]
            coordinates{
            (47,63) +-  (10,7)
            (35,61) +-  (5,10)
            (21,38) +-  (4,5)
            (20,53) +-  (4,6)
            (8,28) +-  (2,3)
            };
            \addlegendentry{experimental data}

                    \end{axis}
\end{tikzpicture}

    \caption{Incomplete shells with $c=3$ and one complete neon shell.}
    \label{incompl01-core03}
\end{figure}

\begin{figure}[h]
    \centering
    
\begin{tikzpicture}%[scale=1]
                    \begin{axis}[
                        use units,
                        x unit=\%,
                        y unit=\%,
                        xmin=0,
                        xmax=100,
                        %axis x line=bottom,
%                       axis x discontinuity=parallel,
                        ymin=0,
                        ymax=100,
                        %axis y line=left,
                        samples=1000,
                        scale=0.55,
                        legend style={draw=none},
                        legend cell align=center,
                        legend pos=south east,
                        axis line style={-}
                        ]
                    \addplot[
                        only marks,
                        mark=10-pointed star,
                        color=blue!50!black,
                        error bars,
                        y dir=both,
                        y explicit
                        ]
                        table[
                        x expr=\thisrowno{0},
                        y expr=\thisrowno{1}
                        %y error expr=\thisrowno{2}
                        ]
                        {data/near_clusters/in-shell01-core04.csv};
                    \addlegendentry{calculations}
                    \filldraw [black,thin,opacity=0.15] (axis cs:37,56) rectangle (axis cs:57,70);
\filldraw [green,thin,opacity=0.25] (axis cs:30,51) rectangle (axis cs:40,71);
\filldraw [blue,thin,opacity=0.25] (axis cs:17,33) rectangle (axis cs:25,43);
\filldraw [cyan,thin,opacity=0.15] (axis cs:16,47) rectangle (axis cs:24,59);
\filldraw [red,thin,opacity=0.15] (axis cs:6,25) rectangle (axis cs:10,31);

\addplot[
            only marks,
            mark=+,
            clip mode=individual,
            color=black,
            error bars,
            x dir=both,
            x explicit,
            y dir=both,
            y explicit
            ]
            coordinates{
            (47,63) +-  (10,7)
            (35,61) +-  (5,10)
            (21,38) +-  (4,5)
            (20,53) +-  (4,6)
            (8,28) +-  (2,3)
            };
            \addlegendentry{experimental data}

                    \end{axis}
\end{tikzpicture}

    \caption{Incomplete shells with $c=4$ and one complete neon shell.}
    \label{incompl00-core01}
\end{figure}

\begin{figure}
    \centering
    
\begin{tikzpicture}%[scale=1]
                    \begin{axis}[
                        use units,
                        x unit=\%,
                        y unit=\%,
                        xmin=0,
                        xmax=100,
                        %axis x line=bottom,
%                       axis x discontinuity=parallel,
                        ymin=0,
                        ymax=100,
                        %axis y line=left,
                        samples=1000,
                        scale=0.55,
                        legend style={draw=none,font=\tiny},
                        legend cell align=center,
                        legend pos=south east,
                        axis line style={-}
                        ]
                    \addplot[
                        only marks,
                        mark=10-pointed star,
                        color=blue!50!black,
                        error bars,
                        y dir=both,
                        y explicit
                        ]
                        table[
                        x expr=\thisrowno{0},
                        y expr=\thisrowno{1}
                        %y error expr=\thisrowno{2}
                        ]
                        {data/near_clusters/in-shell01-core05.csv};
                    \addlegendentry{calculations}
                    \filldraw [black,thin,opacity=0.15] (axis cs:37,56) rectangle (axis cs:57,70);
\filldraw [green,thin,opacity=0.25] (axis cs:30,51) rectangle (axis cs:40,71);
\filldraw [blue,thin,opacity=0.25] (axis cs:17,33) rectangle (axis cs:25,43);
\filldraw [cyan,thin,opacity=0.15] (axis cs:16,47) rectangle (axis cs:24,59);
\filldraw [red,thin,opacity=0.15] (axis cs:6,25) rectangle (axis cs:10,31);

\addplot[
            only marks,
            mark=+,
            clip mode=individual,
            color=black,
            error bars,
            x dir=both,
            x explicit,
            y dir=both,
            y explicit
            ]
            coordinates{
            (47,63) +-  (10,7)
            (35,61) +-  (5,10)
            (21,38) +-  (4,5)
            (20,53) +-  (4,6)
            (8,28) +-  (2,3)
            };
            \addlegendentry{experimental data}

                    \end{axis}
\end{tikzpicture}

    \caption{Incomplete shells with $c=5$ and one complete neon shell.}
    \label{incompl01-core05}
\end{figure}



\section{Randomly Arranged Neon Atoms around Complete Shells}
\begin{figure}[h]
    \centering
                    \begin{tikzpicture}%[scale=1]
                    \begin{axis}[
                        use units,
                        x unit=\%,
                        y unit=\%,
                        xmin=0,
                        xmax=100,
                       % axis x line=bottom,
                        %axis x discontinuity=parallel,
                        ymin=0,
                        ymax=100,
                        %axis y line=left,
                        samples=1000,
                        scale=0.55,
                        legend style={draw=none,font=\tiny},
                        legend cell align=center,
                        legend pos=south east,
                        axis line style={-}
                        ]
                    \addplot[
                        only marks,
                        mark=10-pointed star,
                        color=cyan!65!black,
                        error bars,
                        y dir=both,
                        y explicit
                        ]
                        table[
                        x expr=\thisrowno{0},
                        y expr=\thisrowno{1}
                        %y error expr=\thisrowno{2}
                        ]
                        {data/near_clusters/random-core02.csv};
                        \addlegendentry{0 shells}
                    \addplot[
                        only marks,
                        mark=10-pointed star,
                        color=blue!40!black,
                        error bars,
                        y dir=both,
                        y explicit
                        ]
                        table[
                        x expr=\thisrowno{3},
                        y expr=\thisrowno{4},
                        %y error expr=\thisrowno{5}
                        ]
                        {data/near_clusters/random-core02.csv};
                        \addlegendentry{1 shell}
                    \addplot[
                        only marks,
                        mark=10-pointed star,
                        color=red!40!black,
                        error bars,
                        y dir=both,
                        y explicit
                        ]
                        table[
                        x expr=\thisrowno{6},
                        y expr=\thisrowno{7},
                        %y error expr=\thisrowno{8}
                        ]
                        {data/near_clusters/random-core02.csv};
                        \addlegendentry{2 shells}
                    \filldraw [black,thin,opacity=0.15] (axis cs:37,56) rectangle (axis cs:57,70);
\filldraw [green,thin,opacity=0.25] (axis cs:30,51) rectangle (axis cs:40,71);
\filldraw [blue,thin,opacity=0.25] (axis cs:17,33) rectangle (axis cs:25,43);
\filldraw [cyan,thin,opacity=0.15] (axis cs:16,47) rectangle (axis cs:24,59);
\filldraw [red,thin,opacity=0.15] (axis cs:6,25) rectangle (axis cs:10,31);

\addplot[
            only marks,
            mark=+,
            clip mode=individual,
            color=black,
            error bars,
            x dir=both,
            x explicit,
            y dir=both,
            y explicit
            ]
            coordinates{
            (47,63) +-  (10,7)
            (35,61) +-  (5,10)
            (21,38) +-  (4,5)
            (20,53) +-  (4,6)
            (8,28) +-  (2,3)
            };
            \addlegendentry{experimental data}

                \end{axis}
            \end{tikzpicture}

    \caption{Random arrangements with $c=2$.}
    \label{random-core02}
\end{figure}

\begin{figure}
    \centering
                    \begin{tikzpicture}%[scale=1]
                    \begin{axis}[
                        use units,
                        x unit=\%,
                        y unit=\%,
                        xmin=0,
                        xmax=100,
                       % axis x line=bottom,
                        %axis x discontinuity=parallel,
                        ymin=0,
                        ymax=100,
                        %axis y line=left,
                        samples=1000,
                        scale=0.55,
                        legend style={draw=none,font=\tiny},
                        legend cell align=center,
                        legend pos=south east,
                        axis line style={-}
                        ]
                    \addplot[
                        only marks,
                        mark=10-pointed star,
                        color=cyan!65!black,
                        error bars,
                        y dir=both,
                        y explicit
                        ]
                        table[
                        x expr=\thisrowno{0},
                        y expr=\thisrowno{1}
                        %y error expr=\thisrowno{2}
                        ]
                        {data/near_clusters/random-core03.csv};
                        \addlegendentry{0 shells}
                    \addplot[
                        only marks,
                        mark=10-pointed star,
                        color=blue!40!black,
                        error bars,
                        y dir=both,
                        y explicit
                        ]
                        table[
                        x expr=\thisrowno{3},
                        y expr=\thisrowno{4},
                        %y error expr=\thisrowno{5}
                        ]
                        {data/near_clusters/random-core03.csv};
                        \addlegendentry{1 shell}
%                    \addplot[
%                        only marks,
%                        mark=10-pointed star,
%                        color=red!40!black,
%                        error bars,
%                        y dir=both,
%                        y explicit
%                        ]
%                        table[
%                        x expr=\thisrowno{6},
%                        y expr=\thisrowno{7},
%                        %y error expr=\thisrowno{8}
%                        ]
%                        {data/near_clusters/random-core03.csv};
%                        \addlegendentry{2 shells}
                    \filldraw [black,thin,opacity=0.15] (axis cs:37,56) rectangle (axis cs:57,70);
\filldraw [green,thin,opacity=0.25] (axis cs:30,51) rectangle (axis cs:40,71);
\filldraw [blue,thin,opacity=0.25] (axis cs:17,33) rectangle (axis cs:25,43);
\filldraw [cyan,thin,opacity=0.15] (axis cs:16,47) rectangle (axis cs:24,59);
\filldraw [red,thin,opacity=0.15] (axis cs:6,25) rectangle (axis cs:10,31);

\addplot[
            only marks,
            mark=+,
            clip mode=individual,
            color=black,
            error bars,
            x dir=both,
            x explicit,
            y dir=both,
            y explicit
            ]
            coordinates{
            (47,63) +-  (10,7)
            (35,61) +-  (5,10)
            (21,38) +-  (4,5)
            (20,53) +-  (4,6)
            (8,28) +-  (2,3)
            };
            \addlegendentry{experimental data}

                \end{axis}
            \end{tikzpicture}

    \caption{Random arrangements with $c=3$.}
    \label{random-core03}
\end{figure}

\begin{figure}[h]
    \centering
                    \begin{tikzpicture}%[scale=1]
                    \begin{axis}[
                        use units,
                        x unit=\%,
                        y unit=\%,
                        xmin=0,
                        xmax=100,
                       % axis x line=bottom,
                        %axis x discontinuity=parallel,
                        ymin=0,
                        ymax=100,
                        %axis y line=left,
                        samples=1000,
                        scale=0.55,
                        legend style={draw=none,font=\tiny},
                        legend cell align=center,
                        legend pos=south east,
                        axis line style={-}
                        ]
%                    \addplot[
%                        only marks,
%                        mark=10-pointed star,
%                        color=cyan!65!black,
%                        error bars,
%                        y dir=both,
%                        y explicit
%                        ]
%                        table[
%                        x expr=\thisrowno{0},
%                        y expr=\thisrowno{1}
%                        %y error expr=\thisrowno{2}
%                        ]
%                        {data/near_clusters/random-core04.csv};
%                        \addlegendentry{0 shells}
                    \addplot[
                        only marks,
                        mark=10-pointed star,
                        color=blue!40!black,
                        error bars,
                        y dir=both,
                        y explicit
                        ]
                        table[
                        x expr=\thisrowno{3},
                        y expr=\thisrowno{4},
                        %y error expr=\thisrowno{5}
                        ]
                        {data/near_clusters/random-core04.csv};
                        \addlegendentry{1 shell}
                    \addplot[
                        only marks,
                        mark=10-pointed star,
                        color=red!40!black,
                        error bars,
                        y dir=both,
                        y explicit
                        ]
                        table[
                        x expr=\thisrowno{6},
                        y expr=\thisrowno{7},
                        %y error expr=\thisrowno{8}
                        ]
                        {data/near_clusters/random-core04.csv};
                        \addlegendentry{2 shells}
                    \filldraw [black,thin,opacity=0.15] (axis cs:37,56) rectangle (axis cs:57,70);
\filldraw [green,thin,opacity=0.25] (axis cs:30,51) rectangle (axis cs:40,71);
\filldraw [blue,thin,opacity=0.25] (axis cs:17,33) rectangle (axis cs:25,43);
\filldraw [cyan,thin,opacity=0.15] (axis cs:16,47) rectangle (axis cs:24,59);
\filldraw [red,thin,opacity=0.15] (axis cs:6,25) rectangle (axis cs:10,31);

\addplot[
            only marks,
            mark=+,
            clip mode=individual,
            color=black,
            error bars,
            x dir=both,
            x explicit,
            y dir=both,
            y explicit
            ]
            coordinates{
            (47,63) +-  (10,7)
            (35,61) +-  (5,10)
            (21,38) +-  (4,5)
            (20,53) +-  (4,6)
            (8,28) +-  (2,3)
            };
            \addlegendentry{experimental data}

                \end{axis}
            \end{tikzpicture}

    \caption{Random arrangements with $c=4$.}
    \label{random-core04}
\end{figure}


%Programmes
\chapter{Programmes and Scripts}

\section{HARDRoC --- Hunting Asymptotic Relativistic Decay Rates of Clusters}
\section{Relativistic FanoADC}
\section{\textsc{icoclus}}
\section{\textsc{fccclus}}
\section{Utilities}


\end{appendix}
