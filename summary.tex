\chapter{Summary and Outlook}

In this thesis, the importance of relativstic effects on autoionization
processes, especially \ac{ICD}-like
processes, and cluster environments have been discussed.
For this purpose, asymptotic expressions for the relativistic decay width of the ICD
and both, relativistic and non-relativstic asymptotic expressions for the ETMD3
have been derived. Additionally, the non-relativistically known
FanoADC-Stieltjes approach
using a partitioning of the Hamiltonian by 2h configurations has been implemented
in the relativistic programme package Dirac,
which allows for the
description of decay width including relativistic effects.
In order to model the experimental secondary electron spectra of noble gas
clusters, the model of pairs and triples was introduced and automatized in the
programme HARDRoC. It enables the estimation of
decay widths of the total system from data of the compounds.

In the studies of the atomic Auger process, scalar-relativistic effects were
found to increase the decay width compared to the non-relativistic results
due to larger orbital overlaps of the initial and final states. Especially
in ETMD processes, whose decay widths are governed by the orbital overlap
of the two units involved in the energy transfer, similiar significant
decay width influences might be observed.

Throughout all systems containing heavy elements, the spin-orbit coupling
shows a pronounced effect on the secondary electron spectrum by increasing the
number of possible channels and hence, the number of peaks. This feature cannot
be explained using a non-relativistic methodology.
If the decay channels are close to threshold, this energetic splitting
can cause a non-relativistically closed decay channel to be partly open. On
the other hand, not all relativistic channels corresponding to one specific
open non-relativistic channel need to be energetically accessible.

Additionally, geometry has a great impact on the opening and closing of channels
in \ac{ICD}-like processes. The closer the ionized atoms of the final state are,
the lower is the kinetic energy of the secondary electron and at some
internuclear distance the channel closes.

In clusters, the additional effect of charge stabilization shifts the secondary
electron as well. These shifts were treated by using experimentally obtained
ionization energies for exactly the same experimental conditions as for the
secondary electron spectra.
Furthermore, statistical effects were found to increase the decay width
for both the ICD and the ETMD. Since the ICD decay width scales with the number
of nearest neighbours and the ETMD3 decay width scales with the number of nearest
neighbours squared, the ETMD3 is statistically preferred and can therefore compete
with the usually faster ICD.

Based on the strong structure dependence of the secondary electron spectrum
observed during the PhD, a new
structure analysis method for noble gas clusters was developed. If two
competing ICD-like processes are energetically accessible and can be measured
independently, the comparison of the experimentally obtained cluster
composition and relative peak intensities can be compared to theoretically
modelled spectra for a large variety of structures. From the best agreement,
the mean cluster structure can be deduced as was carried out for
a set of experimentally created NeAr cluster manifolds.

In the future, it would be worth investigating \ac{ICD} processes with electron
transitions forbidden in the non-relativistic description.
Additionally, for the ETMD3 the determination of a physically reasonable
choice of the energy transfer distance $R$ would be beneficial.
Furthermore, it should be investigated using the relativistic
FanoADC-Stieltjes approach studying the scalar-relativistic effects observed
for the case of the Auger effect following an ionization of the Xe4d.

It might be worth implementing the FanoADC based on an energy partitioning in
order to obtain partial decay width for the separate final states and to
improve their accuracy.
Furthermore, a more rigorous investigation of the couplings between the
initial state and Rydberg states would be beneficial for the development
of a method to exclude these couplings from the decay width calculation
automatically.\\
In the Stieltjes Imaging, the pseudo-spectra are currently shifted such,
that the lowest energy has some predefined, small number. This number needs
to be small in order to allow for numerical stability for as high order
moments as possible. However, whether there is an optimal choice of this
number and if it exists, what it depends on, is currently unknown.
It would be worth investigating, because an improvement of the Stieltjes
imaging procedure might lead in more reliable results or decay width profiles,
which are easier to interprete, which might allow for a more automatized
evaluation of the results.\\
So far, no automatic error estimation is available in the FanoADC-Stieltjes
approach. However, it would be very helpful for the validation of the
calculated decay widths.
Due to the drawbacks of the FanoADC-Stieltjes method one might want
to consider a completely different ansatz. It should be based on $\mathcal{L}^2$
functions, it should be able to automatically choose and describe the final
states for a given initial state and should provide accurate decay widths
in an automatizable and unambiguos way.

For the estimation of secondary electron spectra of clusters,
the investigation of cluster dynamics and their influence on the
secondary electron spectra should be developed and included in
HARDRoC. This would at the same time allow for an error estimation of
the calculated decay widths.

The investigated structures have so far been chosen by educated guesses.
For a more detailed study, optimized cluster structures would be beneficial.
Furthermore, it would be a challenging task to develop the possibility to
treat clusters with constituents of lower than spherical symmetry, e.g.,
water molecules.
